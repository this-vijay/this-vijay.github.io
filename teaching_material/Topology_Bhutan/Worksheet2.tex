
\documentclass{amsart}
%\usepackage{cmbright}
\usepackage{amsmath, amsfonts, amssymb, amscd}
%\usepackage{tableau}
\usepackage{color}
\usepackage{xcolor}
\usepackage{graphicx}
\usepackage{array}
\usepackage{mathtools}
\usepackage{multirow}
\usepackage{framed}
\usepackage{tikz}
\usetikzlibrary{matrix,arrows}
\usepackage[square,sort,comma,numbers]{natbib}
\usepackage{enumerate}




%\oddsidemargin.5cm
%\evensidemargin.5cm
\addtolength{\oddsidemargin}{-.525in}
\addtolength{\evensidemargin}{-.525in}
\addtolength{\textwidth}{1in}

\addtolength{\topmargin}{-.87in}
\addtolength{\textheight}{1.5in}




\newcommand\coolunder[2]{\mathrlap{\smash{\underbrace{\phantom{%
    \begin{matrix} #2 \end{matrix}}}_{\mbox{$#1$}}}}#2}

%\input{tableau}


\newcommand{\+}[1]{\ensuremath{\mathbf{#1}}}
\newcommand{\vect}[1]{\boldsymbol{#1}} % Uncomment for BOLD vectors.
%\newcommand{\vect}[1]{\vec{#1}} % Uncomment for ARROW vectors.
\newcommand{\C}{{\mathbb C}}
\newcommand{\Z}{{\mathbb Z}}
\renewcommand{\P}{{\mathbb P}}
\newcommand{\OG}{\operatorname{OG}}
\newcommand{\OF}{\operatorname{OF}}
\newcommand{\bull}{{\scriptscriptstyle \bullet}}
\newcommand{\la}{\lambda}
\newcommand{\euler}[1]{\chi_{_{#1}}}
\newcommand{\cO}{{\mathcal O}}
\newcommand{\cG}{{\mathcal G}}
\newcommand{\cQ}{{\mathcal Q}}
\newcommand{\R}{{\mathbb R}}
\newcommand{\wt}{\widetilde}
\newcommand{\diag}{\operatorname{diag}}
\newcommand{\comp}{\operatorname{comp}}
\newcommand{\comment}[1]{}
\newcommand{\type}{\mathfrak{t}}
\newcommand{\op}{\text{op}}
\newcommand{\row}{{\bf r}}
\newcommand{\col}{{\bf c}}
\newcommand{\sym}{\mathfrak{S}}
\newcommand{\codim}{\text{codim}}
\DeclarePairedDelimiter{\ceil}{\lceil}{\rceil}
\DeclarePairedDelimiter{\floor}{\lfloor}{\rfloor}
\renewcommand{\emptyset}{\varnothing}

\newtheorem{thm}{Theorem}
\newtheorem{lemma}[thm]{Lemma}
\newtheorem{prop}[thm]{Proposition}
\newtheorem{cor}[thm]{Corollary}

\theoremstyle{definition}
\newtheorem{definition}[thm]{Definition}
\newtheorem{example}[thm]{Example}
\newtheorem{conj}[thm]{Conjecture}
\newtheorem{obs}[thm]{Observation}
\newtheorem{fact}[thm]{Fact}
\newtheorem{remark}[thm]{Remark}
\newtheorem{prob}{Problem}
\newtheorem{chal}{Challenge}

\begin{document}
\title{Worksheet 2}
\date{January 26, 2016}
\author{Introduction to Topology}

\maketitle
\begin{prob}
 Show that $\mathcal{B} = \{[a,b) \subset \R \vert a < b\}$
 is a basis for a topology on $\R$ (known as the \emph{lower limit topology}).
\end{prob}


\begin{prob}
 Consider the following six topologies on $\R$: the trivial topology, the discrete topology, the finite complement topology, the standard topology, the lower limit topology, and the upper limit topology.  Show how they compare to each other (finer, strictly finer, coarser, strictly coarser, noncomparable)
 and justify your answer.
\end{prob}



\begin{prob}
Determine which of the following collections of subsets of $\R$ are bases:
\begin{enumerate}[(a)]
 \item $\mathcal{C}_1 = \{(n, n+2) \subset \R \vert n \in \Z\}$ 
 \item $\mathcal{C}_2 = \{[a,b] \subset \R \vert a < b\}$
 \item $\mathcal{C}_3 = \{[a,b] \subset \R \vert a \leq b\}$
 \item $\mathcal{C}_4 = \{(-x,x) \subset \R \vert x \in \R\}$
 \item $\mathcal{C}_5 = \{(a,b) \cup \{b+1\} \subset \R \vert a < b\}$
\end{enumerate}
\end{prob}



\begin{prob}
 For each $n \in \Z$, define
 \begin{equation*}
  B(n) = 
  \begin{cases}
   \{n\} & \text{ if $n$ is odd},\\
   \{n-1,n,n+1\} & \text{if $n$ is even}.\\
  \end{cases}
 \end{equation*}
Show that $\mathcal{B} := \{B(n) \vert n \in \Z \}$
is a basis for a topology on $\Z$ (the resulting topology is known as the \emph{digital line topology}).
\end{prob}

\begin{prob}
 Prove that intervals of the form $[a,b)$ are closed in the lower limit topology on $\R$.
\end{prob}

\begin{prob}
 Show that
 \begin{enumerate}[(a)]
  \item $\R$ with the lower limit topology is Hausdorff;
  \item $\R$ with the finite complement topology is not Hausdorff.
 \end{enumerate}
 \end{prob}



\end{document}