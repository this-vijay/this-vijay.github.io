
\documentclass{amsart}
%\usepackage{cmbright}
\usepackage{amsmath, amsfonts, amssymb, amscd}
%\usepackage{tableau}
\usepackage{color}
\usepackage{xcolor}
\usepackage{graphicx}
\usepackage{array}
\usepackage{mathtools}
\usepackage{multirow}
\usepackage{framed}
\usepackage{tikz}
\usetikzlibrary{matrix,arrows}
\usepackage[square,sort,comma,numbers]{natbib}
\usepackage{enumerate}




%\oddsidemargin.5cm
%\evensidemargin.5cm
\addtolength{\oddsidemargin}{-.525in}
\addtolength{\evensidemargin}{-.525in}
\addtolength{\textwidth}{1in}

\addtolength{\topmargin}{-.87in}
\addtolength{\textheight}{1.5in}




\newcommand\coolunder[2]{\mathrlap{\smash{\underbrace{\phantom{%
    \begin{matrix} #2 \end{matrix}}}_{\mbox{$#1$}}}}#2}

%\input{tableau}


\newcommand{\+}[1]{\ensuremath{\mathbf{#1}}}
\newcommand{\vect}[1]{\boldsymbol{#1}} % Uncomment for BOLD vectors.
%\newcommand{\vect}[1]{\vec{#1}} % Uncomment for ARROW vectors.
\newcommand{\C}{{\mathbb C}}
\newcommand{\Z}{{\mathbb Z}}
\newcommand{\Q}{{\mathbb Q}}
\renewcommand{\P}{{\mathbb P}}
\newcommand{\OG}{\operatorname{OG}}
\newcommand{\OF}{\operatorname{OF}}
\newcommand{\bull}{{\scriptscriptstyle \bullet}}
\newcommand{\la}{\lambda}
\newcommand{\euler}[1]{\chi_{_{#1}}}
\newcommand{\cO}{{\mathcal O}}
\newcommand{\cG}{{\mathcal G}}
\newcommand{\cQ}{{\mathcal Q}}
\newcommand{\R}{{\mathbb R}}
\newcommand{\wt}{\widetilde}
\newcommand{\diag}{\operatorname{diag}}
\newcommand{\comp}{\operatorname{comp}}
\newcommand{\comment}[1]{}
\newcommand{\type}{\mathfrak{t}}
\newcommand{\op}{\text{op}}
\newcommand{\row}{{\bf r}}
\newcommand{\col}{{\bf c}}
\newcommand{\sym}{\mathfrak{S}}
\newcommand{\codim}{\text{codim}}
\DeclarePairedDelimiter{\ceil}{\lceil}{\rceil}
\DeclarePairedDelimiter{\floor}{\lfloor}{\rfloor}
\renewcommand{\emptyset}{\varnothing}

\newtheorem{thm}{Theorem}
\newtheorem{lemma}[thm]{Lemma}
\newtheorem{prop}[thm]{Proposition}
\newtheorem{cor}[thm]{Corollary}

\theoremstyle{definition}
\newtheorem{definition}[thm]{Definition}
\newtheorem{example}[thm]{Example}
\newtheorem{conj}[thm]{Conjecture}
\newtheorem{obs}[thm]{Observation}
\newtheorem{fact}[thm]{Fact}
\newtheorem{remark}[thm]{Remark}
\newtheorem{prob}{Problem}
\newtheorem{chal}{Challenge}

\begin{document}
\title{Worksheet 4}
\date{January 28, 2016}
\author{Introduction to Topology}

\maketitle

\begin{prob}
Let $X = \{(x,0) \in \R^2 \vert x \in \R\}$.  Describe the topology $X$ inherits as a subspace of $\R^2$ with the
standard topology.
\end{prob}



\begin{prob}
 Let $Y = [-1,1]$ have the standard topology.  Which of the following sets are open in $Y$ and
 which are open in $\R$?
 \begin{enumerate}[(a)]
  \item $A = (-1, -1/2) \cup (1/2,1)$ 
  \item $B = (-1, -1/2] \cup [1/2,1)$
  \item $C = [-1, -1/2) \cup (1/2,1]$
  \item $D = [-1, -1/2] \cup [1/2,1]$
 \end{enumerate}
\end{prob}


\begin{prob}
 Let $X$ be a topological space, and let $Y \subset X$ have the subspace topology.  Suppose $C$ is a subset of $Y$.
 Prove that  $C$ is closed in $Y$ if and only if $C = D \cap Y$ for some closed set $D \subset X$. 
\end{prob}

\begin{prob}
For each subset $A \subset \R$, determine whether the standard topology on $A$ is the discrete topology:
\begin{enumerate}[(a)]
 \item $A = \{\frac{1}{n} \in \R \vert n \in \Z_+\}$.
 \item $A = \{\frac{1}{n} \in \R \vert n \in \Z_+\} \cup \{0\}$.
 \item $A = \Q$.
\end{enumerate}
\end{prob}

\begin{prob}
 Is the finite complement topology on $\R^2$ the same as the product topology on $\R^2$ that results from taking
 the product $\R_{fc} \times \R_{fc}$, where $\R_{fc}$ denotes $\R$ with the finite complement topology?
\end{prob}

\begin{prob}
 Let $X$ and $Y$ be topological spaces, and assume that $A \subset X$ and $B \subset Y$.
 Prove that the topology on $A \times B$ as a subspace of the product $X \times Y$
 is the same as the product topology on $A \times B$, where $A$ and $B$ have the subspace topologies
 inherited from $X$ and $Y$ respectively.
\end{prob}

\begin{prob}
 Let $S^2$ be the sphere, $D$ be the disk, $T$ be the torus, $S^1$ be the circle, and $I = [0,1]$ with the standard topology.
 Draw pictures of the product spaces
 $S^2 \times I, T \times I, S^1 \times I, I \times I$, $S^1 \times S^1$, and $S^1 \times D$.
\end{prob}

\begin{prob}
 Show that if $X$ and $Y$ are Hausdorff spaces, then so is the product space $X \times Y$.
\end{prob}

\rule{14cm}{0.4pt}
\vspace{.5cm}

\begin{prob}
 Let $X = \R$ in the standard topology.  Take the partition 
 \[
 X^* = \{\ldots, (-1,0],(0,1],(1,2],\ldots\}.
 \] 
 Describe the open sets in the resulting quotient topology on $X^*$.
\end{prob}


\begin{prob}
 Define a partition of $X = \R^2 \setminus \{0\}$ by taking each ray emanating from the origin as an element
 in the partition.  Which topological space appears to be topologically equivalent to the
 quotient space that results from this partition?
 What if we use the partition consisting of full lines through the origin, minus the origin point.
\end{prob}

\begin{prob}
 Provide an example showing that the quotient space of a Hausdorff space need not be a Hausdorff space.
\end{prob}

\begin{prob}
 Consider the equivalence relation on $\R$ (with the standard topology) defined by $x \sim y$ if $x-y \in \Z$.
 Describe the quotient space that results from the partition on $\R$ into the equivalence classes
 in this equivalence relation.
\end{prob}

\begin{prob}
 Consider the equivalence relation on $\R^2$ (with the standard topology) defined by $(x_1,x_2) \sim (y_1,y_2)$ if
 $x_1 + x_2 = y_1 + y_2$.
 Describe the quotient space that results from the partition on $\R^2$ into the equivalence classes
 in this equivalence relation.
\end{prob}

\end{document}