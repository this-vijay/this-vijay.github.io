
\documentclass{amsart}
%\usepackage{cmbright}
\usepackage{amsmath, amsfonts, amssymb, amscd}
%\usepackage{tableau}
\usepackage{color}
\usepackage{xcolor}
\usepackage{graphicx}
\usepackage{array}
\usepackage{mathtools}
\usepackage{multirow}
\usepackage{framed}
\usepackage{tikz}
\usetikzlibrary{matrix,arrows}
\usepackage[square,sort,comma,numbers]{natbib}
\usepackage{enumerate}




%\oddsidemargin.5cm
%\evensidemargin.5cm
\addtolength{\oddsidemargin}{-.525in}
\addtolength{\evensidemargin}{-.525in}
\addtolength{\textwidth}{1in}

\addtolength{\topmargin}{-.87in}
\addtolength{\textheight}{1.5in}




\newcommand\coolunder[2]{\mathrlap{\smash{\underbrace{\phantom{%
    \begin{matrix} #2 \end{matrix}}}_{\mbox{$#1$}}}}#2}

%\input{tableau}


\newcommand{\+}[1]{\ensuremath{\mathbf{#1}}}
\newcommand{\vect}[1]{\boldsymbol{#1}} % Uncomment for BOLD vectors.
%\newcommand{\vect}[1]{\vec{#1}} % Uncomment for ARROW vectors.
\newcommand{\C}{{\mathbb C}}
\newcommand{\Z}{{\mathbb Z}}
\renewcommand{\P}{{\mathbb P}}
\newcommand{\OG}{\operatorname{OG}}
\newcommand{\OF}{\operatorname{OF}}
\newcommand{\bull}{{\scriptscriptstyle \bullet}}
\newcommand{\la}{\lambda}
\newcommand{\euler}[1]{\chi_{_{#1}}}
\newcommand{\cO}{{\mathcal O}}
\newcommand{\cG}{{\mathcal G}}
\newcommand{\cQ}{{\mathcal Q}}
\newcommand{\R}{{\mathbb R}}
\newcommand{\wt}{\widetilde}
\newcommand{\diag}{\operatorname{diag}}
\newcommand{\comp}{\operatorname{comp}}
\newcommand{\comment}[1]{}
\newcommand{\type}{\mathfrak{t}}
\newcommand{\op}{\text{op}}
\newcommand{\row}{{\bf r}}
\newcommand{\col}{{\bf c}}
\newcommand{\sym}{\mathfrak{S}}
\newcommand{\codim}{\text{codim}}
\DeclarePairedDelimiter{\ceil}{\lceil}{\rceil}
\DeclarePairedDelimiter{\floor}{\lfloor}{\rfloor}
\renewcommand{\emptyset}{\varnothing}

\newtheorem{thm}{Theorem}
\newtheorem{lemma}[thm]{Lemma}
\newtheorem{prop}[thm]{Proposition}
\newtheorem{cor}[thm]{Corollary}

\theoremstyle{definition}
\newtheorem{definition}[thm]{Definition}
\newtheorem{example}[thm]{Example}
\newtheorem{conj}[thm]{Conjecture}
\newtheorem{obs}[thm]{Observation}
\newtheorem{fact}[thm]{Fact}
\newtheorem{remark}[thm]{Remark}
\newtheorem{prob}{Problem}
\newtheorem{chal}{Challenge}

\begin{document}
\title{Worksheet 1}
\date{January 25, 2016}
\author{Introduction to Topology}

\maketitle

\begin{prob}
Consider the Euclidean metric, the taxicab metric, and the max metric on $\R^2$:
\[
d_E(p,q) := \sqrt{(p_1 - q_1)^2 + (p_2 - q_2)^2},
\]
\[
d_T(p,q) := \vert{p_1 - q_1}\vert + \vert{p_2 - q_2}\vert,
\]
\[
d_M(p,q) := \text{max}({\vert{p_1 - q_1}\vert, \vert{p_2 - q_2}\vert}).
\]
Draw the open unit ball $\{q \in \R^2 \vert d_\Box(0,q) < 1\}$ with respect to each of the three metrics.
\end{prob}

\begin{prob}
 Determine all possible topologies on the two-point set $X = \{a,b\}$.
\end{prob}

\begin{prob}
 Prove that a topology $\mathcal{T}$ on a set $X$ is the discrete topology if and only if $\{x\} \in \mathcal{T}$ for all $x \in X$.
\end{prob}

\begin{prob}
Consider the three-point set $X = \{a,b,c\}$. 
For each $n = 2,\ldots,8$ either find a topology on $X$
consisting of exactly $n$ open sets or prove that no such topology exists.
\end{prob}


\begin{prob}
Determine whether each of the following collections of subsets of $\R$ gives a topology on $\R$:
 \begin{enumerate}[(a)]
\item $\{\emptyset, \R\}$.
\item $\{U \vert U \subset \R\}$.
\item $\{\emptyset\} \cup \{U \vert U \subset \R \text{ and } \R \setminus U \text{ is finite}\}$.
\item $\{\emptyset, \R, \{1\}, \{2\}\}$.
\item $\{\emptyset, \R, [0,1)\}$.
\item $\{\emptyset\} \cup \{U \vert U \subset \R\ \text{ and } [0,1] \subset U\}$.
\item $\{\emptyset, \R\} \cup \{(a,b) \subset \R \vert a,b \in \mathbb Q \}$.
\item $\{\emptyset, \R\} \cup \{(-\infty,b) \subset \R \vert b \in \R\}$.
 \end{enumerate}
\end{prob}


\begin{prob}
 Determine the \emph{coarsest} topology on $R$ that contains the open sets
 $(0,2)$ and $(1,3)$.
\end{prob}

\begin{prob}
 Show that $\mathcal{B} = \{[a,b) \subset \R \vert a < b\}$
 is a basis for a topology on $\R$ (known as the \emph{lower limit topology}).
\end{prob}


\begin{prob}
 Consider the following six topologies on $\R$: the trivial topology, the discrete topology, the finite complement topology, the standard topology, the lower limit topology, and the upper limit topology.  Show how they compare to each other (finer, strictly finer, coarser, strictly coarser, noncomparable)
 and justify your answer.
\end{prob}




\end{document}