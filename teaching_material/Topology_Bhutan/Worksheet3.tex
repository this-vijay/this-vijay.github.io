
\documentclass{amsart}
%\usepackage{cmbright}
\usepackage{amsmath, amsfonts, amssymb, amscd}
%\usepackage{tableau}
\usepackage{color}
\usepackage{xcolor}
\usepackage{graphicx}
\usepackage{array}
\usepackage{mathtools}
\usepackage{multirow}
\usepackage{framed}
\usepackage{tikz}
\usetikzlibrary{matrix,arrows}
\usepackage[square,sort,comma,numbers]{natbib}
\usepackage{enumerate}




%\oddsidemargin.5cm
%\evensidemargin.5cm
\addtolength{\oddsidemargin}{-.525in}
\addtolength{\evensidemargin}{-.525in}
\addtolength{\textwidth}{1in}

\addtolength{\topmargin}{-.87in}
\addtolength{\textheight}{1.5in}




\newcommand\coolunder[2]{\mathrlap{\smash{\underbrace{\phantom{%
    \begin{matrix} #2 \end{matrix}}}_{\mbox{$#1$}}}}#2}

%\input{tableau}


\newcommand{\+}[1]{\ensuremath{\mathbf{#1}}}
\newcommand{\vect}[1]{\boldsymbol{#1}} % Uncomment for BOLD vectors.
%\newcommand{\vect}[1]{\vec{#1}} % Uncomment for ARROW vectors.
\newcommand{\C}{{\mathbb C}}
\newcommand{\Z}{{\mathbb Z}}
\newcommand{\Q}{{\mathbb Q}}
\renewcommand{\P}{{\mathbb P}}
\newcommand{\OG}{\operatorname{OG}}
\newcommand{\OF}{\operatorname{OF}}
\newcommand{\bull}{{\scriptscriptstyle \bullet}}
\newcommand{\la}{\lambda}
\newcommand{\euler}[1]{\chi_{_{#1}}}
\newcommand{\cO}{{\mathcal O}}
\newcommand{\cG}{{\mathcal G}}
\newcommand{\cQ}{{\mathcal Q}}
\newcommand{\R}{{\mathbb R}}
\newcommand{\wt}{\widetilde}
\newcommand{\diag}{\operatorname{diag}}
\newcommand{\comp}{\operatorname{comp}}
\newcommand{\comment}[1]{}
\newcommand{\type}{\mathfrak{t}}
\newcommand{\op}{\text{op}}
\newcommand{\row}{{\bf r}}
\newcommand{\col}{{\bf c}}
\newcommand{\sym}{\mathfrak{S}}
\newcommand{\codim}{\text{codim}}
\DeclarePairedDelimiter{\ceil}{\lceil}{\rceil}
\DeclarePairedDelimiter{\floor}{\lfloor}{\rfloor}
\renewcommand{\emptyset}{\varnothing}

\newtheorem{thm}{Theorem}
\newtheorem{lemma}[thm]{Lemma}
\newtheorem{prop}[thm]{Proposition}
\newtheorem{cor}[thm]{Corollary}

\theoremstyle{definition}
\newtheorem{definition}[thm]{Definition}
\newtheorem{example}[thm]{Example}
\newtheorem{conj}[thm]{Conjecture}
\newtheorem{obs}[thm]{Observation}
\newtheorem{fact}[thm]{Fact}
\newtheorem{remark}[thm]{Remark}
\newtheorem{prob}{Problem}
\newtheorem{chal}{Challenge}

\begin{document}
\title{Worksheet 3}
\date{January 27, 2016}
\author{Introduction to Topology}

\maketitle



\begin{prob}
 Determine Int$(A)$, Cl$(A),$ and $\partial A$ in each case.
\begin{enumerate}[(a)]
 \item $A = (0,1]$ in lower limit topology on $\R$.
 \item $A =\{a\}$ in $X = \{a,b,c\}$ with topology $\{X, \emptyset, \{a\}, \{a,b\}\}$.
  \item $A =\{a,c\}$ in $X = \{a,b,c\}$ with topology $\{X, \emptyset, \{a\}, \{a,b\}\}$.
\item $A = (-1,1) \cup \{2\}$ in the standard topology on $\R$.
\end{enumerate}

 \end{prob}


\begin{prob}
Prove that Cl$(\Q)=\R$ in the standard topology on $\R$.
\end{prob}

\begin{prob}
 Let $X$ be a topological space, $A$ a subset of $X$, and $y$ an element of $X$.  Prove that $y \in \text{Cl}(A)$
 if and only if every open set containing $y$ intersects $A$.
\end{prob}


\begin{prob}
 In each case, determine whether the relation in the blank is $\subset$, $\supset$, or $=$.
 In cases where equality does not hold, provide an example indicating so.
 \begin{enumerate}[(a)]
  \item Cl$(A) \cap $Cl$(B)\underline{\phantom{aaaa}}$Cl$(A \cap B)$. 
  \item Cl$(A) \cup $Cl$(B)\underline{\phantom{aaaa}}$Cl$(A \cup B)$. 
 \end{enumerate}
\end{prob}

\begin{prob}
 Determine the set of limit points of
 \begin{enumerate}[(a)]
  \item the the interval $[0,1]$ in the finite complement topology on $\R$.
  \item the set $A = \{\frac{1}{m} + \frac{1}{n} \in \R \vert m,n \in \Z_+\}$ in
  the standard topology on $\R$.
  \item the set $S = \{(x,\sin(\frac{1}{x})) \in \R^2 \vert 0 < x \leq 1\}$ as a subset
  of $\R^2$ in the standard topology.
 \end{enumerate}
\end{prob}



\begin{prob}
 Let $A$ be a subset of $\R^2$ in the standard topology.  Prove that if $x$ is a limit point
 of $A$, then there is a sequence of points in $A$ that converges to $x$.
\end{prob}

\begin{prob}
Determine $\partial A$ where $A = [0,1]$ in the finite complement topology on $\R$.
\end{prob}



\end{document}