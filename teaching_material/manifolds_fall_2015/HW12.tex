
\documentclass{amsart}
%\usepackage{cmbright}
\usepackage{amsmath, amsfonts, amssymb, amscd}
%\usepackage{tableau}
\usepackage{color}
\usepackage{xcolor}
\usepackage{graphicx}
\usepackage{array}
\usepackage{mathtools}
\usepackage{multirow}
\usepackage{framed}
\usepackage{tikz}
\usetikzlibrary{matrix,arrows}
\usepackage[square,sort,comma,numbers]{natbib}
\usepackage{enumerate}




%\oddsidemargin.5cm
%\evensidemargin.5cm
\addtolength{\oddsidemargin}{-.525in}
\addtolength{\evensidemargin}{-.525in}
\addtolength{\textwidth}{1in}

\addtolength{\topmargin}{-.87in}
\addtolength{\textheight}{1.5in}




\newcommand\coolunder[2]{\mathrlap{\smash{\underbrace{\phantom{%
    \begin{matrix} #2 \end{matrix}}}_{\mbox{$#1$}}}}#2}

%\input{tableau}


\newcommand{\+}[1]{\ensuremath{\mathbf{#1}}}
\newcommand{\vect}[1]{\boldsymbol{#1}} % Uncomment for BOLD vectors.
%\newcommand{\vect}[1]{\vec{#1}} % Uncomment for ARROW vectors.
\newcommand{\C}{{\mathbb C}}
\newcommand{\Z}{{\mathbb Z}}
\renewcommand{\P}{{\mathbb P}}
\newcommand{\OG}{\operatorname{OG}}
\newcommand{\OF}{\operatorname{OF}}
\newcommand{\bull}{{\scriptscriptstyle \bullet}}
\newcommand{\la}{\lambda}
\newcommand{\euler}[1]{\chi_{_{#1}}}
\newcommand{\cO}{{\mathcal O}}
\newcommand{\cG}{{\mathcal G}}
\newcommand{\cQ}{{\mathcal Q}}
\newcommand{\R}{{\mathbb R}}
\newcommand{\wt}{\widetilde}
\newcommand{\diag}{\operatorname{diag}}
\newcommand{\comp}{\operatorname{comp}}
\newcommand{\comment}[1]{}
\newcommand{\type}{\mathfrak{t}}
\newcommand{\op}{\text{op}}
\newcommand{\row}{{\bf r}}
\newcommand{\col}{{\bf c}}
\newcommand{\sym}{\mathfrak{S}}
\newcommand{\codim}{\text{codim}}
\DeclarePairedDelimiter{\ceil}{\lceil}{\rceil}
\DeclarePairedDelimiter{\floor}{\lfloor}{\rfloor}
\renewcommand{\emptyset}{\varnothing}
\renewcommand{\tilde}{\widetilde}


\newtheorem{thm}{Theorem}
\newtheorem{lemma}[thm]{Lemma}
\newtheorem{prop}[thm]{Proposition}
\newtheorem{cor}[thm]{Corollary}

\theoremstyle{definition}
\newtheorem{definition}[thm]{Definition}
\newtheorem{example}[thm]{Example}
\newtheorem{conj}[thm]{Conjecture}
\newtheorem{obs}[thm]{Observation}
\newtheorem{fact}[thm]{Fact}
\newtheorem{remark}[thm]{Remark}
\newtheorem{prob}{Problem}
\newtheorem{chal}{Challenge}

\begin{document}
\title{Problem Set 12}
\date{November 16, 2015}
\author{Introduction to Manifolds}

\maketitle


Suppose $M$ is a connected non-orientable smooth manifold with
atlas $\mathcal{A} = \{(U_\alpha, \phi_\alpha=(x_\alpha^1,\ldots,x_\alpha^n))\}$.
Consider the set
\[
\tilde{M} := \bigsqcup_{p \in M} \{\text{orientations of } T_pM\},
\]
and let $\pi:\tilde{M} \to M$ be the projection $[\omega_p] \mapsto p$.  
Here are four problems to help you understand the orientation covering $\tilde{M}$.

\begin{prob}
For every $\alpha$, let 
\[U^+_\alpha = \{\phi_\alpha^*[dx^1_\alpha \wedge \ldots \wedge dx^n]_p : p \in U_\alpha\}\]
and let 
\[U^-_\alpha = \{\phi_\alpha^*[-dx^1_\alpha \wedge \ldots \wedge dx^n]_p : p \in U_\alpha\}.\]
For any $\alpha$, let $\bar{\phi}_\alpha := (-x_\alpha^1,x_\alpha^2,\ldots,x^n_\alpha)$ be
the opposite orientation chart to $\phi_\alpha$.
Show that the collection 
\[\tilde{\mathcal{A}} 
:= \{(U^+_\alpha,\phi_\alpha^+ := \phi_\alpha \circ \pi)\} \cup \{(U_\alpha^-, \phi^-_\alpha := \bar{\phi}_\alpha \circ \pi)\}\]
gives $\tilde{M}$ the structure of a smooth manifold.
\end{prob}

\begin{prob}
 Show that $\tilde{A}$ is an \emph{oriented} atlas on $\tilde{M}$ (i.e.,
 whenever two sets $U^{\pm}_\alpha$ and $U^{\pm}_\beta$ intersect,
 the transition map $\phi^{\pm}_\beta \circ (\phi^{\pm}_\alpha)^{-1}$ is orientation preserving).
\end{prob}

\begin{prob}
 Show that the manifold $\tilde{M}$ is connected.
\end{prob}

\begin{prob}
Describe the oriented atlas $\tilde{\mathcal{A}}$ on $\tilde{M}$ in the special case where $M$ is a M\"obius strip and $\mathcal{A}$ consists of two charts on $M$.
\end{prob}

\rule{14cm}{0.4pt}
\vspace{.5cm}

Now let $\alpha: \tilde{M} \to \tilde{M}$
be the map that interchanges the two points in each fiber of $\pi$.  The following two problems will determine
the top cohomology of the non-orientable manifold $M$.

\begin{prob}
Prove that $\alpha$ is an orientation-reversing diffeomorphism of $\tilde{M}$.
\end{prob}

\begin{prob}
Now suppose $M$ has dimension $n$.  Since $\tilde{M}$
is orientable, there exists an isomorphism 
from $H^n(\tilde{M})$ to $\R$ defined by
\[
[\omega] \mapsto \int_{\tilde{M}} \omega,
\]
by a result mentioned in class.  Assuming this
isomorphism, prove that $H^n(M) = 0$.
\end{prob}

\rule{14cm}{0.4pt}
\vspace{.5cm}

The following three problems relate to integral curves and flows,
and are taken from Chapter $12$ of [Lee].

\begin{prob}
Recall that a vector field is \emph{complete} if it generates a global flow; i.e. if all integral curves are defined
for all $t \in \R$.  Show that every smooth vector field with compact support is complete.
\end{prob}

\begin{prob}
Let $M$ be a connected smooth manifold.
Show that the group of diffeomorphisms of $M$ acts transitively
on $M$.  (For hints see Problem $12.3$ in [Lee].)
\end{prob}

\begin{prob}
 Let $\theta$ be a flow on an oriented manifold.
 Show that for each $t \in \R$, $\theta_t$ is
 orientation-preserving wherever it is defined.
\end{prob}











\end{document}