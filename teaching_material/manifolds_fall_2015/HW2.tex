
\documentclass{amsart}
%\usepackage{cmbright}
\usepackage{amsmath, amsfonts, amssymb, amscd}
%\usepackage{tableau}
\usepackage{color}
\usepackage{xcolor}
\usepackage{graphicx}
\usepackage{array}
\usepackage{mathtools}
\usepackage{multirow}
\usepackage{framed}
\usepackage{tikz}
\usetikzlibrary{matrix,arrows}
\usepackage[square,sort,comma,numbers]{natbib}
\usepackage{enumerate}




%\oddsidemargin.5cm
%\evensidemargin.5cm
\addtolength{\oddsidemargin}{-.525in}
\addtolength{\evensidemargin}{-.525in}
\addtolength{\textwidth}{1in}

\addtolength{\topmargin}{-.87in}
\addtolength{\textheight}{1.5in}




\newcommand\coolunder[2]{\mathrlap{\smash{\underbrace{\phantom{%
    \begin{matrix} #2 \end{matrix}}}_{\mbox{$#1$}}}}#2}

%\input{tableau}


\newcommand{\+}[1]{\ensuremath{\mathbf{#1}}}
\newcommand{\vect}[1]{\boldsymbol{#1}} % Uncomment for BOLD vectors.
%\newcommand{\vect}[1]{\vec{#1}} % Uncomment for ARROW vectors.
\newcommand{\C}{{\mathbb C}}
\newcommand{\Z}{{\mathbb Z}}
\renewcommand{\P}{{\mathbb P}}
\newcommand{\OG}{\operatorname{OG}}
\newcommand{\OF}{\operatorname{OF}}
\newcommand{\bull}{{\scriptscriptstyle \bullet}}
\newcommand{\la}{\lambda}
\newcommand{\euler}[1]{\chi_{_{#1}}}
\newcommand{\cO}{{\mathcal O}}
\newcommand{\cG}{{\mathcal G}}
\newcommand{\cQ}{{\mathcal Q}}
\newcommand{\R}{{\mathbb R}}
\newcommand{\wt}{\widetilde}
\newcommand{\diag}{\operatorname{diag}}
\newcommand{\comp}{\operatorname{comp}}
\newcommand{\comment}[1]{}
\newcommand{\type}{\mathfrak{t}}
\newcommand{\op}{\text{op}}
\newcommand{\row}{{\bf r}}
\newcommand{\col}{{\bf c}}
\newcommand{\sym}{\mathfrak{S}}
\newcommand{\codim}{\text{codim}}
\DeclarePairedDelimiter{\ceil}{\lceil}{\rceil}
\DeclarePairedDelimiter{\floor}{\lfloor}{\rfloor}
\renewcommand{\emptyset}{\varnothing}

\newtheorem{thm}{Theorem}
\newtheorem{lemma}[thm]{Lemma}
\newtheorem{prop}[thm]{Proposition}
\newtheorem{cor}[thm]{Corollary}

\theoremstyle{definition}
\newtheorem{definition}[thm]{Definition}
\newtheorem{example}[thm]{Example}
\newtheorem{conj}[thm]{Conjecture}
\newtheorem{obs}[thm]{Observation}
\newtheorem{fact}[thm]{Fact}
\newtheorem{remark}[thm]{Remark}
\newtheorem{prob}{Problem}
\newtheorem{chal}{Challenge}

\begin{document}
\title{Problem Set 2}
\date{August 12, 2015}
\author{Introduction to Manifolds}

\maketitle



\begin{prob}
 \begin{enumerate}[(a)]
  \item Suppose $\beta \in L_k(\R^n)$ is symmetric and $\gamma \in L_k(\R^n)$ is alternating.
  Compute $S\beta, A\beta, S\omega,$ and $A\omega$.
  \item Show that $\alpha^1 \otimes \alpha^2 \otimes \alpha^3$ cannot be expressed 
  as the sum of a symmetric part and an alternating part, where $\{\alpha^i\}$ is the standard basis for $(\R^n)^*$.
 \end{enumerate}
\end{prob}

\begin{prob}
\begin{enumerate}[(a)]
 \item Prove that $\beta \wedge \beta = 0$ for all $\beta \in A_k(\R^n)$ if $k$ is odd.
 \item Produce an element $\beta \in A_2(\R^n)$ such that $\beta \wedge \beta \neq 0$.
\end{enumerate}
\end{prob} 




\begin{prob}
Suppose $\beta^1, \ldots, \beta^k$ and $\gamma^1, \ldots, \gamma^k$
are elements  of $L_1(\R^n) = (\R^n)^*$.
\begin{enumerate}[(a)]
 \item Prove that if
\begin{equation*}
 \beta^i = \sum^k_{j=1} a^i_j \gamma^j, \text{ for } i = 1,\ldots,k,
\end{equation*}
for some $k \times k$ matrix $A = [a^i_j]$, then
\begin{equation*}
 \beta^1 \wedge \ldots \wedge \beta^k = \det(A) \gamma^1 \wedge
 \ldots \wedge \gamma^k.
\end{equation*}
\item Prove that $\beta^1 \wedge \ldots \wedge \beta^k \neq 0$
if and only if $\beta^1, \ldots, \beta^k$ are linearly independent
in $(\R^n)^*$.
\item Assuming $\beta^1, \ldots, \beta^k$
and $\gamma^1, \ldots, \gamma^k$ are indeed linearly independent,
show that they have the same span in $(\R^n)^*$ if and only if 
\begin{equation*}
  \beta^1 \wedge \ldots \wedge \beta^k = C \gamma^1 \wedge
 \ldots \wedge \gamma^k
\end{equation*}
for some nonzero number $C \in \R$.
\end{enumerate}
\end{prob}




\begin{prob}
Let $\{\alpha^1, \alpha^2, \alpha^3\}$ be the dual basis
to the standard basis $\{e_1, e_2, e_3\}$ on $\R^3$.
To each $1$-covector $\alpha = a_1 \alpha^1 + a_2 \alpha^2 + a_3 \alpha^3
\in A_1(\R^3)$, we associate a vector $\+v_\alpha := a_1 e_1 + a_2 e_2 + a_3 e_3
\in \R^3$.
To each $2$-covector $\gamma = c_1 \alpha^2 \wedge \alpha^3 + c_2 \alpha^3 \wedge \alpha^1 + c_3 \alpha^1 \wedge \alpha^2
\in A_2(\R^3)$, we associate a vector $\+v_\gamma := c_1 e_1 + c_2 e_2 + c_3 e_3
\in \R^3$.
Show that for any $1$-covectors $\alpha$ and $\beta$, we then have
$\+v_{\alpha \wedge \beta} = \+v_\alpha \times \+v_\beta$.
\end{prob}

\vspace{5mm}

{\bf The following two problems are optional.  They
will not contribute to or detract from your grade, but you are encouraged
to attempt them.}

\vspace{5mm}

\begin{chal}
Prove Poincare's Lemma for $1$-forms on $\R^3$:
If $\omega$ is a differential $1$-form on $\R^3$ such
that $d\omega = 0$, then $\omega = df$ for some $f \in C^{\infty}(\R^3)$.
\end{chal}

\begin{chal}
Recall that $\omega \in A_k(V)$ is decomposable if
$\omega = \beta^1 \wedge \ldots \wedge \beta^k$
for some $1$-covectors $\beta^1, \ldots, \beta^k$ on $V$.
Prove that $\omega \in A_{n-1}(\R^n)$ is always decomposable.
\end{chal}






\end{document}