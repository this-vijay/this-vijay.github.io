
\documentclass{amsart}
%\usepackage{cmbright}
\usepackage{amsmath, amsfonts, amssymb, amscd}
%\usepackage{tableau}
\usepackage{color}
\usepackage{xcolor}
\usepackage{graphicx}
\usepackage{array}
\usepackage{mathtools}
\usepackage{multirow}
\usepackage{framed}
\usepackage{tikz}
\usetikzlibrary{matrix,arrows}
\usepackage[square,sort,comma,numbers]{natbib}
\usepackage{enumerate}




%\oddsidemargin.5cm
%\evensidemargin.5cm
\addtolength{\oddsidemargin}{-.525in}
\addtolength{\evensidemargin}{-.525in}
\addtolength{\textwidth}{1in}

\addtolength{\topmargin}{-.87in}
\addtolength{\textheight}{1.5in}




\newcommand\coolunder[2]{\mathrlap{\smash{\underbrace{\phantom{%
    \begin{matrix} #2 \end{matrix}}}_{\mbox{$#1$}}}}#2}

%\input{tableau}


\newcommand{\+}[1]{\ensuremath{\mathbf{#1}}}
\newcommand{\vect}[1]{\boldsymbol{#1}} % Uncomment for BOLD vectors.
%\newcommand{\vect}[1]{\vec{#1}} % Uncomment for ARROW vectors.
\newcommand{\C}{{\mathbb C}}
\newcommand{\Z}{{\mathbb Z}}
\renewcommand{\P}{{\mathbb P}}
\newcommand{\OG}{\operatorname{OG}}
\newcommand{\OF}{\operatorname{OF}}
\newcommand{\bull}{{\scriptscriptstyle \bullet}}
\newcommand{\la}{\lambda}
\newcommand{\euler}[1]{\chi_{_{#1}}}
\newcommand{\cO}{{\mathcal O}}
\newcommand{\cG}{{\mathcal G}}
\newcommand{\cQ}{{\mathcal Q}}
\newcommand{\R}{{\mathbb R}}
\newcommand{\wt}{\widetilde}
\newcommand{\diag}{\operatorname{diag}}
\newcommand{\comp}{\operatorname{comp}}
\newcommand{\comment}[1]{}
\newcommand{\type}{\mathfrak{t}}
\newcommand{\op}{\text{op}}
\newcommand{\row}{{\bf r}}
\newcommand{\col}{{\bf c}}
\newcommand{\sym}{\mathfrak{S}}
\newcommand{\codim}{\text{codim}}
\DeclarePairedDelimiter{\ceil}{\lceil}{\rceil}
\DeclarePairedDelimiter{\floor}{\lfloor}{\rfloor}
\renewcommand{\emptyset}{\varnothing}

\newtheorem{thm}{Theorem}
\newtheorem{lemma}[thm]{Lemma}
\newtheorem{prop}[thm]{Proposition}
\newtheorem{cor}[thm]{Corollary}

\theoremstyle{definition}
\newtheorem{definition}[thm]{Definition}
\newtheorem{example}[thm]{Example}
\newtheorem{conj}[thm]{Conjecture}
\newtheorem{obs}[thm]{Observation}
\newtheorem{fact}[thm]{Fact}
\newtheorem{remark}[thm]{Remark}
\newtheorem{prob}{Problem}
\newtheorem{chal}{Challenge}

\begin{document}
\title{Problem Set 5}
\date{September 1, 2015}
\author{Introduction to Manifolds}

\maketitle

A smooth map $F:N \to M$ is \emph{transverse} to a 
submanifold\footnote{By \emph{submanifold} we always mean \emph{regular submanifold} in Tu's terminology.}
$S \subset M$ if for every $p \in F^{-1}(S)$,
\[
F_{*,p}(T_pN) + T_{F(p)}S = T_{F(p)}M.
\]
We have the following theorem.


\begin{framed}
\begin{thm}\label{T:transversality}
Suppose the smooth map $F: N \to M$ is transverse to a
submanifold $S$ of codimension $k$ in $M$.  Then
$F^{-1}(S)$ is a submanifold of codimension $k$ in $N$.
\end{thm}
\end{framed}

\begin{prob}
Let $F:\R^2 \to \R$ be defined by
\[
 F(x,y) = x^3 + xy + y^3 +1.
\]
Find all values $c \in \R$ for which
the level set $F^{-1}(c)$ is a  submanifold of $\R^2$.
\end{prob}

\begin{prob}
 Prove that the solution set of the system of equations
 \[
  x^3 + y^3 + z^3 = 1, z = xy,
 \]
is a submanifold of $\R^3$.
\end{prob}


\begin{prob}
Suppose that a subset $S$ of $\R^2$ has the property that
locally on $S$ one of the coordinates is a $C^\infty$ function
of the other coordinate.  Show that $S$ is a submanifold of $\R^2$.
\end{prob} 




\begin{prob}
Let $M$ be a smooth manifold.
Show that a submanifold of $M$ is closed in $M$ if and only if
the inclusion map is proper.
\end{prob}




\begin{prob}
Suppose the smooth map $F: N \to M$ is transverse to a
submanifold $S$ of codimension $k$ in $M$.
Let $p \in F^{-1}(S)$ and let $(U,x^1,\ldots,x^m)$ be an
adapted chart centered at $F(p)$ for $M$ relative to $S$
such that $U \cap S = Z(x^{m-k+1},\ldots,x^m)$,
the zero set of the functions $x^{m-k+1},\ldots,x^m$.
Define $G:U \to \R^k$ by $G = (x^{m-k+1},\ldots,x^m)$.
\begin{enumerate}[(a)]
 \item Show that $F^{-1}(U) \cap F^{-1}(S) = (G \circ F)^{-1}(0)$.
 \item Show that $F^{-1}(U) \cap F^{-1}(S)$ is a regular level set
 of the function $G \circ F: F^{-1}(U) \to \R^k$.
 \item Prove Theorem \ref{T:transversality}.
\end{enumerate}

\end{prob}

\vspace{5mm}

{\bf The following two problems are optional.  They
will not contribute to or detract from your grade, but you are encouraged
to attempt them.}

\vspace{5mm}

We say submanifolds $N_1$ and $N_2$ of a smooth manifold $M$
have \emph{transverse intersection} if $T_pN_1 + T_pN_2 = T_pM$
for any $p \in N_1 \cap N_2$.  In this case we write $N_1 \pitchfork N_2$.

\begin{chal}
Let $V$ be a vector space, and let $\Delta$ be the
diagonal of $V \times V$.  For a linear map $A: V \to V$,
consider the graph $W = \{(v,Av):v \in V\}$.  Show
that $W \pitchfork \Delta$ if and only if $+1$ is not an eigenvalue
of $A$.
\end{chal}

\begin{chal}
Let $M$ be a smooth manifold and let $F: M \to M$ be a smooth map with fixed point $p \in M$.  If $+1$ is not an eigenvalue
of $F_{*,p}: T_pM \to T_pM$, then $p$ is called a \emph{Lefschetz
fixed point} of $F$.  We say $F$ is \emph{Lefschetz} if
all its fixed points are Lefschetz.  Prove that if $M$ is compact
and $F$ is Lefschetz, then $F$ has finitely many fixed points.
\end{chal}






\end{document}