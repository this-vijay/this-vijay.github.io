
\documentclass{amsart}
%\usepackage{cmbright}
\usepackage{amsmath, amsfonts, amssymb, amscd}
%\usepackage{tableau}
\usepackage{color}
\usepackage{xcolor}
\usepackage{graphicx}
\usepackage{array}
\usepackage{mathtools}
\usepackage{multirow}
\usepackage{framed}
\usepackage{tikz}
\usetikzlibrary{matrix,arrows}
\usepackage[square,sort,comma,numbers]{natbib}
\usepackage{enumerate}




%\oddsidemargin.5cm
%\evensidemargin.5cm
\addtolength{\oddsidemargin}{-.525in}
\addtolength{\evensidemargin}{-.525in}
\addtolength{\textwidth}{1in}

\addtolength{\topmargin}{-.87in}
\addtolength{\textheight}{1.5in}




\newcommand\coolunder[2]{\mathrlap{\smash{\underbrace{\phantom{%
    \begin{matrix} #2 \end{matrix}}}_{\mbox{$#1$}}}}#2}

%\input{tableau}


\newcommand{\+}[1]{\ensuremath{\mathbf{#1}}}
\newcommand{\vect}[1]{\boldsymbol{#1}} % Uncomment for BOLD vectors.
%\newcommand{\vect}[1]{\vec{#1}} % Uncomment for ARROW vectors.
\newcommand{\C}{{\mathbb C}}
\newcommand{\Z}{{\mathbb Z}}
\renewcommand{\P}{{\mathbb P}}
\newcommand{\OG}{\operatorname{OG}}
\newcommand{\OF}{\operatorname{OF}}
\newcommand{\bull}{{\scriptscriptstyle \bullet}}
\newcommand{\la}{\lambda}
\newcommand{\euler}[1]{\chi_{_{#1}}}
\newcommand{\cO}{{\mathcal O}}
\newcommand{\cG}{{\mathcal G}}
\newcommand{\cQ}{{\mathcal Q}}
\newcommand{\R}{{\mathbb R}}
\newcommand{\wt}{\widetilde}
\newcommand{\diag}{\operatorname{diag}}
\newcommand{\comp}{\operatorname{comp}}
\newcommand{\comment}[1]{}
\newcommand{\type}{\mathfrak{t}}
\newcommand{\op}{\text{op}}
\newcommand{\row}{{\bf r}}
\newcommand{\col}{{\bf c}}
\newcommand{\sym}{\mathfrak{S}}
\newcommand{\codim}{\text{codim}}
\DeclarePairedDelimiter{\ceil}{\lceil}{\rceil}
\DeclarePairedDelimiter{\floor}{\lfloor}{\rfloor}
\renewcommand{\emptyset}{\varnothing}

\newtheorem{thm}{Theorem}
\newtheorem{lemma}[thm]{Lemma}
\newtheorem{prop}[thm]{Proposition}
\newtheorem{cor}[thm]{Corollary}

\theoremstyle{definition}
\newtheorem{definition}[thm]{Definition}
\newtheorem{example}[thm]{Example}
\newtheorem{conj}[thm]{Conjecture}
\newtheorem{obs}[thm]{Observation}
\newtheorem{fact}[thm]{Fact}
\newtheorem{remark}[thm]{Remark}
\newtheorem{prob}{Problem}
\newtheorem{chal}{Challenge}

\begin{document}
\title{Final Exam}
\date{November 26, 2015}
\author{Introduction to Manifolds}

\maketitle


The exam consists of seven questions  worth a total of $105$ points.
You may take up to three hours to complete the exam.


\begin{prob}[$15$ pts]
 Let $x,y,z$ be the standard coordinates on $\R^3$.  A plane in $\R^3$
 is vertical if it is defined by 
 $ax + by = 0$ for some $a,b \neq 0,0$.
 Prove that restricted to a vertical plane,
 $dx \wedge dy = 0$.
\end{prob}


\begin{prob}[$15$ pts]
Determine whether each of the following statements is true or false.  If false, provide a counterexample.  If true, briefly justify your answer.
\begin{enumerate}[(a)]
 \item The sphere $S^2$ cannot be given the structure of a Lie group.
 \item The top de Rham cohomology of an orientable smooth manifold is $\R$.
 \item If the first de Rham cohomology of a manifold $M$ is zero, then $M$ is simply connected.
\end{enumerate}
\end{prob}



\begin{prob}[$15$ pts]
Let $X$ be a smooth vector field on a smooth manifold $M$.  
Let $c(t)$ be an integral curve of $X$ whose
maximal domain of definition is \emph{not} all of $\R$.
Show that the image
of $c$ cannot lie in any compact subset of $M$.
\end{prob}


\begin{prob}[$15$ pts]
Let $M$ be a smooth manifold.  Recall
that the Lie bracket of two smooth vector fields is
again a smooth vector field.  Is the Lie bracket $C^{\infty}$-linear in both variables?  Justify your answer.
\end{prob}




\begin{prob}[$15$ pts]
 Let $\pi$ be the  covering projection $S^2 \to \R P^2$.
 Prove that $\pi^*: H^*(\R P^2) \to H^*(S^2)$ is injective.
\end{prob}






\begin{prob}[$15$ pts]
Let $M$ be a smooth manifold.  Let $f: M \to \R$ be a smooth function, and
suppose $c$ is a regular value of $f$.
Prove that $f^{-1}([c,\infty))$ is a smooth manifold with boundary. 
\end{prob}










\begin{prob}[$15$ pts]
Let $B \subset \R^3$ be a solid ball of radius $2$ centered at the origin.  Let $S^1$ denote
the unit circle in the $xy$ plane, with open tubular neighborhood
\[S := \{p \in \R^3: m(S^1, p) < 0.1 \},\]
where $m$ is the standard Euclidean metric on $\R^3$.
Let $M := B \setminus S$.
\begin{enumerate}[(a)]
 \item Compute the de Rham cohomology vector spaces $H^2(M)$ and $H^3(M)$.
 \item Compute the de Rham cohomology vector space $H^1(M)$.
\end{enumerate}
\end{prob}

%\vspace{3mm}

%The following problems are dispensable.

%\begin{prob}
%Give an example of a Lie group acting  on a manifold
%$M$ in which two different orbits have different dimensions,
%and neither orbit has dimension equal to zero or to the
%dimension of $M$.
%\end{prob}



%\begin{prob}
%Let $M$ be a smooth manifold.
% We say a $1$-form $\omega \in \Omega^1(M)$ is \emph{conservative}
% if the line integral $\int_\gamma \omega$ over any piecewise-smooth curve $\gamma: [0,1] \to M$
% depends only on the starting point and ending point of $\gamma$ (i.e.
% if whenever $\gamma, \gamma': [0,1] \to M$ are two piecewise-smooth curves,
% we have $\int_\gamma \omega = \int_{\gamma'} \omega$.)
% Prove that any conservative $1$-form $ \omega \in \Omega^1(M)$ is exact.
%\end{prob}

%\begin{prob}
%Let $B \subset \R^3$ be a solid ball of radius $2$ centered at the origin.  Let %$S^1$ denote
%the unit circle in the $xy$ plane, with open tubular neighborhood
%\[S := \{p \in \R^3: m(S^1, p) < 0.1 \},\]
%where $m$ is the standard Euclidean metric on $\R^3$.
%Let $M := B \setminus S$.
%Compute the de Rham cohomology vector spaces of $M$.
%\end{prob}

%\vspace{3mm}




%\begin{prob}[$30$ pts]
%For any manifold with boundary $M$, we define its \emph{double} $M^\bigstar$ to
%be 
%\[M \times \{0,1\} / \sim, \text{ where } (x,0)  \sim (x,1) \text{ for all } x \in \partial M.\]
%This problem has three parts; in each part you may use the results of a previous part, even if you were not able to solve that part.
%\begin{enumerate}[(a)]
%\item ($10$ pts) Let $M$ be an orientable manifold of dimension $n$ with nonempty boundary.  Show that $M^\bigstar$ is an orientable manifold without boundary.
%In particular, given an oriented smooth atlas $\mathcal{A}$ on $M$ (for
%which the local model is the upper half space in $\R^n$), produce an oriented smooth atlas $\mathcal{A}^\bigstar$ on $M^\bigstar$ (for
%which the local model is $\R^n$).  You do \emph{not} need to
%bother with the Hausdorff and second countable conditions.
%\item ($10$ pts) Let $\iota: M \hookrightarrow M^\bigstar$ be
%an obvious inclusion map.  Show that
%the restriction map $\iota^*: H^*(M^\bigstar) \to H^*(M)$
%is the zero map.
%\item ($10$ pts) Show that
%$H^n(M) = 0$.
%\end{enumerate}
%\end{prob}








\end{document}