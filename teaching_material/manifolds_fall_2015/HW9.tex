
\documentclass{amsart}
%\usepackage{cmbright}
\usepackage{amsmath, amsfonts, amssymb, amscd}
%\usepackage{tableau}
\usepackage{color}
\usepackage{xcolor}
\usepackage{graphicx}
\usepackage{array}
\usepackage{mathtools}
\usepackage{multirow}
\usepackage{framed}
\usepackage{tikz}
\usetikzlibrary{matrix,arrows}
\usepackage[square,sort,comma,numbers]{natbib}
\usepackage{enumerate}




%\oddsidemargin.5cm
%\evensidemargin.5cm
\addtolength{\oddsidemargin}{-.525in}
\addtolength{\evensidemargin}{-.525in}
\addtolength{\textwidth}{1in}

\addtolength{\topmargin}{-.87in}
\addtolength{\textheight}{1.5in}




\newcommand\coolunder[2]{\mathrlap{\smash{\underbrace{\phantom{%
    \begin{matrix} #2 \end{matrix}}}_{\mbox{$#1$}}}}#2}

%\input{tableau}


\newcommand{\+}[1]{\ensuremath{\mathbf{#1}}}
\newcommand{\vect}[1]{\boldsymbol{#1}} % Uncomment for BOLD vectors.
%\newcommand{\vect}[1]{\vec{#1}} % Uncomment for ARROW vectors.
\newcommand{\C}{{\mathbb C}}
\newcommand{\Z}{{\mathbb Z}}
\newcommand{\QH}{{\mathbb H}}
\renewcommand{\P}{{\mathbb P}}
\newcommand{\OG}{\operatorname{OG}}
\newcommand{\OF}{\operatorname{OF}}
\newcommand{\bull}{{\scriptscriptstyle \bullet}}
\newcommand{\la}{\lambda}
\newcommand{\euler}[1]{\chi_{_{#1}}}
\newcommand{\cO}{{\mathcal O}}
\newcommand{\cG}{{\mathcal G}}
\newcommand{\cQ}{{\mathcal Q}}
\newcommand{\R}{{\mathbb R}}
\newcommand{\wt}{\widetilde}
\newcommand{\diag}{\operatorname{diag}}
\newcommand{\comp}{\operatorname{comp}}
\newcommand{\comment}[1]{}
\newcommand{\type}{\mathfrak{t}}
\newcommand{\op}{\text{op}}
\newcommand{\row}{{\bf r}}
\newcommand{\col}{{\bf c}}
\newcommand{\sym}{\mathfrak{S}}
\newcommand{\codim}{\text{codim}}
\DeclarePairedDelimiter{\ceil}{\lceil}{\rceil}
\DeclarePairedDelimiter{\floor}{\lfloor}{\rfloor}
\renewcommand{\emptyset}{\varnothing}

\newtheorem{thm}{Theorem}
\newtheorem{lemma}[thm]{Lemma}
\newtheorem{prop}[thm]{Proposition}
\newtheorem{cor}[thm]{Corollary}

\theoremstyle{definition}
\newtheorem{definition}[thm]{Definition}
\newtheorem{example}[thm]{Example}
\newtheorem{conj}[thm]{Conjecture}
\newtheorem{obs}[thm]{Observation}
\newtheorem{fact}[thm]{Fact}
\newtheorem{remark}[thm]{Remark}
\newtheorem{prob}{Problem}
\newtheorem{chal}{Challenge}

\begin{document}
\title{Problem Set 9, Version 2}
\date{October 27, 2015}
\author{Introduction to Manifolds}

\maketitle
Some typos were noticed in the previous version.  Corrections have been marked in red.
\begin{prob}
In order to get practice with pullbacks of forms,
do problems 19.1 - 19.4 in [Tu].  Do not submit these. 
\end{prob}

\begin{prob}
\begin{enumerate}
 \item Let $x^1, \ldots, x^n, y^1,
\ldots, y^n$ be a basis for $(\R^{2n})^*$.  Prove that the $2$-covector  $\sum^n_{j=1} dx^j \wedge dy^j$ is nondegenerate.
 \item Let $M$ be a smooth manifold.  Prove that $d\lambda$ is nondegenerate, where $\lambda$ is the Liouville $1$-form on $T^*M$.  (The $2$-form $-d\lambda$ is known as the 
 \emph{standard symplectic form} on $T^*M$.  Since
 a \emph{symplectic} form is a closed, nondegenerate
 $2$-form, this terminology is justified.)
\end{enumerate}
\end{prob}


\begin{prob}
\begin{enumerate}
 \item Suppose $0$ is a regular value of 
 $f(x,y) \in C^{\infty}(\R^2)$.
 Construct a nowhere-vanishing $1$-form
 on the one-dimensional submanifold $f^{-1}(0)$.
 \item Suppose $0$ is a regular value of 
 $f(x,y,z) \in C^{\infty}(\R^3)$.
 Construct a nowhere-vanishing {\color{red}$2$-form}
 on the two-dimensional submanifold $f^{-1}(0)$.
 Hint: first show that the equalities
 \[
 \frac{dx \wedge dy}{f_z} = \frac{dy \wedge dz}{f_x} = \frac{dz \wedge dx}{f_y}
 \]
 hold on $f^{-1}(0)$ whenever they make sense.
\end{enumerate}
\end{prob}

\begin{prob}
Let $\omega$ be a differential form, $X$ a vector field,
and $f$ a smooth function on a manifold $M$.
Recall that the Lie derivative $\mathfrak{L}_X\omega$ is
not $C^{\infty}(M)$-linear in either variable.
However, using Cartan's homotopy formula $\mathfrak{L}_X = d\iota_X + \iota_X\omega$,
prove that the Lie derivative does satisfy:
\[
\mathfrak{L}_{fX}\omega = f\mathfrak{L}_X\omega + df \wedge \iota_X\omega.
\]
\end{prob}

\begin{prob}
Prove that $[\mathfrak{L}_X, \iota_Y] = \iota_{[X,Y]}$
for any $X,Y \in \mathfrak{X}(M)$.  (For hints
see Problem 20.8 in [Tu].)
\end{prob}

\begin{prob}
Let $\omega = dx^1 \wedge \ldots \wedge dx^n \in \Omega^n(\R^n)$
and 
$X = \sum x^i \frac{\partial}{\partial x^i} \in \mathfrak{X}(\R^n)$ 
be the volume form and radial vector field respectively.
Compute the contraction $\iota_X\omega$.
\end{prob}

\begin{prob}
Consider the unit $2$-sphere $S^2 \subset \R^3$.
 Let $\omega = x dy \wedge dz - y dx \wedge dy + z dx \wedge dy \in \Omega^2(S^2)$
 and $X = -y \frac{\partial}{\partial x} + x \frac{\partial}{\partial y} \in \mathfrak{X}(S^2)$.
 Compute the Lie derivative {\color{red}$\mathfrak{L}_X\omega$}.
\end{prob}



\end{document}