
\documentclass{amsart}
%\usepackage{cmbright}
\usepackage{amsmath, amsfonts, amssymb, amscd}
%\usepackage{tableau}
\usepackage{color}
\usepackage{xcolor}
\usepackage{graphicx}
\usepackage{array}
\usepackage{mathtools}
\usepackage{multirow}
\usepackage{framed}
\usepackage{tikz}
\usetikzlibrary{matrix,arrows}
\usepackage[square,sort,comma,numbers]{natbib}
\usepackage{enumerate}




%\oddsidemargin.5cm
%\evensidemargin.5cm
\addtolength{\oddsidemargin}{-.525in}
\addtolength{\evensidemargin}{-.525in}
\addtolength{\textwidth}{1in}

\addtolength{\topmargin}{-.87in}
\addtolength{\textheight}{1.5in}




\newcommand\coolunder[2]{\mathrlap{\smash{\underbrace{\phantom{%
    \begin{matrix} #2 \end{matrix}}}_{\mbox{$#1$}}}}#2}

%\input{tableau}


\newcommand{\+}[1]{\ensuremath{\mathbf{#1}}}
\newcommand{\vect}[1]{\boldsymbol{#1}} % Uncomment for BOLD vectors.
%\newcommand{\vect}[1]{\vec{#1}} % Uncomment for ARROW vectors.
\newcommand{\C}{{\mathbb C}}
\newcommand{\Z}{{\mathbb Z}}
\renewcommand{\P}{{\mathbb P}}
\newcommand{\OG}{\operatorname{OG}}
\newcommand{\OF}{\operatorname{OF}}
\newcommand{\bull}{{\scriptscriptstyle \bullet}}
\newcommand{\la}{\lambda}
\newcommand{\euler}[1]{\chi_{_{#1}}}
\newcommand{\cO}{{\mathcal O}}
\newcommand{\cG}{{\mathcal G}}
\newcommand{\cQ}{{\mathcal Q}}
\newcommand{\R}{{\mathbb R}}
\newcommand{\wt}{\widetilde}
\newcommand{\diag}{\operatorname{diag}}
\newcommand{\comp}{\operatorname{comp}}
\newcommand{\comment}[1]{}
\newcommand{\type}{\mathfrak{t}}
\newcommand{\op}{\text{op}}
\newcommand{\row}{{\bf r}}
\newcommand{\col}{{\bf c}}
\newcommand{\sym}{\mathfrak{S}}
\newcommand{\codim}{\text{codim}}
\DeclarePairedDelimiter{\ceil}{\lceil}{\rceil}
\DeclarePairedDelimiter{\floor}{\lfloor}{\rfloor}
\renewcommand{\emptyset}{\varnothing}

\newtheorem{thm}{Theorem}
\newtheorem{lemma}[thm]{Lemma}
\newtheorem{prop}[thm]{Proposition}
\newtheorem{cor}[thm]{Corollary}

\theoremstyle{definition}
\newtheorem{definition}[thm]{Definition}
\newtheorem{example}[thm]{Example}
\newtheorem{conj}[thm]{Conjecture}
\newtheorem{obs}[thm]{Observation}
\newtheorem{fact}[thm]{Fact}
\newtheorem{remark}[thm]{Remark}
\newtheorem{prob}{Problem}
\newtheorem{chal}{Challenge}

\begin{document}
\title{Problem Set 7}
\date{March 14, 2018}
\author{Analysis II}

\maketitle

\begin{prob}
 Suppose $f^{-1}(c)$ is a regular level set of
 a smooth function $f: A \subset \R^{k+n} \to \R^n$.
 Take $(a,b) \in f^{-1}(c)$.  By the implicit function
 theorem, there exists a smooth function $g: U \subset \R^k \to \R^n$
 on an open set $U$ containing $a$,  such that $g(U) \subset f^{-1}(c)$ and $g(a) = b$.
 Recall that the tangent space $T_{(a,b)}f^{-1}(c) \subset T_{(a,b)}\R^{k+n}$
 is defined to be the orthogonal complement of the row space of $Df(a,b)$ (i.e.
 the kernel of the linear transformation $y \mapsto Df(a,b)y$).
 Show that $T_{(a,b)}f^{-1}(c)$ is also equal to the column space of $DG(a)$,
 where $G:\R^k \to \R^{k+n}$
 is defined by $x \mapsto (x,g(x))$.
\end{prob}

\begin{prob}
 Give a basis for the tangent space and the normal space of the set $S$
 at the point $p$.   Also give equations defining the tangent space and the normal space
 as a subset of $\R^{m}$.
 \begin{enumerate}[(a)]
  \item Let $S$ be the intersection of
  the surfaces $x^2 + y^2 - z^2 = 1$ and
  $x +y + z = 5$ in $\R^3$.  Let $p = (1,2,2)$.
  \item Let $S$ be the surface $z = \ln(\sqrt{x^2 + y^2})$,
  and let $p = (1,-1,\ln(2)/2)$.
 \end{enumerate}
\end{prob}

\begin{prob}
Use the method of Lagrange multipliers
to 
\begin{enumerate}
 \item
find
the minimum value of $x^2 + y^2 + z^2$ subject
to the constraints $x+y-z = 0$
and $x+3y+z = 2$.
\item
find the minimum value of $xyz$
on $f^{-1}(1) \cap \{(x,y,z): x>0, y>0, z>0\}$ , where
\[
f(x,y,z) = \frac{1}{x} + \frac{1}{y} + \frac{1}{z}.
\]
\end{enumerate}
\end{prob}

\begin{prob}
\begin{enumerate}[(a)]
 \item
 Let $f: \R^{2n} \to \R$ be the Euclidean
 dot product of the first $n$ variables
 with the last; i.e. $f(x,y) = \langle x, y \rangle$.
 Use the method of Lagrange multipliers
 to show that $|f(x,y)| \leq 1$ on the
 set $S = \{(x,y): ||x||^2 = ||y||^2 = 1\}$. 
\item Use the previous part
to prove the Cauchy Schwartz Inequality.
In particular, for arbitrary $x$ and $y$ in $\R^n$,
show that $|f(x,y)|^2 \leq ||x||^2 \cdot ||y||^2$.
\end{enumerate}
\end{prob}



\begin{prob}
 Show that the function $xyz(x+y+z-1)$
 has one non-degenerate critical point
 and an infinite set of degenerate
 critical points.
 Show that the non-degenerate
 critical point is a local minimum.
\end{prob}


\end{document}