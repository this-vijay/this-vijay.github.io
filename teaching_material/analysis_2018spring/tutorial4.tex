
\documentclass{amsart}
%\usepackage{cmbright}
\usepackage{amsmath, amsfonts, amssymb, amscd}
%\usepackage{tableau}
\usepackage{color}
\usepackage{xcolor}
\usepackage{graphicx}
\usepackage{array}
\usepackage{mathtools}
\usepackage{multirow}
\usepackage{framed}
\usepackage{tikz}
\usetikzlibrary{matrix,arrows}
\usepackage[square,sort,comma,numbers]{natbib}
\usepackage{enumerate}




%\oddsidemargin.5cm
%\evensidemargin.5cm
\addtolength{\oddsidemargin}{-.525in}
\addtolength{\evensidemargin}{-.525in}
\addtolength{\textwidth}{1in}

\addtolength{\topmargin}{-.87in}
\addtolength{\textheight}{1.5in}




\newcommand\coolunder[2]{\mathrlap{\smash{\underbrace{\phantom{%
    \begin{matrix} #2 \end{matrix}}}_{\mbox{$#1$}}}}#2}

%\input{tableau}


\newcommand{\+}[1]{\ensuremath{\mathbf{#1}}}
\newcommand{\vect}[1]{\boldsymbol{#1}} % Uncomment for BOLD vectors.
%\newcommand{\vect}[1]{\vec{#1}} % Uncomment for ARROW vectors.
\newcommand{\C}{{\mathbb C}}
\newcommand{\Z}{{\mathbb Z}}
\renewcommand{\P}{{\mathbb P}}
\newcommand{\OG}{\operatorname{OG}}
\newcommand{\OF}{\operatorname{OF}}
\newcommand{\bull}{{\scriptscriptstyle \bullet}}
\newcommand{\la}{\lambda}
\newcommand{\euler}[1]{\chi_{_{#1}}}
\newcommand{\cO}{{\mathcal O}}
\newcommand{\cG}{{\mathcal G}}
\newcommand{\cQ}{{\mathcal Q}}
\newcommand{\R}{{\mathbb R}}
\newcommand{\wt}{\widetilde}
\newcommand{\diag}{\operatorname{diag}}
\newcommand{\comp}{\operatorname{comp}}
\newcommand{\comment}[1]{}
\newcommand{\type}{\mathfrak{t}}
\newcommand{\op}{\text{op}}
\newcommand{\row}{{\bf r}}
\newcommand{\col}{{\bf c}}
\newcommand{\sym}{\mathfrak{S}}
\newcommand{\codim}{\text{codim}}
\DeclarePairedDelimiter{\ceil}{\lceil}{\rceil}
\DeclarePairedDelimiter{\floor}{\lfloor}{\rfloor}
\renewcommand{\emptyset}{\varnothing}

\newtheorem{thm}{Theorem}
\newtheorem{lemma}[thm]{Lemma}
\newtheorem{prop}[thm]{Proposition}
\newtheorem{cor}[thm]{Corollary}

\theoremstyle{definition}
\newtheorem{definition}[thm]{Definition}
\newtheorem{example}[thm]{Example}
\newtheorem{conj}[thm]{Conjecture}
\newtheorem{obs}[thm]{Observation}
\newtheorem{fact}[thm]{Fact}
\newtheorem{remark}[thm]{Remark}
\newtheorem{prob}{Problem}
\newtheorem{chal}{Challenge}

\begin{document}
\title{Tutorial 4}
\date{March 23, 2018}
\author{Analysis II}

\maketitle

\begin{prob}
Prove that $\sigma: \R^m \times \R^m \to \R$ is a bilinear
if and only if it is of the form $(x,y) \mapsto x^t A y$
for some $m \times m$ matrix $A$.
\end{prob}


\begin{prob}
Let $A$ be a symmetric $m \times m$ matrix.
We say that $A$ is \emph{positive definite} if 
 all its eigenvalues are positive.
 \begin{enumerate}[(a)]
\item Prove that $A$ is positive definite if and only if $x^t A x > 0$ for all $x \neq 0$
 (i.e. if and only if the map from $\R^m \times \R^m \to \R$ defined by $(x,y) \mapsto x^t A y$ is positive definite).
\item 
\emph{Sylvester's Criterion} states that a symmetric matrix is positive definite if and only if
all principle minors of $A$ (that is, submatrices obtained by deleting the last $k$ rows and the last $k$ columns for some $0 \leq k < m$) 
have positive determinant.
Prove this result when $A$ is a $2 \times 2$ symmetric matrix.
\end{enumerate}
\end{prob}

\begin{prob}
 Let $\phi: \R^m \to \R$ be function of class $C^2$.
 Let $x$ be a critical point of $\phi$ (i.e. suppose $\nabla \phi (x) = 0$).
 Let \[H = \left(\frac{\partial^2\phi}{\partial{x_i}\partial{x_j}}(x)\right)\] be the Hessian of $\phi$ at $x$.
 In class we defined $\frac{\partial^2\phi}{\partial v^2}(x)$ to be $\frac{d^2}{dt^2}[\phi(x+tv)]_{t=0}$.
 Prove that \[\frac{\partial^2\phi}{\partial v^2}(x) = v^t H v\]
 for any $v \in \R^m$.
 (Hint: it will be useful to prove that $\frac{\partial^2\phi}{\partial v^2}(x) = \frac{\partial}{\partial v}(\frac{\partial}{\partial v}\phi)(x)$.)
\end{prob}


\begin{prob}
 Let $f^{-1}(x)$ be a regular level set in $\R^m$, and
 let $x$ be a point in $f^{-1}(c)$.
 A \emph{curve} is a map  $\gamma: \R \to \R^m$ of class $C^1$.
 We say a curve $\gamma$ is \emph{based at $x$} if $\gamma(0) = x$.
 In this case we call $\gamma'(0) \in T_x\R^m$ the \emph{initial velocity vector} of $\gamma$.
 We say a curve \emph{lies in} $f^{-1}(c)$ if every point of its image is contained in $f^{-1}(c)$.
 Let $V$ be the set of all initial velocity vectors of curves lying in $f^{-1}(c)$ and based at $x$.
 Prove that $V = T_x f^{-1}(c) := \ker Df(x)$.  In particular:
 \begin{enumerate}[(a)]
  \item Show that $V$ is a vector space. (One approach: First consider the set $W$, consisting of all velocity vectors
  lying in an open ball $U \subset \R^k$ based at a point $a \in U \subset \R^k$.
  Show that $W$ is a vector space.  Then use the parametrization
  of $f^{-1}(c)$ about $x$ that we get from the implicit function theorem,
  in order to establish a bijection between elements of $V$ and elements of $W$.)
  \item Show that $V \subset  \ker Df(x)$.
  \item Show that $V \supset  \ker Df(x)$.
 \end{enumerate}
\end{prob}


\end{document}