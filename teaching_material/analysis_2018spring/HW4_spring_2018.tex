
\documentclass{amsart}
%\usepackage{cmbright}
\usepackage{amsmath, amsfonts, amssymb, amscd}
%\usepackage{tableau}
\usepackage{color}
\usepackage{xcolor}
\usepackage{graphicx}
\usepackage{array}
\usepackage{mathtools}
\usepackage{multirow}
\usepackage{framed}
\usepackage{tikz}
\usetikzlibrary{matrix,arrows}
\usepackage[square,sort,comma,numbers]{natbib}
\usepackage{enumerate}




%\oddsidemargin.5cm
%\evensidemargin.5cm
\addtolength{\oddsidemargin}{-.525in}
\addtolength{\evensidemargin}{-.525in}
\addtolength{\textwidth}{1in}

\addtolength{\topmargin}{-.87in}
\addtolength{\textheight}{1.5in}




\newcommand\coolunder[2]{\mathrlap{\smash{\underbrace{\phantom{%
    \begin{matrix} #2 \end{matrix}}}_{\mbox{$#1$}}}}#2}

%\input{tableau}


\newcommand{\+}[1]{\ensuremath{\mathbf{#1}}}
\newcommand{\vect}[1]{\boldsymbol{#1}} % Uncomment for BOLD vectors.
%\newcommand{\vect}[1]{\vec{#1}} % Uncomment for ARROW vectors.
\newcommand{\C}{{\mathbb C}}
\newcommand{\Z}{{\mathbb Z}}
\renewcommand{\P}{{\mathbb P}}
\newcommand{\OG}{\operatorname{OG}}
\newcommand{\OF}{\operatorname{OF}}
\newcommand{\bull}{{\scriptscriptstyle \bullet}}
\newcommand{\la}{\lambda}
\newcommand{\euler}[1]{\chi_{_{#1}}}
\newcommand{\cO}{{\mathcal O}}
\newcommand{\cG}{{\mathcal G}}
\newcommand{\cQ}{{\mathcal Q}}
\newcommand{\R}{{\mathbb R}}
\newcommand{\wt}{\widetilde}
\newcommand{\diag}{\operatorname{diag}}
\newcommand{\comp}{\operatorname{comp}}
\newcommand{\comment}[1]{}
\newcommand{\type}{\mathfrak{t}}
\newcommand{\op}{\text{op}}
\newcommand{\row}{{\bf r}}
\newcommand{\col}{{\bf c}}
\newcommand{\sym}{\mathfrak{S}}
\newcommand{\codim}{\text{codim}}
\DeclarePairedDelimiter{\ceil}{\lceil}{\rceil}
\DeclarePairedDelimiter{\floor}{\lfloor}{\rfloor}
\renewcommand{\emptyset}{\varnothing}

\newtheorem{thm}{Theorem}
\newtheorem{lemma}[thm]{Lemma}
\newtheorem{prop}[thm]{Proposition}
\newtheorem{cor}[thm]{Corollary}

\theoremstyle{definition}
\newtheorem{definition}[thm]{Definition}
\newtheorem{example}[thm]{Example}
\newtheorem{conj}[thm]{Conjecture}
\newtheorem{obs}[thm]{Observation}
\newtheorem{fact}[thm]{Fact}
\newtheorem{remark}[thm]{Remark}
\newtheorem{prob}{Problem}
\newtheorem{chal}{Challenge}

\begin{document}
\title{Problem Set 4}
\date{January 30, 2018}
\author{Analysis II}

\maketitle

\begin{prob}
 Prove that any linear transformation from $\R^m \to \R^n$ is continuous.
\end{prob}

\begin{prob}
Suppose $f$ is a  function from a metric space $(X,d_X)$ to a metric space $(Y,d_Y)$.
Prove that $f$ is continuous if and only if $f^{-1}(U)$ is open in $X$ for every open set $U$ of $Y$. 
\end{prob}

\begin{prob}
\begin{enumerate}[(a)]
 \item Show that if $Q$ is a rectangle, then $Q$ equals the closure
 of Int$Q$.
 \item If $D$ is a closed set, what is the relation in general between
 the set $D$ and the closure of Int$D$?
 \item If $U$ is an open set, what is the relation in general
 between the set $U$ and the interior of the closure of $U$?
\end{enumerate}
 \end{prob}

\begin{prob}
 Let $A = \{(x,y) \in \R^2: x > 0 \text{ and } 0 < y < x^2\}$.
\begin{enumerate}[(a)]
 \item Show that every straight line through $(0,0)$
 contains an interval around $(0,0)$ which is in $\R^2 \setminus A$.
 \item Define $f:\R^2 \to \R$ by
 \[
 f(x) = \begin{cases}
     0 &\text{ if } x \not\in A, \text{ and}\\
     1 &\text{ if } x \in A.
    \end{cases}
 \]
 For $h \in \R^2$ define $g_h: \R \to \R$
 by $g_h(t) = f(th)$.
 Show that each $g_h$ is continuous at $0$, but
 $f$ is not continuous at $(0,0)$.
 \end{enumerate}
 \end{prob}



\begin{prob}
 Let $\R^{\infty} = \bigcup^{\infty}_{n=1} \R^n$ where we consider the natural inclusions $\R^n \subset \R^{n+1}$.  
 (You can also think of this as those elements of $\R^\omega$ with finitely many nonzero entries.)
 Note that the dot product gives a well-defined inner product on  $\R^{\infty}$, and hence induces a metric.
 Define $e_i = (0, 0, \ldots, 0, 1, 0, 0, \ldots )$ where $1$ appears in the $i^\text{th}$ place.
 Prove that $X := \{e_i: i \in \mathbb{N}\}$ forms a basis for $\R^{\infty}$, and that $X$ is
 closed, bounded, and noncompact. 
\end{prob}


\vspace{5mm}
\begin{chal}
Let $(X,d)$ be a metric space.  Show that a subset $A \subset X$ is compact (i.e. every open cover of $A$
has a finite subcover) if and only if $A$ is sequentially compact (i.e. every sequence in $A$ has a subsequence
that converges to a point in $A$). 
\end{chal}

\vspace{5mm}

*All questions taken from \emph{Analysis on Manifolds} by James Munkres and \emph{Calculus on Manifolds} by Michael Spivak.


\end{document}