
\documentclass{amsart}
%\usepackage{cmbright}
\usepackage{amsmath, amsfonts, amssymb, amscd}
%\usepackage{tableau}
\usepackage{color}
\usepackage{xcolor}
\usepackage{graphicx}
\usepackage{array}
\usepackage{mathtools}
\usepackage{multirow}
\usepackage{framed}
\usepackage{tikz}
\usetikzlibrary{matrix,arrows}
\usepackage[square,sort,comma,numbers]{natbib}
\usepackage{enumerate}
\usepackage{hyperref}



%\oddsidemargin.5cm
%\evensidemargin.5cm
\addtolength{\oddsidemargin}{-.525in}
\addtolength{\evensidemargin}{-.525in}
\addtolength{\textwidth}{1in}

\addtolength{\topmargin}{-.87in}
\addtolength{\textheight}{1.5in}




\newcommand\coolunder[2]{\mathrlap{\smash{\underbrace{\phantom{%
    \begin{matrix} #2 \end{matrix}}}_{\mbox{$#1$}}}}#2}

%\input{tableau}


\newcommand{\+}[1]{\ensuremath{\mathbf{#1}}}
\newcommand{\vect}[1]{\boldsymbol{#1}} % Uncomment for BOLD vectors.
%\newcommand{\vect}[1]{\vec{#1}} % Uncomment for ARROW vectors.
\newcommand{\C}{{\mathbb C}}
\newcommand{\Z}{{\mathbb Z}}
\renewcommand{\P}{{\mathbb P}}
\newcommand{\OG}{\operatorname{OG}}
\newcommand{\OF}{\operatorname{OF}}
\newcommand{\bull}{{\scriptscriptstyle \bullet}}
\newcommand{\la}{\lambda}
\newcommand{\euler}[1]{\chi_{_{#1}}}
\newcommand{\cO}{{\mathcal O}}
\newcommand{\cG}{{\mathcal G}}
\newcommand{\cQ}{{\mathcal Q}}
\newcommand{\R}{{\mathbb R}}
\newcommand{\wt}{\widetilde}
\newcommand{\diag}{\operatorname{diag}}
\newcommand{\comp}{\operatorname{comp}}
\newcommand{\comment}[1]{}
\newcommand{\type}{\mathfrak{t}}
\newcommand{\op}{\text{op}}
\newcommand{\row}{{\bf r}}
\newcommand{\col}{{\bf c}}
\newcommand{\sym}{\mathfrak{S}}
\newcommand{\codim}{\text{codim}}
\DeclarePairedDelimiter{\ceil}{\lceil}{\rceil}
\DeclarePairedDelimiter{\floor}{\lfloor}{\rfloor}
\renewcommand{\emptyset}{\varnothing}

\newtheorem{thm}{Theorem}
\newtheorem{lemma}[thm]{Lemma}
\newtheorem{prop}[thm]{Proposition}
\newtheorem{cor}[thm]{Corollary}

\theoremstyle{definition}
\newtheorem{definition}[thm]{Definition}
\newtheorem{example}[thm]{Example}
\newtheorem{conj}[thm]{Conjecture}
\newtheorem{obs}[thm]{Observation}
\newtheorem{fact}[thm]{Fact}
\newtheorem{remark}[thm]{Remark}
\newtheorem{prob}{Problem}
\newtheorem{chal}{Challenge}

\begin{document}
\title{Problem Set 10}
\date{April 12, 2018}
\author{Analysis II}

\maketitle


\begin{prob}
 Let $V$ be a vector space.  We say two ordered bases $\beta$ and $\beta'$ for $V$ are \emph{equivalent} if there exists
 a linear transformation $A:V \to V$ with positive determinant taking $\beta$ to $\beta'$.
 \begin{enumerate}[(a)]
  \item Show that there are exactly two equivalence classes of ordered bases for $V$.
  We define an \emph{orientation} on $V$ to be a choice of one of these equivalence classes.
  \item Suppose $(V,[\beta])$ is an \emph{oriented vector space}; i.e. a vector space along with a choice of orientation.
  What are the \emph{orientation preserving} linear transformations from $(V,[\beta])$ to $(V,[\beta])$? What about from
  $(V, [\beta])$ to $(V, \text{diag}(-1,1,\ldots,1)[\beta])$?
  \item More generally, an \emph{orientation} on $\R^n$ is a continuously varying choice of 
  ordered basis for every point in $\R^n$.  We say an invertible differentiable map $f: \R^n \to \R^n$ is \emph{orientation preserving} if $Df(p)$
  has positive determinant for every $p$.
  Give an example of a map from $\R^2$ to $\R^2$ that is not orientation preserving.  Is it possible for a map to be
  orientation preserving in some region of $\R^n$, but not orientation preserving in another region?  
 \end{enumerate}
\end{prob}

\vspace{5mm}

\hrule

\vspace{5mm}


 Let $\Sigma$ denote the Riemann sphere, which we realize as the unit sphere in $\R^3$.  Identify the $xy$-plane with
 the complex plane $\C$ in the standard way.  Let $N$ denote the north pole $(0,0,1)$.
 As in class, we define the stereographic projection $\Phi: \C \to \Sigma$ as the map sending a point $p \in \C$
 to the point $\hat{p} := (\Sigma \setminus \{N\}) \cap \overline{pN}$, where $\overline{pN}$ denotes the line through $p$ and $N$.
 
\begin{prob}
 Prove that $\Phi$ takes circles to circles.
\end{prob}

\begin{prob}
 Prove that the geometric inversion map $\{z \mapsto \frac{1}{\bar{z}}\}$ corresponds to 
 reflection of the Riemann sphere $\Sigma$ through the complex plane (i.e the $xy$-plane).
 Using the fact that compositions of conformal maps are conformal, show
 that $\{z \mapsto \frac{1}{z}\}$ is conformal everywhere.
\end{prob}

\begin{prob}
 Give an explicit formula for $\Phi: \R^2 \to \R^3$.  Give an explicit formula for its inverse.
\end{prob}

\vspace{5mm}

\hrule

\vspace{5mm}

\begin{prob}
Let $C$ denote the unit circle in the $xy$-plane.
 As mentioned in class, the cylindrical projection of Archimedes is a map from $\Sigma \setminus \{N,S\} \to C \times [-1,1]$
 defined by $(x,y,z) \mapsto (\frac{x}{\sqrt{x^2+y^2}},\frac{y}{\sqrt{x^2+y^2}},z)$.
 Prove that this map preserves area, assuming the following facts about area of surfaces:
 \begin{enumerate}[(i)]
\item The area of a cylinder is given by the height times the circumference.
\item The area of a frustrum of a cone is $2\pi$ times the product of the radius and the slant height.
\item The area of a small slice of the sphere is approximated by the area of the frustrum of a cone with appropriate radius and slant height.
\end{enumerate}
(You can find a sketch of the proof \href{http://www.cmi.ac.in/~vijayr/analysis_2018spring/projections.pdf}{here}.)
\end{prob}




\end{document}