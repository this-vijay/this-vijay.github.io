
\documentclass{amsart}
%\usepackage{cmbright}
\usepackage{amsmath, amsfonts, amssymb, amscd}
%\usepackage{tableau}
\usepackage{color}
\usepackage{xcolor}
\usepackage{graphicx}
\usepackage{array}
\usepackage{mathtools}
\usepackage{multirow}
\usepackage{framed}
\usepackage{tikz}
\usetikzlibrary{matrix,arrows}
\usepackage[square,sort,comma,numbers]{natbib}
\usepackage{enumerate}




%\oddsidemargin.5cm
%\evensidemargin.5cm
\addtolength{\oddsidemargin}{-.525in}
\addtolength{\evensidemargin}{-.525in}
\addtolength{\textwidth}{1in}

\addtolength{\topmargin}{-.87in}
\addtolength{\textheight}{1.5in}




\newcommand\coolunder[2]{\mathrlap{\smash{\underbrace{\phantom{%
    \begin{matrix} #2 \end{matrix}}}_{\mbox{$#1$}}}}#2}

%\input{tableau}


\newcommand{\+}[1]{\ensuremath{\mathbf{#1}}}
\newcommand{\vect}[1]{\boldsymbol{#1}} % Uncomment for BOLD vectors.
%\newcommand{\vect}[1]{\vec{#1}} % Uncomment for ARROW vectors.
\newcommand{\C}{{\mathbb C}}
\newcommand{\Z}{{\mathbb Z}}
\renewcommand{\P}{{\mathbb P}}
\newcommand{\OG}{\operatorname{OG}}
\newcommand{\OF}{\operatorname{OF}}
\newcommand{\bull}{{\scriptscriptstyle \bullet}}
\newcommand{\la}{\lambda}
\newcommand{\euler}[1]{\chi_{_{#1}}}
\newcommand{\cO}{{\mathcal O}}
\newcommand{\cG}{{\mathcal G}}
\newcommand{\cQ}{{\mathcal Q}}
\newcommand{\R}{{\mathbb R}}
\newcommand{\wt}{\widetilde}
\newcommand{\diag}{\operatorname{diag}}
\newcommand{\comp}{\operatorname{comp}}
\newcommand{\comment}[1]{}
\newcommand{\type}{\mathfrak{t}}
\newcommand{\op}{\text{op}}
\newcommand{\row}{{\bf r}}
\newcommand{\col}{{\bf c}}
\newcommand{\sym}{\mathfrak{S}}
\newcommand{\codim}{\text{codim}}
\DeclarePairedDelimiter{\ceil}{\lceil}{\rceil}
\DeclarePairedDelimiter{\floor}{\lfloor}{\rfloor}
\renewcommand{\emptyset}{\varnothing}

\newtheorem{thm}{Theorem}
\newtheorem{lemma}[thm]{Lemma}
\newtheorem{prop}[thm]{Proposition}
\newtheorem{cor}[thm]{Corollary}

\theoremstyle{definition}
\newtheorem{definition}[thm]{Definition}
\newtheorem{example}[thm]{Example}
\newtheorem{conj}[thm]{Conjecture}
\newtheorem{obs}[thm]{Observation}
\newtheorem{fact}[thm]{Fact}
\newtheorem{remark}[thm]{Remark}
\newtheorem{prob}{Problem}
\newtheorem{chal}{Challenge}

\begin{document}
\title{Tutorial 1}
\date{Jan 5, 2018}
\author{Analysis II}

\maketitle



{\bf I. If you haven't already:}

\vspace{3mm}

\begin{prob}
Let $f$ be a bounded function on an interval $[a,b]$ and let $c$ be a point in $(a,b)$.
Prove that $f$ is integrable on $[a,b]$
if and only if $f$ is integrable on both $[a,c]$ and $[c,b]$. 
\end{prob}


\vspace{3mm}
{\bf II. Integrability and sequences of functions:}
\vspace{3mm}

\begin{prob}
 Provide an example or explain why the request is impossible:
 \begin{enumerate}[(a)]
  \item A sequence $(f_n) \to f$ pointwise where each $f_n$ has (at most) finitely many discontinuities
  but $f$ is not integrable.
  \item A sequence $(g_n) \to g$ uniformly where each $g_n$ has finitely many discontinuities
  but $g$ is not integrable.
  \item A sequence $(h_n) \to h$ uniformly where each $h_n$ is not integrable but
   $h$ is  integrable.
 \end{enumerate}
\end{prob}


\vspace{3mm}
{\bf III. Integrability of some pathological functions:}
\vspace{3mm}

\begin{prob}
Consider Thomae's function $t(x)$ on the interval $[0,1]$:
\[
t(x) = \begin{cases}
        1 &\text{ if } x = 0 \\
        1/n &\text{ if } x = m/n \in \mathbb{Q} \setminus \{0\} \text{ is in lowest terms } \\
        0 &\text{ if } x \not\in \mathbb{Q}.
       \end{cases}
\]
Recall that $t(x)$ is continuous at all irrational points and
has discontinuities at all rational points (why is this?).
Show that $t(x)$ is integrable on $[0,1]$. 
\end{prob}

\begin{prob}
 Let $C$ denote the Cantor set in $[0,1]$.
 Define a Cantor function on $[0,1]$ by
 \[
h(x) = \begin{cases}
        1 &\text{ if } x \in C \\
        0 &\text{ if } x \not\in C.
       \end{cases}
 \]
The function $h(x)$ is continuous on all points
not in $C$, and has discontinuities at all points in $C$ (why  is this?).
Show that $h(x)$ is integrable on $[0,1]$.
\end{prob}

\vspace{3mm}
{\bf IV. The measure of the set of discontinuities:}
\vspace{3mm}

Let $E$ be a subset of $\R$.
We define the \emph{outer measure} of $E$ to be
\[
\lambda^*(E) = \inf\{\sum^{\infty}_{n=1} \ell(I_n): (I_n) \text{ is a sequence of closed intervals covering } E\}.
\]
We say that a set $E \subset \R$ has \emph{measure zero} if $\lambda^*(E) = 0$: i.e. if
for any $\epsilon > 0$
there exists a sequence of closed intervals covering $E$ with total length less than $\epsilon$. 

\begin{prob}
 Show that any countably infinite set has measure zero.
\end{prob}

\begin{prob}
 Show that the Cantor set has measure zero.
\end{prob}


\vspace{5mm}

*All questions taken from \emph{Understanding Analysis: 2nd Edition} by Stephen Abbott.


\end{document}