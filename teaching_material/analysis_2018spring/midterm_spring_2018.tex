
\documentclass{amsart}
%\usepackage{cmbright}
\usepackage{amsmath, amsfonts, amssymb, amscd}
%\usepackage{tableau}
\usepackage{color}
\usepackage{xcolor}
\usepackage{graphicx}
\usepackage{array}
\usepackage{mathtools}
\usepackage{multirow}
\usepackage{framed}
\usepackage{tikz}
\usetikzlibrary{matrix,arrows}
\usepackage[square,sort,comma,numbers]{natbib}
\usepackage{enumerate}
\usepackage{comment}



%\oddsidemargin.5cm
%\evensidemargin.5cm
\addtolength{\oddsidemargin}{-.525in}
\addtolength{\evensidemargin}{-.525in}
\addtolength{\textwidth}{1in}

\addtolength{\topmargin}{-.87in}
\addtolength{\textheight}{1.5in}




\newcommand\coolunder[2]{\mathrlap{\smash{\underbrace{\phantom{%
    \begin{matrix} #2 \end{matrix}}}_{\mbox{$#1$}}}}#2}

%\input{tableau}


\newcommand{\+}[1]{\ensuremath{\mathbf{#1}}}
\newcommand{\vect}[1]{\boldsymbol{#1}} % Uncomment for BOLD vectors.
%\newcommand{\vect}[1]{\vec{#1}} % Uncomment for ARROW vectors.
\newcommand{\C}{{\mathbb C}}
\newcommand{\Z}{{\mathbb Z}}
\renewcommand{\P}{{\mathbb P}}
\newcommand{\OG}{\operatorname{OG}}
\newcommand{\OF}{\operatorname{OF}}
\newcommand{\bull}{{\scriptscriptstyle \bullet}}
\newcommand{\la}{\lambda}
\newcommand{\euler}[1]{\chi_{_{#1}}}
\newcommand{\cO}{{\mathcal O}}
\newcommand{\cG}{{\mathcal G}}
\newcommand{\cQ}{{\mathcal Q}}
\newcommand{\R}{{\mathbb R}}
\newcommand{\wt}{\widetilde}
\newcommand{\diag}{\operatorname{diag}}
\newcommand{\comp}{\operatorname{comp}}
\newcommand{\type}{\mathfrak{t}}
\newcommand{\op}{\text{op}}
\newcommand{\row}{{\bf r}}
\newcommand{\col}{{\bf c}}
\newcommand{\sym}{\mathfrak{S}}
\newcommand{\codim}{\text{codim}}
\DeclarePairedDelimiter{\ceil}{\lceil}{\rceil}
\DeclarePairedDelimiter{\floor}{\lfloor}{\rfloor}
\renewcommand{\emptyset}{\varnothing}

\newtheorem{thm}{Theorem}
\newtheorem{lemma}[thm]{Lemma}
\newtheorem{prop}[thm]{Proposition}
\newtheorem{cor}[thm]{Corollary}

\theoremstyle{definition}
\newtheorem{definition}[thm]{Definition}
\newtheorem{example}[thm]{Example}
\newtheorem{conj}[thm]{Conjecture}
\newtheorem{obs}[thm]{Observation}
\newtheorem{fact}[thm]{Fact}
\newtheorem{remark}[thm]{Remark}
\newtheorem{prob}{Problem}
\newtheorem{chal}{Challenge}

\begin{document}
\title{Midterm Exam}
\date{February 22, 2018}
\author{Intro to Real Analysis II}

\maketitle


The exam consists of five questions, each worth $20$ points (but there are also $5$ bonus points available).
You may take up to three hours to complete the exam.

\vspace{3mm}


\begin{prob}
Consider a function $f: \R^m \to \R$.
Determine whether each statement about $f$ is true or false.
Give a brief explanation (one or two lines).
If false, then try to correct the statement and provide a counterexample.
\begin{enumerate}[(a)]
 \item Suppose all directional derivatives of $f$ exist at a point $a \in \R^m$.
Then $D(f,\+u + \+v) (a) = D(f,\+u)(a) + D(f,\+v)(a)$ for any two vectors $\+u$ and $\+v$ in $\R^m$.
\item Suppose $f$ is differentiable at a point $a$.
Then $f$ is continuous at $a$.
\item Suppose $f$ is differentiable everywhere.  Then $f$ has an
extreme value (i.e. a maximum or a minimum) at $a \in \R^m$ if and only if
all the partial derivatives of $f$ vanish at $a$.
\end{enumerate}
\end{prob}

\vspace{3mm}


\begin{prob}
Find partial derivatives for the following functions,
where $g: \R \to \R$ is continuous:
\begin{enumerate}[(a)]
 \item $f(x,y) = \int^{x+y}_a g$.
 \item $f(x,y) = \int^{xy}_a g$.
\end{enumerate}
\end{prob}


\vspace{3mm}


\begin{prob}
\begin{enumerate}[(a)]
 \item Show that the sup norm and the euclidean norm 
 on $\R^n$ are \emph{topologically equivalent}, in the sense that
 a set $U \subset \R^n$ is open under the sup norm
 if and only if it is open under the euclidean norm.
 \item Show that any linear map $T: \R^m \to \R^n$
 is continuous (you can use part (a), but you are also free to 
 supply a different proof).
\end{enumerate}
\end{prob}

\vspace{3mm}




\begin{prob}
\begin{enumerate}[(a)]
 \item 
 We say a differentiable function $f: \R^2 \to \R$
 is an \emph{antiderivative} of a pair of
 functions $g_1$ and $g_2$
 from $\R^2 \to \R$
 if  $D_1 f = g_1$ and $D_2 f = g_2$.
  For each pair of functions $g_1$ and $g_2$,
 construct an antiderivative $f$,
 or show that no such $f$ exists.
 \begin{enumerate}[(i)]
  \item $g_1(x,y) = x$, $g_2(x,y) = y$.
  \item $g_1(x,y) = y$, $g_2(x,y)=x$.
  \item $g_1(x,y) = y^2$, $g_2(x,y) = x^2$.
 \end{enumerate}
 \item
 Assume $g_1$ and $g_2$ are differentiable everywhere.
 Conjecture a necessary and sufficient condition
 on $g_1$ and $g_2$ for the 
 existence of an antiderivative $f$.
 Prove that it is necessary.
 For extra credit, prove that it is sufficient.
 \end{enumerate}
 \end{prob}

 \vspace{3mm}

\begin{prob}
 Let $g_n$ and $g$ be uniformly bounded on $[0,1]$ (i.e. there exists
 a number $M >0$ such that $|g(x)| \leq M$ and $|g_n(x)| \leq M$
 for all $n \in \mathbb{N}$ and all $x \in [0,1]$).
 Assume $g_n \to g$ pointwise on $[0,1]$ and uniformly
 on any set of the form $[0,\alpha]$, where $0 < \alpha < 1$.
 Finally, assume all $g_n$ and $g$ are integrable.
 Show that 
 \[\lim_{n \to \infty} \int^1_0 g_n = \int^1_0 g.
 \]
\end{prob}






\end{document}