
\documentclass{amsart}
%\usepackage{cmbright}
\usepackage{amsmath, amsfonts, amssymb, amscd}
%\usepackage{tableau}
\usepackage{color}
\usepackage{xcolor}
\usepackage{graphicx}
\usepackage{array}
\usepackage{mathtools}
\usepackage{multirow}
\usepackage{framed}
\usepackage{tikz}
\usetikzlibrary{matrix,arrows}
\usepackage[square,sort,comma,numbers]{natbib}
\usepackage{enumerate}




%\oddsidemargin.5cm
%\evensidemargin.5cm
\addtolength{\oddsidemargin}{-.525in}
\addtolength{\evensidemargin}{-.525in}
\addtolength{\textwidth}{1in}

\addtolength{\topmargin}{-.87in}
\addtolength{\textheight}{1.5in}




\newcommand\coolunder[2]{\mathrlap{\smash{\underbrace{\phantom{%
    \begin{matrix} #2 \end{matrix}}}_{\mbox{$#1$}}}}#2}

%\input{tableau}


\newcommand{\+}[1]{\ensuremath{\mathbf{#1}}}
\newcommand{\vect}[1]{\boldsymbol{#1}} % Uncomment for BOLD vectors.
%\newcommand{\vect}[1]{\vec{#1}} % Uncomment for ARROW vectors.
\newcommand{\C}{{\mathbb C}}
\newcommand{\Z}{{\mathbb Z}}
\renewcommand{\P}{{\mathbb P}}
\newcommand{\OG}{\operatorname{OG}}
\newcommand{\OF}{\operatorname{OF}}
\newcommand{\bull}{{\scriptscriptstyle \bullet}}
\newcommand{\la}{\lambda}
\newcommand{\euler}[1]{\chi_{_{#1}}}
\newcommand{\cO}{{\mathcal O}}
\newcommand{\cG}{{\mathcal G}}
\newcommand{\cQ}{{\mathcal Q}}
\newcommand{\R}{{\mathbb R}}
\newcommand{\wt}{\widetilde}
\newcommand{\diag}{\operatorname{diag}}
\newcommand{\comp}{\operatorname{comp}}
\newcommand{\comment}[1]{}
\newcommand{\type}{\mathfrak{t}}
\newcommand{\op}{\text{op}}
\newcommand{\row}{{\bf r}}
\newcommand{\col}{{\bf c}}
\newcommand{\sym}{\mathfrak{S}}
\newcommand{\codim}{\text{codim}}
\DeclarePairedDelimiter{\ceil}{\lceil}{\rceil}
\DeclarePairedDelimiter{\floor}{\lfloor}{\rfloor}
\renewcommand{\emptyset}{\varnothing}

\newtheorem{thm}{Theorem}
\newtheorem{lemma}[thm]{Lemma}
\newtheorem{prop}[thm]{Proposition}
\newtheorem{cor}[thm]{Corollary}

\theoremstyle{definition}
\newtheorem{definition}[thm]{Definition}
\newtheorem{example}[thm]{Example}
\newtheorem{conj}[thm]{Conjecture}
\newtheorem{obs}[thm]{Observation}
\newtheorem{fact}[thm]{Fact}
\newtheorem{remark}[thm]{Remark}
\newtheorem{prob}{Problem}
\newtheorem{chal}{Challenge}

\begin{document}
\title{Tutorial 2}
\date{Jan 23, 2018}
\author{Analysis II}

\maketitle

\begin{prob}
Let $x$ and $y$ be vectors in $\R^n$.
\begin{enumerate}[(a)]
\item For which values of $x$ and $y$ do we have $||x + y|| = ||x|| + ||y||$?
\item For which values of $x$ and $y$ do we have $||x + y|| ^2 = ||x||^2 + ||y||^2$?
\end{enumerate}
\end{prob}

\begin{prob}
For each of the following matrices,
\[
\begin{bmatrix}
1 & 1 & 1 & 1 \\ 1 & 1 & 1 & 1 \\ 0 & 1 & 2 & 3 \\  0 & 1 & 2 & 3  
\end{bmatrix},
\begin{bmatrix}
1 & 2 & 1 \\ 2 & 2 & 2 \\ 1 & 0 & 1
\end{bmatrix},
\begin{bmatrix}
1 & 2 & 3 \\ 2 & 3 & 4
\end{bmatrix},
\]
 \begin{enumerate}[(a)]
  \item Compute the reduced row echelon form of the matrix $A$, and determine the rank of $A$.
  \item Let $T_A: \R^m \to \R^n$ be the map whose matrix representation with respect
  to the standard bases is $A$.  Give an example of a vector in the kernel of $T_A$.
  \item Find bases for $\R^m$ and $\R^n$
  with respect to which the matrix representation of $T_A$ is 
\[
  \begin{bmatrix}
\+I_p & \+0 \\ \+0 & \+0
\end{bmatrix},
\]
for some $p$.
 \end{enumerate}
\end{prob}


\begin{prob}
Let $T$ be a linear transformation from $\R^m$ to $\R^n$, and consider the usual Euclidean inner product
on both spaces.
\begin{enumerate}[(a)]
\item Prove that $T$ is norm preserving (i.e. $||T(x)|| = ||x||$ for all $x \in V$)
if and only if $T$ is inner product preserving (i.e. $\langle T(x), T(y) \rangle = \langle x, y \rangle$ for all $x,y \in V$).
\item Prove that $T$ must be $1-1$ in this case. 
\end{enumerate}
\end{prob}


\begin{prob}
Recall that in Euclidean space, the angle $\theta(x,y)$ between nonzero vectors $x$ and $y$ is related to their inner product by
\[
\cos(\theta(x,y)) = \frac{\langle x, y \rangle}{||x|| \cdot ||y||}.
\]
We say a linear transformation $T$ is \emph{angle preserving} if it is $1-1$ and $\theta(Tx,Ty) = \theta(x,y)$
for all nonzero vectors $x$ and $y$.
\begin{enumerate}
 \item Prove that if $T$ is norm preserving, then it is angle preserving.
 \item Give an example of an angle preserving map that is not norm preserving.
 \item Describe all angle preserving maps.
\end{enumerate}
\end{prob}

\begin{prob}
If $T: \R^m \to \R^n$ is a linear transformation, show that there is a
number $M$ such that $|| T(x) || \leq M || x ||$ for all $x \in \R^m$.
\end{prob}


\begin{prob}
 Let $X$ denote $\R^2$ endowed with the Euclidean metric, and let $Y$ denote $\R^2$ endowed with the sup metric.
 Let $U$ be a subset of $\R^2$.  Prove that $U$ is an open set in $X$ if and only if $U$ is an open set in $Y$.
 (This shows that $X$ and $Y$ induce the same \emph{topology} on $\R^2$).
\end{prob}

\begin{prob}
 Let $\R^{\infty} = \bigcup^{\infty}_{n=1} \R^n$ where we consider the natural inclusions $\R^n \subset \R^{n+1}$.  
 (You can also think of this as those elements of $\R^\omega$ with finitely many nonzero entries.)
 Note that the dot product gives a well-defined inner product on  $\R^{\infty}$, and hence induces a metric.
 Define $e_i = (0, 0, \ldots, 0, 1, 0, 0, \ldots )$ where $1$ appears in the $i^\text{th}$ place.
 Prove that $X := \{e_i: i \in \mathbb{N}\}$ forms a basis for $\R^{\infty}$, and that $X$ is
 closed, bounded, and noncompact. 
\end{prob}





\vspace{5mm}

*All questions taken from \emph{Analysis on Manifolds} by James Munkres and \emph{Calculus on Manifolds} by Michael Spivak.

\end{document}