
\documentclass{amsart}
%\usepackage{cmbright}
\usepackage{amsmath, amsfonts, amssymb, amscd}
%\usepackage{tableau}
\usepackage{color}
\usepackage{xcolor}
\usepackage{graphicx}
\usepackage{array}
\usepackage{mathtools}
\usepackage{multirow}
\usepackage{framed}
\usepackage{tikz}
\usetikzlibrary{matrix,arrows}
\usepackage[square,sort,comma,numbers]{natbib}
\usepackage{enumerate}




%\oddsidemargin.5cm
%\evensidemargin.5cm
\addtolength{\oddsidemargin}{-.525in}
\addtolength{\evensidemargin}{-.525in}
\addtolength{\textwidth}{1in}

\addtolength{\topmargin}{-.87in}
\addtolength{\textheight}{1.5in}




\newcommand\coolunder[2]{\mathrlap{\smash{\underbrace{\phantom{%
    \begin{matrix} #2 \end{matrix}}}_{\mbox{$#1$}}}}#2}

%\input{tableau}


\newcommand{\+}[1]{\ensuremath{\mathbf{#1}}}
\newcommand{\vect}[1]{\boldsymbol{#1}} % Uncomment for BOLD vectors.
%\newcommand{\vect}[1]{\vec{#1}} % Uncomment for ARROW vectors.
\newcommand{\C}{{\mathbb C}}
\newcommand{\Z}{{\mathbb Z}}
\renewcommand{\P}{{\mathbb P}}
\newcommand{\OG}{\operatorname{OG}}
\newcommand{\OF}{\operatorname{OF}}
\newcommand{\bull}{{\scriptscriptstyle \bullet}}
\newcommand{\la}{\lambda}
\newcommand{\euler}[1]{\chi_{_{#1}}}
\newcommand{\cO}{{\mathcal O}}
\newcommand{\cG}{{\mathcal G}}
\newcommand{\cQ}{{\mathcal Q}}
\newcommand{\R}{{\mathbb R}}
\newcommand{\wt}{\widetilde}
\newcommand{\diag}{\operatorname{diag}}
\newcommand{\comp}{\operatorname{comp}}
\newcommand{\comment}[1]{}
\newcommand{\type}{\mathfrak{t}}
\newcommand{\op}{\text{op}}
\newcommand{\row}{{\bf r}}
\newcommand{\col}{{\bf c}}
\newcommand{\sym}{\mathfrak{S}}
\newcommand{\codim}{\text{codim}}
\DeclarePairedDelimiter{\ceil}{\lceil}{\rceil}
\DeclarePairedDelimiter{\floor}{\lfloor}{\rfloor}
\renewcommand{\emptyset}{\varnothing}

\newtheorem{thm}{Theorem}
\newtheorem{lemma}[thm]{Lemma}
\newtheorem{prop}[thm]{Proposition}
\newtheorem{cor}[thm]{Corollary}

\theoremstyle{definition}
\newtheorem{definition}[thm]{Definition}
\newtheorem{example}[thm]{Example}
\newtheorem{conj}[thm]{Conjecture}
\newtheorem{obs}[thm]{Observation}
\newtheorem{fact}[thm]{Fact}
\newtheorem{remark}[thm]{Remark}
\newtheorem{prob}{Problem}
\newtheorem{chal}{Challenge}

\begin{document}
\title{Problem Set 8}
\date{March 29, 2018}
\author{Analysis II}

\maketitle

\begin{prob}
 \begin{enumerate}[(a)]
 \item Given a symmetric $m \times m$ matrix $A$, let $Q_A: \R^m \to \R$ be defined by
 $x \mapsto x^t A x$.  Show that $\nabla Q_A (x) = 2 A x$, for any $x \in \R^m$.
  \item  Let $\phi: \R^m \to \R$ be a function of class $C^2$.
 Let $x$ be a critical point of $\phi$ (i.e. suppose $\nabla \phi (x) = 0$).
 Let \[H(x) = \left(\frac{\partial^2\phi}{\partial{x_i}\partial{x_j}}(x)\right)\] be the Hessian of $\phi$ at $x$.
 In class we defined $\frac{\partial^2\phi}{\partial v^2}(x)$ to be $\frac{d^2}{dt^2}[\phi(x+tv)]_{t=0}$.
 Prove that \[\frac{\partial^2\phi}{\partial v^2}(x) = v^t H(x) v\]
 for any $v \in \R^m$.
 (Hint: it will be useful to prove that $\frac{\partial^2\phi}{\partial v^2}(x) = \frac{\partial}{\partial v}(\frac{\partial}{\partial v}\phi)(x)$.)
 \end{enumerate}
\end{prob}

\begin{prob}
 Prove the \emph{second order Taylor's theorem} discussed in class.  Namely, let $B = U(a,\epsilon) \subset \R^m$
 be an open ball centered at a point $a$ in $\R^m$, and suppose
 $f: B \to \R$ is of class $C^2$.
 Show that for any $x \in B$, we have
 \[
 f(x) = f(a) + (x-a)^t \nabla f (a) + \frac{1}{2}(x-a)^t H(b) (x-a),
 \]
 for some $b$ on the line segment joining $a$ to $x$,
 where $H(b)$ denotes the Hessian of $f$ at $b$.
\end{prob}

\begin{prob}
 In class we defined the \emph{operator norm} on the set of $m \times m$ matrices by
 \[||A|| := \sup\{||Ax|| / ||x||: x \in \R^m\} = \sup\{||Ax||: x \in S^{m-1}\},\]
 where $||v||$ denotes the Euclidean norm on $v \in \R^m$.
 Prove that:
 \begin{enumerate}[(a)]
  \item The operator norm is indeed a norm.
  \item The operator norm is a \emph{continuous} function from 
  $\R^{m^2}$ to $R$ (here, we identify the set of $m \times m$ matrices with the Euclidean space $\R^{m^2}$).
 \end{enumerate}
\end{prob}

\begin{prob}
 Suppose $A$ is a symmetric $m \times m$ matrix with eigenvalues $\lambda_1, \ldots, \lambda_m$.
 Moreover, suppose the eigenvalues are indexed so that $|\lambda_1| \geq |\lambda_2| \geq \ldots \geq |\lambda_m|$.
 Show that $||A|| = |\lambda_1|$.
\end{prob}


\begin{prob}
 Prove the second derivative test.  Namely, let $U \subset \R^m$ be an open set.
 Suppose $\phi: U \to \R$ is of class $C^2$, and that $\nabla \phi (x) = 0$
 for some $x$ in $U$.
 Show that if $\frac{\partial^2 \phi}{\partial v^2}(x) > 0$ for any $v \in \R^m$
 then $\phi$ attains a local minimum at $x$.
 (By \emph{local minimum}, we mean that $\phi$ takes strictly greater values in some deleted neighborhood of $x$.)
\end{prob}

\end{document}