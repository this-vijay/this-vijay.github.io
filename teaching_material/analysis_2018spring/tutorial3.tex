
\documentclass{amsart}
%\usepackage{cmbright}
\usepackage{amsmath, amsfonts, amssymb, amscd}
%\usepackage{tableau}
\usepackage{color}
\usepackage{xcolor}
\usepackage{graphicx}
\usepackage{array}
\usepackage{mathtools}
\usepackage{multirow}
\usepackage{framed}
\usepackage{tikz}
\usetikzlibrary{matrix,arrows}
\usepackage[square,sort,comma,numbers]{natbib}
\usepackage{enumerate}




%\oddsidemargin.5cm
%\evensidemargin.5cm
\addtolength{\oddsidemargin}{-.525in}
\addtolength{\evensidemargin}{-.525in}
\addtolength{\textwidth}{1in}

\addtolength{\topmargin}{-.87in}
\addtolength{\textheight}{1.5in}




\newcommand\coolunder[2]{\mathrlap{\smash{\underbrace{\phantom{%
    \begin{matrix} #2 \end{matrix}}}_{\mbox{$#1$}}}}#2}

%\input{tableau}


\newcommand{\+}[1]{\ensuremath{\mathbf{#1}}}
\newcommand{\vect}[1]{\boldsymbol{#1}} % Uncomment for BOLD vectors.
%\newcommand{\vect}[1]{\vec{#1}} % Uncomment for ARROW vectors.
\newcommand{\C}{{\mathbb C}}
\newcommand{\Z}{{\mathbb Z}}
\renewcommand{\P}{{\mathbb P}}
\newcommand{\OG}{\operatorname{OG}}
\newcommand{\OF}{\operatorname{OF}}
\newcommand{\bull}{{\scriptscriptstyle \bullet}}
\newcommand{\la}{\lambda}
\newcommand{\euler}[1]{\chi_{_{#1}}}
\newcommand{\cO}{{\mathcal O}}
\newcommand{\cG}{{\mathcal G}}
\newcommand{\cQ}{{\mathcal Q}}
\newcommand{\R}{{\mathbb R}}
\newcommand{\wt}{\widetilde}
\newcommand{\diag}{\operatorname{diag}}
\newcommand{\comp}{\operatorname{comp}}
\newcommand{\comment}[1]{}
\newcommand{\type}{\mathfrak{t}}
\newcommand{\op}{\text{op}}
\newcommand{\row}{{\bf r}}
\newcommand{\col}{{\bf c}}
\newcommand{\sym}{\mathfrak{S}}
\newcommand{\codim}{\text{codim}}
\DeclarePairedDelimiter{\ceil}{\lceil}{\rceil}
\DeclarePairedDelimiter{\floor}{\lfloor}{\rfloor}
\renewcommand{\emptyset}{\varnothing}

\newtheorem{thm}{Theorem}
\newtheorem{lemma}[thm]{Lemma}
\newtheorem{prop}[thm]{Proposition}
\newtheorem{cor}[thm]{Corollary}

\theoremstyle{definition}
\newtheorem{definition}[thm]{Definition}
\newtheorem{example}[thm]{Example}
\newtheorem{conj}[thm]{Conjecture}
\newtheorem{obs}[thm]{Observation}
\newtheorem{fact}[thm]{Fact}
\newtheorem{remark}[thm]{Remark}
\newtheorem{prob}{Problem}
\newtheorem{chal}{Challenge}

\begin{document}
\title{Tutorial 3}
\date{Feb 9, 2018}
\author{Analysis II}

\maketitle

\begin{prob}
Let $f:\R^n \supset A \to \R^m$, and let $a$ be a limit point of $A$.
Show that $\lim_{x \to a} f(x) = \+0$ if and only if $\lim_{x \to a} || f(x) || = 0$.
In particular, this shows that a function $f$ is differentiable at $a \in$ Int$(A)$
if and only if there exists an $n \times m$ matrix $B$ such that
\[
\lim_{h \to a} \frac{||f(a+h) - f(a) - B \cdot h ||}{|| h ||} = 0.
\]
\end{prob}

\begin{prob}
Given a function $f: \R^n \to \R$, we
can interpret $Df(x_1,\ldots,x_n)$ as a vector
field in $\R^n$.  (This is known as
the \emph{gradient} vector field of $f$, and it
points in the direction of steepest increase
of the function $f$.)
Sketch the gradient vector fields of the following
functions:
\begin{enumerate}[(a)]
 \item $f(x,y) = -xy$
 \item $f(x,y) = x^2 + y^2$
\end{enumerate}
\end{prob}


\begin{prob}
Let $A \subset \R^m$.
Show that if the maximum (or minimum) of a function $f: A \to \R$ occurs at a point $a \in $ Int$A$ and
if $D_i f(a)$ exists, then $D_if(a) = 0$.
\end{prob}

\begin{prob}
 Sketch the graphs of the following surfaces.  Check if the functions attain any maximum or
 minimum values.
 \begin{enumerate}[(a)]
  \item $f(x,y) = y^2 + x^2$.
  \item $f(x,y = y^2 - x^2$.
 \end{enumerate}
\end{prob}

\begin{prob}
Let $f: \R \to \R^3$ be defined by $f(t) = (t, t^2, t^3)$.
\begin{enumerate}[(a)]
 \item Sketch the image of $f$ in $\R^3$.
 \item Determine $Df(t)$.  We can interpret this $3 \times 1$ matrix
 as giving the velocity vector of a parametrized curve at time $t$.
 Plot several velocity vectors in your sketch from part $(a)$.
 \item Let $\pi: \R^3 \to \R^2$ denote the projection onto the last two coordinates.
 Sketch the image of $\pi \circ f$ in $\R^2$, and calculate the derivative of this function.
 \item Can you interpret your sketch in part $(c)$ as the graph of a function from $\R \to \R$ 
 (perhaps after rotating it)?  Calculate the derivative of this function as well.
\end{enumerate}
\end{prob}





\end{document}