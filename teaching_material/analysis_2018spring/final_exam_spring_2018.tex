
\documentclass{amsart}
%\usepackage{cmbright}
\usepackage{amsmath, amsfonts, amssymb, amscd}
%\usepackage{tableau}
\usepackage{color}
\usepackage{xcolor}
\usepackage{graphicx}
\usepackage{array}
\usepackage{mathtools}
\usepackage{multirow}
\usepackage{framed}
\usepackage{tikz}
\usetikzlibrary{matrix,arrows}
\usepackage[square,sort,comma,numbers]{natbib}
\usepackage{enumerate}
\usepackage{comment}



%\oddsidemargin.5cm
%\evensidemargin.5cm
\addtolength{\oddsidemargin}{-.525in}
\addtolength{\evensidemargin}{-.525in}
\addtolength{\textwidth}{1in}

\addtolength{\topmargin}{-.87in}
\addtolength{\textheight}{1.5in}




\newcommand\coolunder[2]{\mathrlap{\smash{\underbrace{\phantom{%
    \begin{matrix} #2 \end{matrix}}}_{\mbox{$#1$}}}}#2}

%\input{tableau}


\newcommand{\+}[1]{\ensuremath{\mathbf{#1}}}
\newcommand{\vect}[1]{\boldsymbol{#1}} % Uncomment for BOLD vectors.
%\newcommand{\vect}[1]{\vec{#1}} % Uncomment for ARROW vectors.
\newcommand{\C}{{\mathbb C}}
\newcommand{\Z}{{\mathbb Z}}
\renewcommand{\P}{{\mathbb P}}
\newcommand{\OG}{\operatorname{OG}}
\newcommand{\OF}{\operatorname{OF}}
\newcommand{\bull}{{\scriptscriptstyle \bullet}}
\newcommand{\la}{\lambda}
\newcommand{\euler}[1]{\chi_{_{#1}}}
\newcommand{\cO}{{\mathcal O}}
\newcommand{\cG}{{\mathcal G}}
\newcommand{\cQ}{{\mathcal Q}}
\newcommand{\R}{{\mathbb R}}
\newcommand{\wt}{\widetilde}
\newcommand{\diag}{\operatorname{diag}}
\newcommand{\comp}{\operatorname{comp}}
\newcommand{\type}{\mathfrak{t}}
\newcommand{\op}{\text{op}}
\newcommand{\row}{{\bf r}}
\newcommand{\col}{{\bf c}}
\newcommand{\sym}{\mathfrak{S}}
\newcommand{\codim}{\text{codim}}
\DeclarePairedDelimiter{\ceil}{\lceil}{\rceil}
\DeclarePairedDelimiter{\floor}{\lfloor}{\rfloor}
\renewcommand{\emptyset}{\varnothing}

\newtheorem{thm}{Theorem}
\newtheorem{lemma}[thm]{Lemma}
\newtheorem{prop}[thm]{Proposition}
\newtheorem{cor}[thm]{Corollary}

\theoremstyle{definition}
\newtheorem{definition}[thm]{Definition}
\newtheorem{example}[thm]{Example}
\newtheorem{conj}[thm]{Conjecture}
\newtheorem{obs}[thm]{Observation}
\newtheorem{fact}[thm]{Fact}
\newtheorem{remark}[thm]{Remark}
\newtheorem{prob}{Problem}
\newtheorem{chal}{Challenge}

\begin{document}
\title{Final Exam}
\date{April 26, 2018}
\author{Intro to Real Analysis II}

\maketitle


The exam consists of five questions, each worth $20$ points.
You may take up to three hours to complete the exam.

\vspace{5mm}

\begin{prob}
 Suppose $A$ and $B$ are both open subsets of $\R^n$,
 and that $f: A \to B$ is a one-to-one, onto
 function of class  $C^r$.
 Suppose $g: B \to A$ is a differentiable inverse to $f$.
 Prove directly that $g$ is of class $C^r$ (in particular do not use the Inverse Function Theorem).
\end{prob}

\vspace{5mm}


\begin{prob}
 Suppose $\alpha$ is a real root of multiplicity $1$ 
 of the real polynomial
 $a_0 + a_1x_1 + \ldots + a_n x^n$.
 Show that for any $\epsilon > 0$
 there exists a $\delta > 0$ such that
 the polynomial  $a'_0 + a'_1x_1 + \ldots + a'_n x^n$
 has a root in $(\alpha - \epsilon, \alpha + \epsilon)$,
 provided $|a'_j - a_j| < \delta$ for all $j$.
\end{prob}

\vspace{5mm}


\begin{prob}
Let $p_1, \ldots, p_m$ be $m$ points in $\R^n$.
Show that $\sum^{m}_{i=1} ||x - p_i||^2$
achieves its absolute minimum at $x = \frac{1}{m}\sum^m_{i=1} p_i$.
\end{prob}

\vspace{5mm}


\begin{prob}
Fix an element $a \in \R^n$.
 Use the method of Lagrange multipliers to
find the extreme values of $f(x,y) = x^2 - y^2$
on $S := \{x^2 + y^2 = 1\}$.
\end{prob}

\vspace{5mm}


\begin{prob}
Consider a complex function $f: \C \to \C$.
 Show that if \[f'(a) := \lim_{\Delta z \to 0} \frac{f(a + \Delta z) - f(a)}{\Delta z}\] exists and equals $\alpha + i \beta$, then
 \[
 Df(a) = 
  \begin{bmatrix}
\alpha & -\beta \\ \beta & \alpha \\
\end{bmatrix}.
 \]
\end{prob}

% \begin{prob}
% Let $\phi: \R^m \to \R$ be a function of class $C^2$.
%  Let $x$ be a critical point of $\phi$.
%  
%  Let \[H(x) = \left(\frac{\partial^2\phi}{\partial{x_i}\partial{x_j}}(x)\right)\] be the Hessian of $\phi$ at $x$,
%  where $\frac{\partial^2\phi}{\partial v^2}(x)$ is defined to be $\frac{d^2}{dt^2}[\phi(x+tv)]_{t=0}$.
%  
%  Prove that \[\frac{\partial^2\phi}{\partial v^2}(x) = v^t H(x) v\]
%  for any $v \in \R^m$.
% \end{prob}


 \end{document}