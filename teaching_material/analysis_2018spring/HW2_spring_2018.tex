
\documentclass{amsart}
%\usepackage{cmbright}
\usepackage{amsmath, amsfonts, amssymb, amscd}
%\usepackage{tableau}
\usepackage{color}
\usepackage{xcolor}
\usepackage{graphicx}
\usepackage{array}
\usepackage{mathtools}
\usepackage{multirow}
\usepackage{framed}
\usepackage{tikz}
\usetikzlibrary{matrix,arrows}
\usepackage[square,sort,comma,numbers]{natbib}
\usepackage{enumerate}




%\oddsidemargin.5cm
%\evensidemargin.5cm
\addtolength{\oddsidemargin}{-.525in}
\addtolength{\evensidemargin}{-.525in}
\addtolength{\textwidth}{1in}

\addtolength{\topmargin}{-.87in}
\addtolength{\textheight}{1.5in}




\newcommand\coolunder[2]{\mathrlap{\smash{\underbrace{\phantom{%
    \begin{matrix} #2 \end{matrix}}}_{\mbox{$#1$}}}}#2}

%\input{tableau}


\newcommand{\+}[1]{\ensuremath{\mathbf{#1}}}
\newcommand{\vect}[1]{\boldsymbol{#1}} % Uncomment for BOLD vectors.
%\newcommand{\vect}[1]{\vec{#1}} % Uncomment for ARROW vectors.
\newcommand{\C}{{\mathbb C}}
\newcommand{\Z}{{\mathbb Z}}
\renewcommand{\P}{{\mathbb P}}
\newcommand{\OG}{\operatorname{OG}}
\newcommand{\OF}{\operatorname{OF}}
\newcommand{\bull}{{\scriptscriptstyle \bullet}}
\newcommand{\la}{\lambda}
\newcommand{\euler}[1]{\chi_{_{#1}}}
\newcommand{\cO}{{\mathcal O}}
\newcommand{\cG}{{\mathcal G}}
\newcommand{\cQ}{{\mathcal Q}}
\newcommand{\R}{{\mathbb R}}
\newcommand{\wt}{\widetilde}
\newcommand{\diag}{\operatorname{diag}}
\newcommand{\comp}{\operatorname{comp}}
\newcommand{\comment}[1]{}
\newcommand{\type}{\mathfrak{t}}
\newcommand{\op}{\text{op}}
\newcommand{\row}{{\bf r}}
\newcommand{\col}{{\bf c}}
\newcommand{\sym}{\mathfrak{S}}
\newcommand{\codim}{\text{codim}}
\DeclarePairedDelimiter{\ceil}{\lceil}{\rceil}
\DeclarePairedDelimiter{\floor}{\lfloor}{\rfloor}
\renewcommand{\emptyset}{\varnothing}

\newtheorem{thm}{Theorem}
\newtheorem{lemma}[thm]{Lemma}
\newtheorem{prop}[thm]{Proposition}
\newtheorem{cor}[thm]{Corollary}

\theoremstyle{definition}
\newtheorem{definition}[thm]{Definition}
\newtheorem{example}[thm]{Example}
\newtheorem{conj}[thm]{Conjecture}
\newtheorem{obs}[thm]{Observation}
\newtheorem{fact}[thm]{Fact}
\newtheorem{remark}[thm]{Remark}
\newtheorem{prob}{Problem}
\newtheorem{chal}{Challenge}

\begin{document}
\title{Problem Set 2}
\date{Jan 9, 2018}
\author{Analysis II}

\maketitle

\begin{prob}
Let $f$ and $g$ be integrable functions on $[a,b]$.
\begin{enumerate}[(a)]
 \item Show that if $P$ is any partition of $[a,b]$, then
 \[
 U(f+g,P) \leq U(f,P) + U(g,P).
 \]
 Provide a specific example where the inequality is strict.
 What does the corresponding inequality for lower sums
 look like?
 \item Prove that $\int_a^b (f + g) = \int_a^b f + \int_a^b g$.
\end{enumerate}
\end{prob}

\vspace{3mm}

\begin{prob}
Show that products of integrable functions are
also integrable as follows:
\begin{enumerate}[(a)]
 \item If $f$ satisfies $|f(x)| \leq M$ on $[a,b]$, show
 \[
 |(f(x))^2 - (f(y))^2| \leq 2M|f(x) - f(y)|.
 \]
\item Prove that if $f$ is integrable on $[a,b]$, then
so is $f^2$.
\item Show that for any integrable functions $f$ and $g$, the
product $fg$ is also integrable. (Hint: consider the square
of $(f+g)$).
 \end{enumerate}
\end{prob}

\vspace{3mm}


\begin{prob}
 For each $n \in \mathbb{N}$, let
 \[
 h_n(x)= \begin{cases}
          1/2^n &\text{ if } 1/2^n < x \leq 1 \\
          0 &\text{ if } 0 \leq x \leq 1/2^n,
         \end{cases}
 \]
 and set $H(x) = \sum^\infty_{n=1} h_n(x)$.  Show that
 $H$ is integrable and compute $\int_0^1 H$.
\end{prob}

\vspace{3mm}


\begin{prob}[Integration by parts]
 \begin{enumerate}[(a)]
  \item Assume $h(x)$ and $k(x)$ have continuous derivatives
  on $[a,b]$ and derive the familiar integration by parts formula:
  \[
  \int_a^b h(t) k'(t) dt = h(b) k(b) - h(a) k(a) - \int_a^b h'(t) k(t) dt.
  \]
  \item How can Problem 2 above be used to weaken the
  hypotheses in part (a).
 \end{enumerate}
\end{prob}

\vspace{3mm}


\begin{prob}
 Given a function $f$ on $[a,b]$, define the \emph{total variation}
 of $f$ to be
 \[
 Vf = \sup \left\{ \sum^n_{k=1} |f(x_k) - f(x_{k-1})|\right\},
 \]
 where the supremum is taken over all partitions $P$
 of $[a,b]$.
 \begin{enumerate}[(a)]
  \item If $f'$ exists and is continuous,
  use the Fundamental Theorem of Calculus to show that $Vf \leq \int_a^b|f'|$.
  \item Use the Mean Value Theorem to establish the
  reverse inequality and conclude that $Vf = \int_a^b|f'|$.
 \end{enumerate}
\end{prob}



\vspace{5mm}

*All questions taken from \emph{Understanding Analysis: 2nd Edition} by Stephen Abbott.


\end{document}