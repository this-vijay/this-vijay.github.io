
\documentclass{amsart}
%\usepackage{cmbright}
\usepackage{amsmath, amsfonts, amssymb, amscd}
%\usepackage{tableau}
\usepackage{color}
\usepackage{xcolor}
\usepackage{graphicx}
\usepackage{array}
\usepackage{mathtools}
\usepackage{multirow}
\usepackage{framed}
\usepackage{tikz}
\usetikzlibrary{matrix,arrows}
\usepackage[square,sort,comma,numbers]{natbib}
\usepackage{enumerate}




%\oddsidemargin.5cm
%\evensidemargin.5cm
\addtolength{\oddsidemargin}{-.525in}
\addtolength{\evensidemargin}{-.525in}
\addtolength{\textwidth}{1in}

\addtolength{\topmargin}{-.87in}
\addtolength{\textheight}{1.5in}




\newcommand\coolunder[2]{\mathrlap{\smash{\underbrace{\phantom{%
    \begin{matrix} #2 \end{matrix}}}_{\mbox{$#1$}}}}#2}

%\input{tableau}


\newcommand{\+}[1]{\ensuremath{\mathbf{#1}}}
\newcommand{\vect}[1]{\boldsymbol{#1}} % Uncomment for BOLD vectors.
%\newcommand{\vect}[1]{\vec{#1}} % Uncomment for ARROW vectors.
\newcommand{\C}{{\mathbb C}}
\newcommand{\Z}{{\mathbb Z}}
\renewcommand{\P}{{\mathbb P}}
\newcommand{\OG}{\operatorname{OG}}
\newcommand{\OF}{\operatorname{OF}}
\newcommand{\bull}{{\scriptscriptstyle \bullet}}
\newcommand{\la}{\lambda}
\newcommand{\euler}[1]{\chi_{_{#1}}}
\newcommand{\cO}{{\mathcal O}}
\newcommand{\cG}{{\mathcal G}}
\newcommand{\cQ}{{\mathcal Q}}
\newcommand{\R}{{\mathbb R}}
\newcommand{\wt}{\widetilde}
\newcommand{\diag}{\operatorname{diag}}
\newcommand{\comp}{\operatorname{comp}}
\newcommand{\comment}[1]{}
\newcommand{\type}{\mathfrak{t}}
\newcommand{\op}{\text{op}}
\newcommand{\row}{{\bf r}}
\newcommand{\col}{{\bf c}}
\newcommand{\sym}{\mathfrak{S}}
\newcommand{\codim}{\text{codim}}
\DeclarePairedDelimiter{\ceil}{\lceil}{\rceil}
\DeclarePairedDelimiter{\floor}{\lfloor}{\rfloor}
\renewcommand{\emptyset}{\varnothing}
\renewcommand{\tilde}{\widetilde}


\newtheorem{thm}{Theorem}
\newtheorem{lemma}[thm]{Lemma}
\newtheorem{prop}[thm]{Proposition}
\newtheorem{cor}[thm]{Corollary}

\theoremstyle{definition}
\newtheorem{definition}[thm]{Definition}
\newtheorem{example}[thm]{Example}
\newtheorem{conj}[thm]{Conjecture}
\newtheorem{obs}[thm]{Observation}
\newtheorem{fact}[thm]{Fact}
\newtheorem{remark}[thm]{Remark}
\newtheorem{prob}{Problem}
\newtheorem{chal}{Challenge}

\begin{document}
\title{Problem Set 6}
\author{Topics in Manifolds, Spring 2016}

\maketitle



\begin{prob}
Consider the following five tessellations: the square and triangle tessellations of $\mathbb{E}^2$; the icosahedral, octahedral, and tetrahedral tessellations of $\mathbb{S}^2$.
For each of these tessellations, let $\Gamma^+$ be its group of orientation preserving symmetries:
\begin{enumerate}[(a)]
 \item Draw a fundamental region for $\Gamma^+$.
 \item Draw the Caley graph of $\Gamma^+$.
 \item Give a presentation for the group $\Gamma^+$.
\end{enumerate}
(See problems $7.1.1 - 7.1.4$ in Stillwell).
\end{prob}


\begin{prob}
Find a presentation of the orientation preserving subgroup of the group 
generated by reflections in the sides of a $(p,q,r)$ triangle.
(See problem $7.3.1$ in Stillwell).
\end{prob}


\begin{prob}
\begin{enumerate}[(a)]
 \item Show that reflections in the sides of the $(2,3,\infty)$ triangle
with vertices $i,\omega := \frac{1}{2} + \frac{\sqrt{3}}{2}i,\infty$
induce the side-pairing transformations $z \mapsto 1 + z$
and $z \mapsto -\frac{1}{z}$ in its double.  
\item Verify that the group $\Gamma$ generated by these transformations
has the free group $F_2$ as a subgroup.
What is the index of $F_2$ in $\Gamma$?
\item Show that the group $\Gamma$ has a presentation
\[
\langle g,h \vert g^2 = h^3 = 1 \rangle
\]
\end{enumerate}
(See problem $7.3.6$ and $7.3.7$ in Stillwell).
\end{prob}


\begin{prob}
 Show that $336$ is the correct number of $(2,3,7)$ triangles
 to tessellate a hyperbolic surface of genus $3$. 
 (See problem $7.3.4$ in Stillwell).
\end{prob}

\begin{prob}
 Show that a hyperbolic surface of genus $2$ can be tessellated symmetrically
 by $96$ $(2,3,8)$ triangles.
 (See problem $7.3.5$ in Stillwell).
\end{prob}




\end{document}