
\documentclass{amsart}
%\usepackage{cmbright}
\usepackage{amsmath, amsfonts, amssymb, amscd}
%\usepackage{tableau}
\usepackage{color}
\usepackage{xcolor}
\usepackage{graphicx}
\usepackage{array}
\usepackage{mathtools}
\usepackage{multirow}
\usepackage{framed}
\usepackage{tikz}
\usetikzlibrary{matrix,arrows}
\usepackage[square,sort,comma,numbers]{natbib}
\usepackage{enumerate}




%\oddsidemargin.5cm
%\evensidemargin.5cm
\addtolength{\oddsidemargin}{-.525in}
\addtolength{\evensidemargin}{-.525in}
\addtolength{\textwidth}{1in}

\addtolength{\topmargin}{-.87in}
\addtolength{\textheight}{1.5in}




\newcommand\coolunder[2]{\mathrlap{\smash{\underbrace{\phantom{%
    \begin{matrix} #2 \end{matrix}}}_{\mbox{$#1$}}}}#2}

%\input{tableau}


\newcommand{\+}[1]{\ensuremath{\mathbf{#1}}}
\newcommand{\vect}[1]{\boldsymbol{#1}} % Uncomment for BOLD vectors.
%\newcommand{\vect}[1]{\vec{#1}} % Uncomment for ARROW vectors.
\newcommand{\C}{{\mathbb C}}
\newcommand{\Z}{{\mathbb Z}}
\renewcommand{\P}{{\mathbb P}}
\newcommand{\OG}{\operatorname{OG}}
\newcommand{\OF}{\operatorname{OF}}
\newcommand{\bull}{{\scriptscriptstyle \bullet}}
\newcommand{\la}{\lambda}
\newcommand{\euler}[1]{\chi_{_{#1}}}
\newcommand{\cO}{{\mathcal O}}
\newcommand{\cG}{{\mathcal G}}
\newcommand{\cQ}{{\mathcal Q}}
\newcommand{\R}{{\mathbb R}}
\newcommand{\wt}{\widetilde}
\newcommand{\diag}{\operatorname{diag}}
\newcommand{\comp}{\operatorname{comp}}
\newcommand{\comment}[1]{}
\newcommand{\type}{\mathfrak{t}}
\newcommand{\op}{\text{op}}
\newcommand{\row}{{\bf r}}
\newcommand{\col}{{\bf c}}
\newcommand{\sym}{\mathfrak{S}}
\newcommand{\codim}{\text{codim}}
\DeclarePairedDelimiter{\ceil}{\lceil}{\rceil}
\DeclarePairedDelimiter{\floor}{\lfloor}{\rfloor}
\renewcommand{\emptyset}{\varnothing}
\renewcommand{\tilde}{\widetilde}


\newtheorem{thm}{Theorem}
\newtheorem{lemma}[thm]{Lemma}
\newtheorem{prop}[thm]{Proposition}
\newtheorem{cor}[thm]{Corollary}

\theoremstyle{definition}
\newtheorem{definition}[thm]{Definition}
\newtheorem{example}[thm]{Example}
\newtheorem{conj}[thm]{Conjecture}
\newtheorem{obs}[thm]{Observation}
\newtheorem{fact}[thm]{Fact}
\newtheorem{remark}[thm]{Remark}
\newtheorem{prob}{Problem}
\newtheorem{chal}{Challenge}

\begin{document}
\title{Problem Set 4}
\author{Topics in Manifolds, Spring 2016}

\maketitle


\begin{prob}
Give a euclidean geometric construction showing the existence and uniqueness of an $\mathbb{H}^2$-line
through any $z_1, z_2 \in \mathbb{H}^2$ (Stillwell $4.2.3$).
\end{prob}

\begin{prob}
Define an $\mathbb{H}^2$-circle to be the set of points $\mathbb{H}^2$ equidistant to a given point.  Show that
all $\mathbb{H}^2$-circles are in fact eucliden circles in $\mathbb{H}^2$ (for hints see Stillwell $4.2.4$).
\end{prob}

\begin{prob}
 Show that the $\mathbb{D}^2$-circumference of a $\mathbb{D}^2$-circle of $\mathbb{D}^2$-radius $\rho$ is
 $2\pi \sinh \rho$ (Stillwell $4.2.5$).
\end{prob}

\begin{prob}
  Show that the $\mathbb{S}^2$-circumference of a $\mathbb{S}^2$-circle of $\mathbb{S}^2$-radius $\rho$ is
 $2\pi \sin \rho$ (Stillwell $4.2.6$).
\end{prob}

\begin{prob}
 Show that the $\mathbb{D}^2$-distance between  $w_1,w_2 \in \mathbb{D}^2$ is
 \[
 2\tanh^{-1} \left| \frac{w_2 - w_1}{1 - \bar{w_1}w_2} \right|
 \]
 (see Stillwell $4.4.1$).
 \end{prob}
 
 \begin{prob}
  Show that any $\mathbb{D}^2$-isometry is of the form
  \[
 \text{ (rotation about $0$)(limit rotation about $1$)(rotation about $0$)}  
 \]
  (see Stillwell $4.4.4$ and $4.4.5$ for hints).
 \end{prob}












\end{document}