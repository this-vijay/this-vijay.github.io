
\documentclass{amsart}
%\usepackage{cmbright}
\usepackage{amsmath, amsfonts, amssymb, amscd}
%\usepackage{tableau}
\usepackage{color}
\usepackage{xcolor}
\usepackage{graphicx}
\usepackage{array}
\usepackage{mathtools}
\usepackage{multirow}
\usepackage{framed}
\usepackage{tikz}
\usetikzlibrary{matrix,arrows}
\usepackage[square,sort,comma,numbers]{natbib}
\usepackage{enumerate}




%\oddsidemargin.5cm
%\evensidemargin.5cm
\addtolength{\oddsidemargin}{-.525in}
\addtolength{\evensidemargin}{-.525in}
\addtolength{\textwidth}{1in}

\addtolength{\topmargin}{-.87in}
\addtolength{\textheight}{1.5in}




\newcommand\coolunder[2]{\mathrlap{\smash{\underbrace{\phantom{%
    \begin{matrix} #2 \end{matrix}}}_{\mbox{$#1$}}}}#2}

%\input{tableau}


\newcommand{\+}[1]{\ensuremath{\mathbf{#1}}}
\newcommand{\vect}[1]{\boldsymbol{#1}} % Uncomment for BOLD vectors.
%\newcommand{\vect}[1]{\vec{#1}} % Uncomment for ARROW vectors.
\newcommand{\C}{{\mathbb C}}
\newcommand{\Z}{{\mathbb Z}}
\renewcommand{\P}{{\mathbb P}}
\newcommand{\OG}{\operatorname{OG}}
\newcommand{\OF}{\operatorname{OF}}
\newcommand{\bull}{{\scriptscriptstyle \bullet}}
\newcommand{\la}{\lambda}
\newcommand{\euler}[1]{\chi_{_{#1}}}
\newcommand{\cO}{{\mathcal O}}
\newcommand{\cG}{{\mathcal G}}
\newcommand{\cQ}{{\mathcal Q}}
\newcommand{\R}{{\mathbb R}}
\newcommand{\wt}{\widetilde}
\newcommand{\diag}{\operatorname{diag}}
\newcommand{\comp}{\operatorname{comp}}
\newcommand{\comment}[1]{}
\newcommand{\type}{\mathfrak{t}}
\newcommand{\op}{\text{op}}
\newcommand{\row}{{\bf r}}
\newcommand{\col}{{\bf c}}
\newcommand{\sym}{\mathfrak{S}}
\newcommand{\codim}{\text{codim}}
\DeclarePairedDelimiter{\ceil}{\lceil}{\rceil}
\DeclarePairedDelimiter{\floor}{\lfloor}{\rfloor}
\renewcommand{\emptyset}{\varnothing}
\renewcommand{\tilde}{\widetilde}


\newtheorem{thm}{Theorem}
\newtheorem{lemma}[thm]{Lemma}
\newtheorem{prop}[thm]{Proposition}
\newtheorem{cor}[thm]{Corollary}

\theoremstyle{definition}
\newtheorem{definition}[thm]{Definition}
\newtheorem{example}[thm]{Example}
\newtheorem{conj}[thm]{Conjecture}
\newtheorem{obs}[thm]{Observation}
\newtheorem{fact}[thm]{Fact}
\newtheorem{remark}[thm]{Remark}
\newtheorem{prob}{Problem}
\newtheorem{chal}{Challenge}

\begin{section}{Luna-Vust for Horospherical Varieties}
Suppose $G/H$ is horospherical; i.e. that $H$ contains a maximal unipotent
subgroup $U_B$ for some Borel $B$.  We have the following well-known results:
\begin{enumerate}
 \item $P := N_G(H)$ is a parabolic subgroup containing $B$. 
 \item There exists a subgroup $M \subset \mathcal{X}(P)$ such that $H = \cap_{\chi \in M} \text{ker}(\chi)$.
\end{enumerate}
Let $N$ be the dual lattice to $N$, and let $N_\R = N \otimes_\Z \R$.

Let $X$ be a horospherical variety; i.e let $X$ be a normal $G$-variety with a $G$-equivariant embedding $G/H \hookrightarrow X$ such that the image of $G/H$ sits as a dense open subset in $X$.

\begin{definition}
Let $D(X)$ denote the set of prime divisors of $X$ that are not $G$-stable. Let $D^B(X) \subset D(X)$ denote those divisors
that are $B$-stable.  We refer to elements of 
$D^B(X)$ as \emph{colors} of $X$.
We say an open $G$-stable subset $X' \subset X$ is a \emph{simple embedding} if $X'$ contains a unique closed $G$-orbit.
\end{definition}



\begin{remark}
In the special case that $X = G/H$, the colors are simply the $B$-stable divisors, since no divisor will be $G$-stable. 
Moreover, the colors of $X$ are in bijection with the colors of $G/H$, since they are precisely the closures in $X$ of the $B$-stable divisors in $G/H$.  Thus, the set of colors of $X$ depends only on $G/H$.  However, we will be more interested in the colors of $X$ which contain particular $G$-orbits of $X$, as this data shall classify the equivariant embedding $G/H \hookrightarrow X$.

We also note that this definition of colors is the same for general spherical varieties.
\end{remark}


We construct a colored fan $\mathbb{F}(X) \subset N_\R$ as follows. 

Given a spherical variety $X$ and its colored fan $\mathbb{F}(X) = \{(\mathcal{C_\alpha}, \mathcal{F_\alpha})\} \subset N_\R$, we can consider the associated ordinary fan $\{C_\alpha\} \subset N_\R$.  This ordinary fan determies a toroidal variety $Y$ which maps surjectively onto $X$ by a proper birational $G$-equivariant map (see Timoshev $\S 29.1$).  Moreover, this ordinary fan also determines a toric variety $Z$, by the usual construction.

In fact, by (Timoshev $\S 29.1$) we have
\begin{thm}
The toroidal variety $Y$ is covered by $G$-translates of an open
$P(G/H)$-stable subset
\[
Y^\circ := Y \setminus \cup_{D \in D^B(Y) \cong P(G/H)_u \times Z},
\]
where $P(G/H)$ is the parabolic corresponding to $G/H$ (see Timoshev Theorem $4.6$, page $17$).
\end{thm}

If we assume $X$ is horospherical, then the associated toroidal $Y$ is also horospherical, and by (Perrin Fact $13.2.6$, page $99$):
\begin{thm}
$Z$ is a toric variety for the torus $P/H$, and $Y = G \times^P Z$.
\end{thm}
In other words, we have a $G$-homogeneous compactification
$G/H \to G \times^P Z$ of fiber bundles over $G/P$,
where the fibers on the left are isomorphic to the torus $P/H$, 
and the fibers on the right are isomorphic to the $(P/H)$-toric variety $Z$.
\end{section}


DEFINITION OF COLOR
DEFINITION OF FAN
DEFINITION OF TOROIDAL







\end{document}