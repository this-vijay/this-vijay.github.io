
\documentclass{amsart}
%\usepackage{cmbright}
\usepackage{amsmath, amsfonts, amssymb, amscd}
%\usepackage{tableau}
\usepackage{color}
\usepackage{xcolor}
\usepackage{graphicx}
\usepackage{array}
\usepackage{mathtools}
\usepackage{multirow}
\usepackage{framed}
\usepackage{tikz}
\usetikzlibrary{matrix,arrows}
\usepackage[square,sort,comma,numbers]{natbib}
\usepackage{enumerate}




%\oddsidemargin.5cm
%\evensidemargin.5cm
\addtolength{\oddsidemargin}{-.525in}
\addtolength{\evensidemargin}{-.525in}
\addtolength{\textwidth}{1in}

\addtolength{\topmargin}{-.87in}
\addtolength{\textheight}{1.5in}




\newcommand\coolunder[2]{\mathrlap{\smash{\underbrace{\phantom{%
    \begin{matrix} #2 \end{matrix}}}_{\mbox{$#1$}}}}#2}

%\input{tableau}


\newcommand{\+}[1]{\ensuremath{\mathbf{#1}}}
\newcommand{\vect}[1]{\boldsymbol{#1}} % Uncomment for BOLD vectors.
%\newcommand{\vect}[1]{\vec{#1}} % Uncomment for ARROW vectors.
\newcommand{\C}{{\mathbb C}}
\newcommand{\Z}{{\mathbb Z}}
\renewcommand{\P}{{\mathbb P}}
\newcommand{\OG}{\operatorname{OG}}
\newcommand{\OF}{\operatorname{OF}}
\newcommand{\bull}{{\scriptscriptstyle \bullet}}
\newcommand{\la}{\lambda}
\newcommand{\euler}[1]{\chi_{_{#1}}}
\newcommand{\cO}{{\mathcal O}}
\newcommand{\cG}{{\mathcal G}}
\newcommand{\cQ}{{\mathcal Q}}
\newcommand{\R}{{\mathbb R}}
\newcommand{\wt}{\widetilde}
\newcommand{\diag}{\operatorname{diag}}
\newcommand{\comp}{\operatorname{comp}}
\newcommand{\comment}[1]{}
\newcommand{\type}{\mathfrak{t}}
\newcommand{\op}{\text{op}}
\newcommand{\row}{{\bf r}}
\newcommand{\col}{{\bf c}}
\newcommand{\sym}{\mathfrak{S}}
\newcommand{\codim}{\text{codim}}
\DeclarePairedDelimiter{\ceil}{\lceil}{\rceil}
\DeclarePairedDelimiter{\floor}{\lfloor}{\rfloor}
\renewcommand{\emptyset}{\varnothing}
\renewcommand{\tilde}{\widetilde}


\newtheorem{thm}{Theorem}
\newtheorem{lemma}[thm]{Lemma}
\newtheorem{prop}[thm]{Proposition}
\newtheorem{cor}[thm]{Corollary}

\theoremstyle{definition}
\newtheorem{definition}[thm]{Definition}
\newtheorem{example}[thm]{Example}
\newtheorem{conj}[thm]{Conjecture}
\newtheorem{obs}[thm]{Observation}
\newtheorem{fact}[thm]{Fact}
\newtheorem{remark}[thm]{Remark}
\newtheorem{prob}{Problem}
\newtheorem{chal}{Challenge}

\begin{document}
\title{Problem Set 5}
\author{Topics in Manifolds, Spring 2016}

\maketitle

\begin{prob}
Consider the horocycle $C = \{x + iy \in \mathbb{H}^2 : x^2 + (y - \frac{1}{2})^2 = (\frac{1}{2})^2\}$, where $\mathbb{H}^2$ denotes the upper half plane with the hyperbolic metric $ds$.   Let $\mathbb{E}^1$ denote the real line $\R^1$ with the standard Euclidean metric $dt$.  Produce an explicit isometry from $\mathbb{E}^1 \to C$; i.e. produce a continuous map $\gamma: \mathbb{E}^1 \to C$ such that $\gamma^*(ds) = dt$.
\end{prob}

\begin{prob}
This problem refers to the hyperbolic polygon $R$ in Figure $5.12$ from Stillwell.
\begin{enumerate}[(a)]
 \item Show that the result of identifying edges of $R$ as shown in the figure
 is topologically the torus with one puncture (see Stillwell $5.4.1$).
\item Find explicit hyperbolic translations $g$, $h$ which 
realize the identifications shown in the figure, and show
that they generate a free group (see Stillwell $5.4.2$).
\item Show that $R$ is the Dirichlet region for the group $\langle g, h \rangle$ (see Stillwell $5.8.1$).
\end{enumerate}
\end{prob}


\begin{prob}
By a suitable choice of hyperbolic polygon, show that the sphere with $n \geq 4$ punctures
is a hyperbolic surface (see Stillwell $5.5.1$).
\end{prob}









\end{document}