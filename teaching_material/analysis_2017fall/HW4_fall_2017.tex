
\documentclass{amsart}
%\usepackage{cmbright}
\usepackage{amsmath, amsfonts, amssymb, amscd}
%\usepackage{tableau}
\usepackage{color}
\usepackage{xcolor}
\usepackage{graphicx}
\usepackage{array}
\usepackage{mathtools}
\usepackage{multirow}
\usepackage{framed}
\usepackage{tikz}
\usetikzlibrary{matrix,arrows}
\usepackage[square,sort,comma,numbers]{natbib}
\usepackage{enumerate}




%\oddsidemargin.5cm
%\evensidemargin.5cm
\addtolength{\oddsidemargin}{-.525in}
\addtolength{\evensidemargin}{-.525in}
\addtolength{\textwidth}{1in}

\addtolength{\topmargin}{-.87in}
\addtolength{\textheight}{1.5in}




\newcommand\coolunder[2]{\mathrlap{\smash{\underbrace{\phantom{%
    \begin{matrix} #2 \end{matrix}}}_{\mbox{$#1$}}}}#2}

%\input{tableau}


\newcommand{\+}[1]{\ensuremath{\mathbf{#1}}}
\newcommand{\vect}[1]{\boldsymbol{#1}} % Uncomment for BOLD vectors.
%\newcommand{\vect}[1]{\vec{#1}} % Uncomment for ARROW vectors.
\newcommand{\C}{{\mathbb C}}
\newcommand{\Z}{{\mathbb Z}}
\renewcommand{\P}{{\mathbb P}}
\newcommand{\OG}{\operatorname{OG}}
\newcommand{\OF}{\operatorname{OF}}
\newcommand{\bull}{{\scriptscriptstyle \bullet}}
\newcommand{\la}{\lambda}
\newcommand{\euler}[1]{\chi_{_{#1}}}
\newcommand{\cO}{{\mathcal O}}
\newcommand{\cG}{{\mathcal G}}
\newcommand{\cQ}{{\mathcal Q}}
\newcommand{\R}{{\mathbb R}}
\newcommand{\wt}{\widetilde}
\newcommand{\diag}{\operatorname{diag}}
\newcommand{\comp}{\operatorname{comp}}
\newcommand{\comment}[1]{}
\newcommand{\type}{\mathfrak{t}}
\newcommand{\op}{\text{op}}
\newcommand{\row}{{\bf r}}
\newcommand{\col}{{\bf c}}
\newcommand{\sym}{\mathfrak{S}}
\newcommand{\codim}{\text{codim}}
\DeclarePairedDelimiter{\ceil}{\lceil}{\rceil}
\DeclarePairedDelimiter{\floor}{\lfloor}{\rfloor}
\renewcommand{\emptyset}{\varnothing}

\newtheorem{thm}{Theorem}
\newtheorem{lemma}[thm]{Lemma}
\newtheorem{prop}[thm]{Proposition}
\newtheorem{cor}[thm]{Corollary}

\theoremstyle{definition}
\newtheorem{definition}[thm]{Definition}
\newtheorem{example}[thm]{Example}
\newtheorem{conj}[thm]{Conjecture}
\newtheorem{obs}[thm]{Observation}
\newtheorem{fact}[thm]{Fact}
\newtheorem{remark}[thm]{Remark}
\newtheorem{prob}{Problem}
\newtheorem{chal}{Challenge}

\begin{document}
\title{Problem Set 4}
\date{August 28, 2017}
\author{Intro to Real Analysis}

\maketitle




\begin{prob}
\begin{enumerate}[(a)]
\item Use the Monotone Convergence Theorem
to prove the Archimedean Property
without making any use of the Axiom of Completeness (i.e. without using the Least Upper Bound Property).
\item Use the Monotone Convergence Theorem to prove the Nested Interval Property, again without
using the AoC.
\end{enumerate}
\end{prob}

\begin{prob}
 For each series, find an explicit formula for the sequence of partial sums and determine
 if the series converges.
 \begin{enumerate}[(a)]
  \item $\sum^{\infty}_{n=1}\frac{1}{2^n}$
  \vspace{2mm}
  \item $\sum^{\infty}_{n=1}\frac{1}{n(n+1)}$
  \vspace{2mm}
  \item $\sum^{\infty}_{n=1}\text{log}\left( \frac{n+1}{n} \right)$
 \end{enumerate}
\end{prob}


\begin{prob}
 Let $(b_n)$ be a decreasing sequence of nonnegative terms.  Show that if the series $\sum^{\infty}_{n=0}2^n b_{2^n}$ diverges, then so does
 the series $\sum^{\infty}_{n=1} b_n$.
\end{prob}

\begin{prob}
 \begin{enumerate}[(a)]
  \item Prove that if an infinite series converges, then the associative property holds.
  In particular, assume $\sum^{\infty}_{n=1} a_n = L$.  Show that any regrouping of terms
  \[
  (a_1 + \cdots + a_{n_1}) + (a_{n_1+1} + \cdots + a_{n_2}) + (a_{n_2 +1} + \cdots + a_{n_3}) + \cdots
  \]
  leads to a series that also converges to $L$.
  \item We have seen that addition fails to be associate for the series $\sum^{\infty}_{n=1} (-1)^n$.  Explain
  where the proof in (a) breaks down in this example.
 \end{enumerate}
\end{prob}

\begin{prob}
Provide a proof of the Axiom of Completeness (i.e. the Least Upper Bound Property) using only
the Nested Interval Property and the property that $(\frac{1}{2^n}) \to 0$.  The proof strategy
we used for the Bolzano-Weierstrass Theorem may come in handy.
\end{prob}




\vspace{5mm}

{\bf The following problem is optional.  It
will not contribute to or detract from your grade, but you are encouraged
to attempt it.}

\vspace{5mm}

\begin{chal}
Let $(a_n)$ be a bounded sequence.
\begin{enumerate}[(a)]
 \item Prove that the sequence defined by $y_n = \text{sup}\{a_k:k \geq n\}$ converges.
 \item The \emph{limit superior} of $(a_n)$, or lim sup $a_n$, is defined by 
 \[
 \text{lim sup } a_n = \text{lim } y_n,
 \]
 where $y_n$ is the sequence from part $(a)$.  Provide a reasonable definition for
 lim inf $a_n$ and briefly explain why it always exists for any bounded sequence.
 \item Prove that lim inf $a_n \leq $ lim sup $a_n$ for every bounded sequence, and give
 an example of a sequence for which the inequality is strict.
 \item Show that lim inf $a_n = $ lim sup $a_n$ if and only if lim $a_n$ exists.
 In this case all three share the same value.
\end{enumerate}
\end{chal}




\vspace{5mm}

*All questions taken from \emph{Understanding Analysis: 2nd Edition} by Stephen Abbott.


\end{document}