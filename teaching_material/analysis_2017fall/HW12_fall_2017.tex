
\documentclass{amsart}
%\usepackage{cmbright}
\usepackage{amsmath, amsfonts, amssymb, amscd}
%\usepackage{tableau}
\usepackage{color}
\usepackage{xcolor}
\usepackage{graphicx}
\usepackage{array}
\usepackage{mathtools}
\usepackage{multirow}
\usepackage{framed}
\usepackage{tikz}
\usetikzlibrary{matrix,arrows}
\usepackage[square,sort,comma,numbers]{natbib}
\usepackage{enumerate}




%\oddsidemargin.5cm
%\evensidemargin.5cm
\addtolength{\oddsidemargin}{-.525in}
\addtolength{\evensidemargin}{-.525in}
\addtolength{\textwidth}{1in}

\addtolength{\topmargin}{-.87in}
\addtolength{\textheight}{1.5in}




\newcommand\coolunder[2]{\mathrlap{\smash{\underbrace{\phantom{%
    \begin{matrix} #2 \end{matrix}}}_{\mbox{$#1$}}}}#2}

%\input{tableau}


\newcommand{\+}[1]{\ensuremath{\mathbf{#1}}}
\newcommand{\vect}[1]{\boldsymbol{#1}} % Uncomment for BOLD vectors.
%\newcommand{\vect}[1]{\vec{#1}} % Uncomment for ARROW vectors.
\newcommand{\C}{{\mathbb C}}
\newcommand{\Z}{{\mathbb Z}}
\renewcommand{\P}{{\mathbb P}}
\newcommand{\OG}{\operatorname{OG}}
\newcommand{\OF}{\operatorname{OF}}
\newcommand{\bull}{{\scriptscriptstyle \bullet}}
\newcommand{\la}{\lambda}
\newcommand{\euler}[1]{\chi_{_{#1}}}
\newcommand{\cO}{{\mathcal O}}
\newcommand{\cG}{{\mathcal G}}
\newcommand{\cQ}{{\mathcal Q}}
\newcommand{\R}{{\mathbb R}}
\newcommand{\wt}{\widetilde}
\newcommand{\diag}{\operatorname{diag}}
\newcommand{\comp}{\operatorname{comp}}
\newcommand{\comment}[1]{}
\newcommand{\type}{\mathfrak{t}}
\newcommand{\op}{\text{op}}
\newcommand{\row}{{\bf r}}
\newcommand{\col}{{\bf c}}
\newcommand{\sym}{\mathfrak{S}}
\newcommand{\codim}{\text{codim}}
\DeclarePairedDelimiter{\ceil}{\lceil}{\rceil}
\DeclarePairedDelimiter{\floor}{\lfloor}{\rfloor}
\renewcommand{\emptyset}{\varnothing}

\newtheorem{thm}{Theorem}
\newtheorem{lemma}[thm]{Lemma}
\newtheorem{prop}[thm]{Proposition}
\newtheorem{cor}[thm]{Corollary}

\theoremstyle{definition}
\newtheorem{definition}[thm]{Definition}
\newtheorem{example}[thm]{Example}
\newtheorem{conj}[thm]{Conjecture}
\newtheorem{obs}[thm]{Observation}
\newtheorem{fact}[thm]{Fact}
\newtheorem{remark}[thm]{Remark}
\newtheorem{prob}{Problem}
\newtheorem{chal}{Challenge}

\begin{document}
\title{Problem Set 12}
\date{November 12, 2017}
\author{Intro to Real Analysis}

\maketitle



\begin{prob}
Let 
\[
g(x) = \frac{nx + x^2}{2n},
\]
and set $g(x) = \lim g_n(x)$.  Show that
$g$ is differentiable in two ways:
\begin{enumerate}[(a)]
 \item Compute g(x) by algebraically taking the limit
 as $n \to \infty$ and then find $g'(x)$.
 \item Compute $g'_n(x)$ for each $n \in \mathbb{N}$ and
 show that the sequence of derivatives $(g'_n)$
 converges uniformly on every interval $[-M,M]$.
 Cite the appropriate theorem to conclude
 that $g'(x) = \lim g'_n(x)$.
\end{enumerate}
\end{prob}

\begin{prob}
Decide whether each proposition is true or false, providing
a short justification or counterexample as appropriate.
\begin{enumerate}[(a)]
 \item If $\sum^{\infty}_{n=1} g_n$ converges
 uniformly, then $(g_n)$ converges uniformly to zero.
 \item If $0 \leq f_n(x) \leq g_n(x)$ and
 $\sum^{\infty}_{n=1} g_n$ converges uniformly,
 then $\sum^{\infty}_{n=1} f_n$ converges uniformly.
 \item If $\sum^{\infty}_{n=1} f_n$ converges uniformly
 on $A$, then there exist constants $M_n$ such that
 $|f_n(x)| \leq M_n$ for all $x \in A$
 and $\sum^{\infty}_{n=1} M_n$ converges.
\end{enumerate}

\end{prob}

\begin{prob}
 \begin{enumerate}[(a)]
  \item Prove that 
  \[
 h(x) = \sum^{\infty}_{n=1}\frac{x^n}{n^2}
 = x + \frac{x^2}{4} + \frac{x^3}{9} + \cdots
 \]
 is continuous on $[-1,1]$.
 \item Note that the series
   \[
 f(x) = \sum^{\infty}_{n=1}\frac{x^n}{n}
 = x + \frac{x^2}{2} + \frac{x^3}{3} + \cdots
 \]
 converges for every $x$ in the interval $[-1,1)$ but
 does not converge when $x = 1$.  For a fixed
 $x_0 \in (-1,1)$, explain how we can still use the 
 Weierstrass M-Test to prove that $f$
 is continuous at $x_0$.
 \end{enumerate}

\end{prob}

\begin{prob}
 Let \[
f(x) = \frac{1}{x} - \frac{1}{x+1} + \frac{1}{x+2}
- \frac{1}{x+3} + \cdots .
\]
Show that $f$ is defined for all $x > 0$.
Is $f$ continuous on $(0, \infty)$?
Is $f$ differentiable?
\end{prob}

\begin{prob}
Let $\{r_1, r_2, \ldots\}$
be an enumeration of the set of rational numbers.
For each $r_n \in \mathbb{Q}$, define
\[
 u_n(x) =
 \begin{cases}
  1/2^n &\text{ for } x > r_n \\
  0 &\text{ for } x \leq r_n.
 \end{cases}
\]
Let $h(x) = \sum^{\infty}_{n=1} u_n(x).$
Prove that $h$ is a monotone function defined on
all of $\R$ that is continuous at every irrational point.
\end{prob}



\vspace{5mm}

*All questions taken from \emph{Understanding Analysis: 2nd Edition} by Stephen Abbott.


\end{document}