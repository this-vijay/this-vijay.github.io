
\documentclass{amsart}
%\usepackage{cmbright}
\usepackage{amsmath, amsfonts, amssymb, amscd}
%\usepackage{tableau}
\usepackage{color}
\usepackage{xcolor}
\usepackage{graphicx}
\usepackage{array}
\usepackage{mathtools}
\usepackage{multirow}
\usepackage{framed}
\usepackage{tikz}
\usetikzlibrary{matrix,arrows}
\usepackage[square,sort,comma,numbers]{natbib}
\usepackage{enumerate}




%\oddsidemargin.5cm
%\evensidemargin.5cm
\addtolength{\oddsidemargin}{-.525in}
\addtolength{\evensidemargin}{-.525in}
\addtolength{\textwidth}{1in}

\addtolength{\topmargin}{-.87in}
\addtolength{\textheight}{1.5in}




\newcommand\coolunder[2]{\mathrlap{\smash{\underbrace{\phantom{%
    \begin{matrix} #2 \end{matrix}}}_{\mbox{$#1$}}}}#2}

%\input{tableau}


\newcommand{\+}[1]{\ensuremath{\mathbf{#1}}}
\newcommand{\vect}[1]{\boldsymbol{#1}} % Uncomment for BOLD vectors.
%\newcommand{\vect}[1]{\vec{#1}} % Uncomment for ARROW vectors.
\newcommand{\C}{{\mathbb C}}
\newcommand{\Z}{{\mathbb Z}}
\renewcommand{\P}{{\mathbb P}}
\newcommand{\OG}{\operatorname{OG}}
\newcommand{\OF}{\operatorname{OF}}
\newcommand{\bull}{{\scriptscriptstyle \bullet}}
\newcommand{\la}{\lambda}
\newcommand{\euler}[1]{\chi_{_{#1}}}
\newcommand{\cO}{{\mathcal O}}
\newcommand{\cG}{{\mathcal G}}
\newcommand{\cQ}{{\mathcal Q}}
\newcommand{\R}{{\mathbb R}}
\newcommand{\wt}{\widetilde}
\newcommand{\diag}{\operatorname{diag}}
\newcommand{\comp}{\operatorname{comp}}
\newcommand{\comment}[1]{}
\newcommand{\type}{\mathfrak{t}}
\newcommand{\op}{\text{op}}
\newcommand{\row}{{\bf r}}
\newcommand{\col}{{\bf c}}
\newcommand{\sym}{\mathfrak{S}}
\newcommand{\codim}{\text{codim}}
\DeclarePairedDelimiter{\ceil}{\lceil}{\rceil}
\DeclarePairedDelimiter{\floor}{\lfloor}{\rfloor}
\renewcommand{\emptyset}{\varnothing}

\newtheorem{thm}{Theorem}
\newtheorem{lemma}[thm]{Lemma}
\newtheorem{prop}[thm]{Proposition}
\newtheorem{cor}[thm]{Corollary}

\theoremstyle{definition}
\newtheorem{definition}[thm]{Definition}
\newtheorem{example}[thm]{Example}
\newtheorem{conj}[thm]{Conjecture}
\newtheorem{obs}[thm]{Observation}
\newtheorem{fact}[thm]{Fact}
\newtheorem{remark}[thm]{Remark}
\newtheorem{prob}{Problem}
\newtheorem{chal}{Challenge}

\begin{document}
\title{Problem Set 3}
\date{August 21, 2017}
\author{Intro to Real Analysis}

\maketitle




\begin{prob}
\begin{enumerate}[(a)]
 \item Show that $(a,b) \sim \R$ for any interval $(a,b)$.
 \item Show that $(a, \infty) \sim \R$
 for any unbounded interval $(a, \infty)$.
 \item Show that $[0,1) \sim (0,1)$
 by exhibiting a $1-1$ onto function between the sets.
\end{enumerate}
\end{prob}

\begin{prob}
\begin{enumerate}[(a)]
 \item Give an example of a countable collection of 
 disjoint open intervals.
 \item Give an example of an uncountable collection of 
 disjoint open intervals, or argue that no such 
 collection exists.
\end{enumerate}
\end{prob}

\begin{prob}
 A real number $x \in \R$ is called \emph{algebraic}
 if it is a root of a polynomial with integer coefficients;
 in other words, if there exists integers $a_0, a_1, a_2 \ldots, a_n \in \Z$,
 not all zero, such that
 \[
 a_nx^n + a_{n-1}x^{n-1} + \cdots + a_1x + a_0 = 0.
 \]
 Real numbers that are not algebraic are called
 \emph{transcendental}.
 \begin{enumerate}[(a)]
  \item Show that $\sqrt{2}$ and $\sqrt{2} + \sqrt{3}$
  are algebraic.
  \item Fix $n \in \mathbb{N}$, and let $A_n$ be the algebraic
  numbers obtained as roots of polynomials with integer coefficients
  that have degree $n$.
  Show that $A_n$ is countable.
  \item Conclude that the set of algebraic numbers
  is countable (feel free to use a result from class here).
  What does this imply about the set of
  transcendental numbers?
 \end{enumerate}
\end{prob}


\begin{prob}
 \begin{enumerate}
  \item Let $C \subset [0,1]$ be uncountable. 
  Show that there exists $a \in (0,1)$
  such that $C \cap [a,1]$ is uncountable.
  \item Now let $A$ be the set of all
  $a \in (0,1)$ such that $C \cap [a,1]$
  is uncountable, and set $\alpha = $ sup$A$.
  Is $C \cap [\alpha,1]$ an uncountable set?
  \item Does the statement in (a) remain
  true if 'uncountable'
  is replaced by 'infinite'?
 \end{enumerate}
\end{prob}


\begin{prob}
 Prove that the limit of a sequence, when it exists, must be unique.
 To get started, assume that $(a_n) \to a$
 and also that $(a_n) \to b$.  Now argue $a = b$.
\end{prob}


\vspace{5mm}

{\bf The following problems are optional.  They
will not contribute to or detract from your grade, but you are encouraged
to think about them.}

\vspace{5mm}

\begin{chal}
 Construct a $1-1$ function from $\R$
 to $P(\mathbb{N})$,
 the power set of the natural numbers.
 Construct a $1-1$ function in the reverse
 direction as well.
 (By the Schroeder-Bertstein Theorem (see Exercise 1.5.11
 of the textbook), this implies that $\R \sim P(\mathbb{N})$).
 
\end{chal}

\begin{chal}
Given a set $B$, a subset $\mathcal{A}$ of the power set 
$P(B)$ is called an \emph{antichain}
if no element of $\mathcal{A}$ is a subset
of any other element of $\mathcal{A}$.
Does $P(\mathbb{N})$
contain an uncountable antichain?
\end{chal}



\vspace{5mm}

*All questions taken from \emph{Understanding Analysis: 2nd Edition} by Stephen Abbott.


\end{document}