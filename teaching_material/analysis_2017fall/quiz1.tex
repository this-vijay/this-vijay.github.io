
\documentclass{amsart}
%\usepackage{cmbright}
\usepackage{amsmath, amsfonts, amssymb, amscd}
%\usepackage{tableau}
\usepackage{color}
\usepackage{xcolor}
\usepackage{graphicx}
\usepackage{array}
\usepackage{mathtools}
\usepackage{multirow}
\usepackage{framed}
\usepackage{tikz}
\usetikzlibrary{matrix,arrows}
\usepackage[square,sort,comma,numbers]{natbib}
\usepackage{enumerate}




%\oddsidemargin.5cm
%\evensidemargin.5cm
\addtolength{\oddsidemargin}{-.525in}
\addtolength{\evensidemargin}{-.525in}
\addtolength{\textwidth}{1in}

\addtolength{\topmargin}{-.87in}
\addtolength{\textheight}{1.5in}




\newcommand\coolunder[2]{\mathrlap{\smash{\underbrace{\phantom{%
    \begin{matrix} #2 \end{matrix}}}_{\mbox{$#1$}}}}#2}

%\input{tableau}


\newcommand{\+}[1]{\ensuremath{\mathbf{#1}}}
\newcommand{\vect}[1]{\boldsymbol{#1}} % Uncomment for BOLD vectors.
%\newcommand{\vect}[1]{\vec{#1}} % Uncomment for ARROW vectors.
\newcommand{\C}{{\mathbb C}}
\newcommand{\Z}{{\mathbb Z}}
\renewcommand{\P}{{\mathbb P}}
\newcommand{\OG}{\operatorname{OG}}
\newcommand{\OF}{\operatorname{OF}}
\newcommand{\bull}{{\scriptscriptstyle \bullet}}
\newcommand{\la}{\lambda}
\newcommand{\euler}[1]{\chi_{_{#1}}}
\newcommand{\cO}{{\mathcal O}}
\newcommand{\cG}{{\mathcal G}}
\newcommand{\cQ}{{\mathcal Q}}
\newcommand{\R}{{\mathbb R}}
\newcommand{\wt}{\widetilde}
\newcommand{\diag}{\operatorname{diag}}
\newcommand{\comp}{\operatorname{comp}}
\newcommand{\comment}[1]{}
\newcommand{\type}{\mathfrak{t}}
\newcommand{\op}{\text{op}}
\newcommand{\row}{{\bf r}}
\newcommand{\col}{{\bf c}}
\newcommand{\sym}{\mathfrak{S}}
\newcommand{\codim}{\text{codim}}
\DeclarePairedDelimiter{\ceil}{\lceil}{\rceil}
\DeclarePairedDelimiter{\floor}{\lfloor}{\rfloor}
\renewcommand{\emptyset}{\varnothing}

\newtheorem{thm}{Theorem}
\newtheorem{lemma}[thm]{Lemma}
\newtheorem{prop}[thm]{Proposition}
\newtheorem{cor}[thm]{Corollary}

\theoremstyle{definition}
\newtheorem{definition}[thm]{Definition}
\newtheorem{example}[thm]{Example}
\newtheorem{conj}[thm]{Conjecture}
\newtheorem{obs}[thm]{Observation}
\newtheorem{fact}[thm]{Fact}
\newtheorem{remark}[thm]{Remark}
\newtheorem{prob}{Problem}
\newtheorem{chal}{Challenge}

\begin{document}
\title{Quiz}
\date{August 30, 2017}
\author{Intro to Real Analysis}

\maketitle




\begin{prob}
\begin{enumerate}[(a)]
\item Show that 
\[\sqrt{2}, \sqrt{2+\sqrt{2}}, \sqrt{2+\sqrt{2+\sqrt{2}}}, \ldots\]
converges and find the limit.
\item Does
\[\sqrt{2}, \sqrt{2\sqrt{2}}, \sqrt{2\sqrt{2\sqrt{2}}}, \ldots\]
converge? If so, find the limit.
\end{enumerate}
\end{prob}

\begin{prob}
 Give an example of each of the following, or argue that such a request is impossible.
   \vspace{2mm}
 \begin{enumerate}[(a)]
  \item A sequence that has a subsequence that is bounded but contains no subsequence that converges.
  \vspace{2mm}
  \item A sequence that does not contain $0$ or $1$ as a term but contains subsequences converging to each of these values.
  \vspace{2mm}
  \item A sequence that contains subsequences converging to every point in the infinite set 
  \[\{1, \frac{1}{2}, \frac{1}{3}, \frac{1}{4}, \frac{1}{5}, \ldots\}.\] 
   \vspace{2mm}
   \item A sequence that contains subsequences converging to every point in the infinite set 
   \[\{1, \frac{1}{2}, \frac{1}{3}, \frac{1}{4}, \frac{1}{5}, \ldots\},\] 
   and no subsequences converging to points outside of this set. 
 \end{enumerate}
\end{prob}

\begin{prob}
Decide whether the following propositions are true or false, providing a short justification for each conclusion.
  \vspace{2mm}
 \begin{enumerate}[(a)]
  \item If every proper subsequence of $(x_n)$ converges, then $(x_n)$ converges as well.
  \vspace{2mm}
  \item If $(x_n)$ contains a divergent subsequence, then $(x_n)$ diverges.
  \vspace{2mm}
  \item If $(x_n)$ is bounded and diverges, then there exist two subsequences of $(x_n)$ that converge 
  to different limits.
  \vspace{2mm}
  \item If $(x_n)$ is monotone and contains a convergent subsequence, then $(x_n)$ converges.
 \end{enumerate}

\end{prob}




\end{document}