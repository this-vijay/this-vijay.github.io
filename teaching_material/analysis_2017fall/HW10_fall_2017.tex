
\documentclass{amsart}
%\usepackage{cmbright}
\usepackage{amsmath, amsfonts, amssymb, amscd}
%\usepackage{tableau}
\usepackage{color}
\usepackage{xcolor}
\usepackage{graphicx}
\usepackage{array}
\usepackage{mathtools}
\usepackage{multirow}
\usepackage{framed}
\usepackage{tikz}
\usetikzlibrary{matrix,arrows}
\usepackage[square,sort,comma,numbers]{natbib}
\usepackage{enumerate}




%\oddsidemargin.5cm
%\evensidemargin.5cm
\addtolength{\oddsidemargin}{-.525in}
\addtolength{\evensidemargin}{-.525in}
\addtolength{\textwidth}{1in}

\addtolength{\topmargin}{-.87in}
\addtolength{\textheight}{1.5in}




\newcommand\coolunder[2]{\mathrlap{\smash{\underbrace{\phantom{%
    \begin{matrix} #2 \end{matrix}}}_{\mbox{$#1$}}}}#2}

%\input{tableau}


\newcommand{\+}[1]{\ensuremath{\mathbf{#1}}}
\newcommand{\vect}[1]{\boldsymbol{#1}} % Uncomment for BOLD vectors.
%\newcommand{\vect}[1]{\vec{#1}} % Uncomment for ARROW vectors.
\newcommand{\C}{{\mathbb C}}
\newcommand{\Z}{{\mathbb Z}}
\renewcommand{\P}{{\mathbb P}}
\newcommand{\OG}{\operatorname{OG}}
\newcommand{\OF}{\operatorname{OF}}
\newcommand{\bull}{{\scriptscriptstyle \bullet}}
\newcommand{\la}{\lambda}
\newcommand{\euler}[1]{\chi_{_{#1}}}
\newcommand{\cO}{{\mathcal O}}
\newcommand{\cG}{{\mathcal G}}
\newcommand{\cQ}{{\mathcal Q}}
\newcommand{\R}{{\mathbb R}}
\newcommand{\wt}{\widetilde}
\newcommand{\diag}{\operatorname{diag}}
\newcommand{\comp}{\operatorname{comp}}
\newcommand{\comment}[1]{}
\newcommand{\type}{\mathfrak{t}}
\newcommand{\op}{\text{op}}
\newcommand{\row}{{\bf r}}
\newcommand{\col}{{\bf c}}
\newcommand{\sym}{\mathfrak{S}}
\newcommand{\codim}{\text{codim}}
\DeclarePairedDelimiter{\ceil}{\lceil}{\rceil}
\DeclarePairedDelimiter{\floor}{\lfloor}{\rfloor}
\renewcommand{\emptyset}{\varnothing}

\newtheorem{thm}{Theorem}
\newtheorem{lemma}[thm]{Lemma}
\newtheorem{prop}[thm]{Proposition}
\newtheorem{cor}[thm]{Corollary}

\theoremstyle{definition}
\newtheorem{definition}[thm]{Definition}
\newtheorem{example}[thm]{Example}
\newtheorem{conj}[thm]{Conjecture}
\newtheorem{obs}[thm]{Observation}
\newtheorem{fact}[thm]{Fact}
\newtheorem{remark}[thm]{Remark}
\newtheorem{prob}{Problem}
\newtheorem{chal}{Challenge}

\begin{document}
\title{Problem Set 10}
\date{October 30, 2017}
\author{Intro to Real Analysis}

\maketitle

\begin{prob}
 Recall that a function $f: A \to \R$ is \emph{Lipschitz} on $A$
 if there exists an $M > 0$ such that
 \[
 \left| \frac{f(x) - f(y)}{x-y} \right| \leq M
 \]
 for all $x \neq y$ in $A$.
 \begin{enumerate}[(a)]
  \item  Show that if $f$ is differentiable on a closed interval $[a,b]$
  and if $f'$ is continuous on $[a,b]$, then $f$ is Lipschitz on $[a,b]$.
  \item  The function $f$ is \emph{contractive} if there is a constant $c$
  such that $0<c<1$ and
  $|f(x) - f(y)| \leq c|x-y|$ for all $x,y \in A$.
  If we add the assumption that $|f'(x)| < 1$ on $[a,b]$, does
  it follow that $f$ is contractive on this set?
 \end{enumerate}
\end{prob}


\begin{prob}
 Let $f$ be differentiable on an interval $A$.  If $f'(x) \neq 0$ on $A$,
 show that $f$ is one-to-one on $A$.
 Provide an example to show that the converse statement need not be true.
\end{prob}

\begin{prob}
 A \emph{fixed point} of a function $f$ is a value $x$ where $f(x) = x$.
 Show that if $f$ is differentiable on an interval throughout which we have $f'(x) \neq 1$,
 then $f$ can have at most one fixed point.
\end{prob}

\begin{prob}
 Suppose $f$ and $g$ are continuous functions on an interval containing $a$,
 and that  $f$ and
 $g$ are differentiable on this interval (including at $a$). 
 Moreover, suppose that $f(a) = g(a) = 0$, that $f'$ and $g'$ are continuous at $a$,
 and that $g'(x) \neq 0$ throughout the interval (including at $a$).
 Find a short proof that
 \[
\lim_{x \to a} \frac{f'(x)}{g'(x)} = L \text{ implies } \lim_{x \to a} \frac{f(x)}{g(x)} = L.
 \]
 \end{prob}


\begin{prob}
Let $f: [a,b] \to \R$ be a one-to-one function that is both continuous and differentiable
on $[a,b]$.
Show that if $f'(x) \neq 0$ for all $x \in [a,b]$, then $f^{-1}$
is differentiable on the range of $f$, with
\[(f^{-1})'(y) = \frac{1}{f'(x)} \text{ where } y=f(x).\]
\end{prob}


\vspace{5mm}

*All questions taken from \emph{Understanding Analysis: 2nd Edition} by Stephen Abbott.


\end{document}