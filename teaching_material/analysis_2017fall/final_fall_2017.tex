
\documentclass{amsart}
%\usepackage{cmbright}
\usepackage{amsmath, amsfonts, amssymb, amscd}
%\usepackage{tableau}
\usepackage{color}
\usepackage{xcolor}
\usepackage{graphicx}
\usepackage{array}
\usepackage{mathtools}
\usepackage{multirow}
\usepackage{framed}
\usepackage{tikz}
\usetikzlibrary{matrix,arrows}
\usepackage[square,sort,comma,numbers]{natbib}
\usepackage{enumerate}
\usepackage{comment}



%\oddsidemargin.5cm
%\evensidemargin.5cm
\addtolength{\oddsidemargin}{-.525in}
\addtolength{\evensidemargin}{-.525in}
\addtolength{\textwidth}{1in}

\addtolength{\topmargin}{-.87in}
\addtolength{\textheight}{1.5in}




\newcommand\coolunder[2]{\mathrlap{\smash{\underbrace{\phantom{%
    \begin{matrix} #2 \end{matrix}}}_{\mbox{$#1$}}}}#2}

%\input{tableau}


\newcommand{\+}[1]{\ensuremath{\mathbf{#1}}}
\newcommand{\vect}[1]{\boldsymbol{#1}} % Uncomment for BOLD vectors.
%\newcommand{\vect}[1]{\vec{#1}} % Uncomment for ARROW vectors.
\newcommand{\C}{{\mathbb C}}
\newcommand{\Z}{{\mathbb Z}}
\renewcommand{\P}{{\mathbb P}}
\newcommand{\OG}{\operatorname{OG}}
\newcommand{\OF}{\operatorname{OF}}
\newcommand{\bull}{{\scriptscriptstyle \bullet}}
\newcommand{\la}{\lambda}
\newcommand{\euler}[1]{\chi_{_{#1}}}
\newcommand{\cO}{{\mathcal O}}
\newcommand{\cG}{{\mathcal G}}
\newcommand{\cQ}{{\mathcal Q}}
\newcommand{\R}{{\mathbb R}}
\newcommand{\wt}{\widetilde}
\newcommand{\diag}{\operatorname{diag}}
\newcommand{\comp}{\operatorname{comp}}
\newcommand{\type}{\mathfrak{t}}
\newcommand{\op}{\text{op}}
\newcommand{\row}{{\bf r}}
\newcommand{\col}{{\bf c}}
\newcommand{\sym}{\mathfrak{S}}
\newcommand{\codim}{\text{codim}}
\DeclarePairedDelimiter{\ceil}{\lceil}{\rceil}
\DeclarePairedDelimiter{\floor}{\lfloor}{\rfloor}
\renewcommand{\emptyset}{\varnothing}

\newtheorem{thm}{Theorem}
\newtheorem{lemma}[thm]{Lemma}
\newtheorem{prop}[thm]{Proposition}
\newtheorem{cor}[thm]{Corollary}

\theoremstyle{definition}
\newtheorem{definition}[thm]{Definition}
\newtheorem{example}[thm]{Example}
\newtheorem{conj}[thm]{Conjecture}
\newtheorem{obs}[thm]{Observation}
\newtheorem{fact}[thm]{Fact}
\newtheorem{remark}[thm]{Remark}
\newtheorem{prob}{Problem}
\newtheorem{chal}{Challenge}

\begin{document}
\title{Final Exam}
\date{November 30, 2017}
\author{Intro to Real Analysis}

\maketitle


The exam consists of five questions, each worth $20$ points.
You may take up to three hours to complete the exam.

\vspace{3mm}



\begin{prob}
Consider the functions $f_n: \R \to \R$ defined by
\[
f_n(x) = 
\begin{cases}
 0 &\text{ if } x \leq n \\
 x-n &\text{ if } n < x \leq n+1 \\
 1 &\text{ if } x > n+1.
\end{cases}
\]
 Is the sequence $\{f_n\}_{n \in \mathbb{N}}$ uniformly convergent?
Justify your answer.
 \end{prob}

 
\vspace{3mm}


\begin{prob} Determine whether each statement is true or false.
If false, provide a counterexample.  If true, then very briefly justify
your answer.
(As usual, all functions are assumed to be real-valued
with domains contained in $\R$,
and all sets assumed to be subsets of $\R$.)
 \begin{enumerate}[(a)]
  \item If a sequence of functions $(f_n)$ converges pointwise
  to a function $f$ on a compact set $K$, then
  $f_n \to f$ uniformly on $K$.
  \item If $(x_n)$ is a Cauchy sequence, then its
  image $(f(x_n))$ under a continuous function $f$
  is also a Cauchy sequence.
  \item A continuous function $f$ on a compact set $K$
  always attains a maximum value and a minimum value on $K$.
  \item If a function $f$ is represented
  by a power series $\sum a_n x^n$
  on its interval of convergence $I$,
  then $\sum a_n x^n$ converges uniformly
  to $f$ on $I$.
 \end{enumerate}
\end{prob}


\vspace{3mm}


%\begin{prob}
%Suppose $A \subset \R$ is both open and closed. 
%Prove that $A$ is either $\R$ or $\emptyset$.
%\end{prob}

\begin{prob}
 Let $f$ be a continuous function on the closed
 interval $[0,1]$
 with range also contained in $[0,1]$.
 Prove that $f$ must have a fixed point;
 i.e. that $f(x) = x$ for some $x \in [0,1]$.
\end{prob}


\vspace{3mm}


\begin{prob}
Assume $f$ is continuous on an interval containing zero
and differentiable for all $x \neq 0$.
If $\lim_{x \to 0} f'(x) = L$,
show that $f'(0)$ exists and equals $L$.
\end{prob}


\vspace{3mm}


\begin{prob}
\begin{enumerate}[(a)]
 \item  Suppose $f: A \to \R$ is uniformly continuous
 and $(x_n) \subset A$ is a Cauchy sequence.
 Show that $(f(x_n))$ is a Cauchy sequence.
 \item Let $g$ be a continuous function on
 an open interval $(a,b)$.
 Show that $g$ is uniformly continuous on $(a,b)$
 if and only if it is possible to define
values $g(a)$ and $g(b)$ at the endpoints
so that the extended function $g$ is continuous on $[a,b]$.
 \end{enumerate}
\end{prob}






\end{document}