
\documentclass{amsart}
%\usepackage{cmbright}
\usepackage{amsmath, amsfonts, amssymb, amscd}
%\usepackage{tableau}
\usepackage{color}
\usepackage{xcolor}
\usepackage{graphicx}
\usepackage{array}
\usepackage{mathtools}
\usepackage{multirow}
\usepackage{framed}
\usepackage{tikz}
\usetikzlibrary{matrix,arrows}
\usepackage[square,sort,comma,numbers]{natbib}
\usepackage{enumerate}
\usepackage{comment}



%\oddsidemargin.5cm
%\evensidemargin.5cm
\addtolength{\oddsidemargin}{-.525in}
\addtolength{\evensidemargin}{-.525in}
\addtolength{\textwidth}{1in}

\addtolength{\topmargin}{-.87in}
\addtolength{\textheight}{1.5in}




\newcommand\coolunder[2]{\mathrlap{\smash{\underbrace{\phantom{%
    \begin{matrix} #2 \end{matrix}}}_{\mbox{$#1$}}}}#2}

%\input{tableau}


\newcommand{\+}[1]{\ensuremath{\mathbf{#1}}}
\newcommand{\vect}[1]{\boldsymbol{#1}} % Uncomment for BOLD vectors.
%\newcommand{\vect}[1]{\vec{#1}} % Uncomment for ARROW vectors.
\newcommand{\C}{{\mathbb C}}
\newcommand{\Z}{{\mathbb Z}}
\renewcommand{\P}{{\mathbb P}}
\newcommand{\OG}{\operatorname{OG}}
\newcommand{\OF}{\operatorname{OF}}
\newcommand{\bull}{{\scriptscriptstyle \bullet}}
\newcommand{\la}{\lambda}
\newcommand{\euler}[1]{\chi_{_{#1}}}
\newcommand{\cO}{{\mathcal O}}
\newcommand{\cG}{{\mathcal G}}
\newcommand{\cQ}{{\mathcal Q}}
\newcommand{\R}{{\mathbb R}}
\newcommand{\wt}{\widetilde}
\newcommand{\diag}{\operatorname{diag}}
\newcommand{\comp}{\operatorname{comp}}
\newcommand{\type}{\mathfrak{t}}
\newcommand{\op}{\text{op}}
\newcommand{\row}{{\bf r}}
\newcommand{\col}{{\bf c}}
\newcommand{\sym}{\mathfrak{S}}
\newcommand{\codim}{\text{codim}}
\DeclarePairedDelimiter{\ceil}{\lceil}{\rceil}
\DeclarePairedDelimiter{\floor}{\lfloor}{\rfloor}
\renewcommand{\emptyset}{\varnothing}

\newtheorem{thm}{Theorem}
\newtheorem{lemma}[thm]{Lemma}
\newtheorem{prop}[thm]{Proposition}
\newtheorem{cor}[thm]{Corollary}

\theoremstyle{definition}
\newtheorem{definition}[thm]{Definition}
\newtheorem{example}[thm]{Example}
\newtheorem{conj}[thm]{Conjecture}
\newtheorem{obs}[thm]{Observation}
\newtheorem{fact}[thm]{Fact}
\newtheorem{remark}[thm]{Remark}
\newtheorem{prob}{Problem}
\newtheorem{chal}{Challenge}

\begin{document}
\title{Midterm Exam}
\date{September 25, 2017}
\author{Intro to Real Analysis}

\maketitle


The exam consists of five questions, each worth $20$ points.
You may take up to three hours to complete the exam.

\vspace{3mm}

\begin{prob}
Complete the following definitions.  Write in complete English sentences, 
and avoid use of quantifier symbols like $\forall$ and $\exists$.
\begin{enumerate}[(a)]
 \item A sequence $(a_n)$ \emph{converges} to a real number $a$ if $\ldots$  
 \item A sequence $(a_n)$ is a \emph{Cauchy sequence} if $\ldots$
 \item An infinite series $\sum^{\infty}_{n=1}a_n$ \emph{converges absolutely} if $\ldots$
\end{enumerate}
\end{prob}

\begin{prob}
Let us define an \emph{$AB$-sequence} to be a function from $\mathbb{N}$ to the two-element set $\{A,B\}$.
Thus, an $AB$-sequence can be thought of as an infinite string of the letters $A$ and $B$.
\begin{enumerate}[(a)]
 \item Is the set of $AB$-sequences uncountable? Give a proof of your answer.
 \item Define an $AB$-word to be a finite string of the letters $A$ and $B$.  Is the set of $AB$-words
 uncountable? Justify your answer.
\end{enumerate}
\end{prob}

\begin{prob}
 Suppose $(a_n)$ and $(b_n)$ are convergent sequences with limits $a$ and $b$ respectively.
 Prove that $(a_n+b_n)$ converges to $a+b$.  (Prove this directly from the definition of convergence,
 without appealing to the Algebraic Limit Theorem.)
\end{prob}

\begin{prob}
 Assume $(a_n)$ is a bounded sequence with the property that
 every convergent subsequence of $(a_n)$ converges to the same limit
 $a \in \R$.  Show that $(a_n)$ must converge to $a$.
\end{prob}



\begin{prob}
Prove that a set $F$ is closed if and only if its complement $F^c$ is open.
(Recall the definitions of open and closed sets:
we say a set $U \subset \R$ is \emph{open} if for any point $x \in U$, there exists a neighborhood
 $V_\epsilon(x)$ contained in $U$.  
And we say a set $F \subset \R$ is \emph{closed} if it contains all its limit points.) 
\end{prob}




\end{document}