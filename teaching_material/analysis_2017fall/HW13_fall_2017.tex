
\documentclass{amsart}
%\usepackage{cmbright}
\usepackage{amsmath, amsfonts, amssymb, amscd}
%\usepackage{tableau}
\usepackage{color}
\usepackage{xcolor}
\usepackage{graphicx}
\usepackage{array}
\usepackage{mathtools}
\usepackage{multirow}
\usepackage{framed}
\usepackage{tikz}
\usetikzlibrary{matrix,arrows}
\usepackage[square,sort,comma,numbers]{natbib}
\usepackage{enumerate}




%\oddsidemargin.5cm
%\evensidemargin.5cm
\addtolength{\oddsidemargin}{-.525in}
\addtolength{\evensidemargin}{-.525in}
\addtolength{\textwidth}{1in}

\addtolength{\topmargin}{-.87in}
\addtolength{\textheight}{1.5in}




\newcommand\coolunder[2]{\mathrlap{\smash{\underbrace{\phantom{%
    \begin{matrix} #2 \end{matrix}}}_{\mbox{$#1$}}}}#2}

%\input{tableau}


\newcommand{\+}[1]{\ensuremath{\mathbf{#1}}}
\newcommand{\vect}[1]{\boldsymbol{#1}} % Uncomment for BOLD vectors.
%\newcommand{\vect}[1]{\vec{#1}} % Uncomment for ARROW vectors.
\newcommand{\C}{{\mathbb C}}
\newcommand{\Z}{{\mathbb Z}}
\renewcommand{\P}{{\mathbb P}}
\newcommand{\OG}{\operatorname{OG}}
\newcommand{\OF}{\operatorname{OF}}
\newcommand{\bull}{{\scriptscriptstyle \bullet}}
\newcommand{\la}{\lambda}
\newcommand{\euler}[1]{\chi_{_{#1}}}
\newcommand{\cO}{{\mathcal O}}
\newcommand{\cG}{{\mathcal G}}
\newcommand{\cQ}{{\mathcal Q}}
\newcommand{\R}{{\mathbb R}}
\newcommand{\wt}{\widetilde}
\newcommand{\diag}{\operatorname{diag}}
\newcommand{\comp}{\operatorname{comp}}
\newcommand{\comment}[1]{}
\newcommand{\type}{\mathfrak{t}}
\newcommand{\op}{\text{op}}
\newcommand{\row}{{\bf r}}
\newcommand{\col}{{\bf c}}
\newcommand{\sym}{\mathfrak{S}}
\newcommand{\codim}{\text{codim}}
\DeclarePairedDelimiter{\ceil}{\lceil}{\rceil}
\DeclarePairedDelimiter{\floor}{\lfloor}{\rfloor}
\renewcommand{\emptyset}{\varnothing}

\newtheorem{thm}{Theorem}
\newtheorem{lemma}[thm]{Lemma}
\newtheorem{prop}[thm]{Proposition}
\newtheorem{cor}[thm]{Corollary}

\theoremstyle{definition}
\newtheorem{definition}[thm]{Definition}
\newtheorem{example}[thm]{Example}
\newtheorem{conj}[thm]{Conjecture}
\newtheorem{obs}[thm]{Observation}
\newtheorem{fact}[thm]{Fact}
\newtheorem{remark}[thm]{Remark}
\newtheorem{prob}{Problem}
\newtheorem{chal}{Challenge}

\begin{document}
\title{Power Series Practice}
\date{November 24, 2017}
\author{Intro to Real Analysis}

\maketitle

\begin{prob}
Recall the construction of the series representation
\[\arctan(x) = x - \frac{x^3}{3} + \frac{x^5}{5} - \frac{x^7}{7} + \cdots\]
for all $x \in (-1,1)$.
Notice that the series converges at $x = 1$.
Using the fact that $\arctan(x)$ is continuous,
prove that the series at $x=1$ is equal to $\arctan(1)$.
What identity does this equality yield?
\end{prob}

\begin{prob}
 Write down Taylor series for $\frac{1}{1-x}$
 and $\sin(x)$.  Manipulate these to produce
 Taylor series for the following functions.  For which
 values of $x$ is each series representation valid?
\begin{enumerate}[(a)]
 \item $x\cos(x^2)$
 \item $x/(1+4x^2)^2$
 \item $\ln (1+x^2)$
\end{enumerate}
\end{prob}

\begin{prob}
 Find an example of each of the following or explain why
 no such function exists.
 \begin{enumerate}[(a)]
  \item An infinitely differentiable function
  $g(x)$ on all of $\R$ with a Taylor series that
  converges to $g(x)$ only for $x \in (-1,1)$.
  \item An infinitely differentiable function
  $h(x)$ with the same Taylor series as $\sin(x)$
  but such that $h(x) \neq \sin(x)$ for all $x \neq 0$.
  \item An infinitely differentiable function $f(x)$
  on all of $\R$ with a Taylor series that converges
  to $f(x)$ if and only if $x \leq 0$.
 \end{enumerate}
\end{prob}

\begin{prob}
 Let $\sum a_n x^n$ be a power series with $a_n \neq 0$ for all $n$.
 Assume that
 \[
 L = \lim_{n \to \infty} \left| \frac{a_{n+1}}{a_n} \right|
 \]
 exists.
 \begin{enumerate}[(a)]
  \item Show that if $L \neq 0$, then the series
  converges for all $x$ in $(-1/L, 1/L)$.
  \item  Show that if $L = 0$, then the series
  converges for all $x \in \R$.
  \item
What can we say about the power series
$\sum a_n x^n$ if the sequence
$\left(\left|\frac{a_{n+1}}{a_n}\right|\right)$
is unbounded?

  \vspace{3mm}
  \text{\bf Extra, for those who have looked
  at $\lim \sup$ already}:
  \vspace{3mm}
  \item Show that (a) and (b)
  continue to hold if $L$ is replaced by the limit
  \[
  L' = \lim_{n \to \infty} \text{ where } s_n = \sup \left\{\left|\frac{a_{k+1}}{a_k}\right|:k \geq n\right\}.
  \]
(Recall that this \emph{limit superior} exists if
the sequence $\left(\left|\frac{a_{n+1}}{a_n}\right|\right)$
is bounded.)  
  \end{enumerate}
\end{prob}





\end{document}