
\documentclass{amsart}
%\usepackage{cmbright}
\usepackage{amsmath, amsfonts, amssymb, amscd}
%\usepackage{tableau}
\usepackage{color}
\usepackage{xcolor}
\usepackage{graphicx}
\usepackage{array}
\usepackage{mathtools}
\usepackage{multirow}
\usepackage{framed}
\usepackage{tikz}
\usetikzlibrary{matrix,arrows}
\usepackage[square,sort,comma,numbers]{natbib}
\usepackage{enumerate}




%\oddsidemargin.5cm
%\evensidemargin.5cm
\addtolength{\oddsidemargin}{-.525in}
\addtolength{\evensidemargin}{-.525in}
\addtolength{\textwidth}{1in}

\addtolength{\topmargin}{-.87in}
\addtolength{\textheight}{1.5in}




\newcommand\coolunder[2]{\mathrlap{\smash{\underbrace{\phantom{%
    \begin{matrix} #2 \end{matrix}}}_{\mbox{$#1$}}}}#2}

%\input{tableau}


\newcommand{\+}[1]{\ensuremath{\mathbf{#1}}}
\newcommand{\vect}[1]{\boldsymbol{#1}} % Uncomment for BOLD vectors.
%\newcommand{\vect}[1]{\vec{#1}} % Uncomment for ARROW vectors.
\newcommand{\C}{{\mathbb C}}
\newcommand{\Z}{{\mathbb Z}}
\renewcommand{\P}{{\mathbb P}}
\newcommand{\OG}{\operatorname{OG}}
\newcommand{\OF}{\operatorname{OF}}
\newcommand{\bull}{{\scriptscriptstyle \bullet}}
\newcommand{\la}{\lambda}
\newcommand{\euler}[1]{\chi_{_{#1}}}
\newcommand{\cO}{{\mathcal O}}
\newcommand{\cG}{{\mathcal G}}
\newcommand{\cQ}{{\mathcal Q}}
\newcommand{\R}{{\mathbb R}}
\newcommand{\wt}{\widetilde}
\newcommand{\diag}{\operatorname{diag}}
\newcommand{\comp}{\operatorname{comp}}
\newcommand{\comment}[1]{}
\newcommand{\type}{\mathfrak{t}}
\newcommand{\op}{\text{op}}
\newcommand{\row}{{\bf r}}
\newcommand{\col}{{\bf c}}
\newcommand{\sym}{\mathfrak{S}}
\newcommand{\codim}{\text{codim}}
\DeclarePairedDelimiter{\ceil}{\lceil}{\rceil}
\DeclarePairedDelimiter{\floor}{\lfloor}{\rfloor}
\renewcommand{\emptyset}{\varnothing}

\newtheorem{thm}{Theorem}
\newtheorem{lemma}[thm]{Lemma}
\newtheorem{prop}[thm]{Proposition}
\newtheorem{cor}[thm]{Corollary}

\theoremstyle{definition}
\newtheorem{definition}[thm]{Definition}
\newtheorem{example}[thm]{Example}
\newtheorem{conj}[thm]{Conjecture}
\newtheorem{obs}[thm]{Observation}
\newtheorem{fact}[thm]{Fact}
\newtheorem{remark}[thm]{Remark}
\newtheorem{prob}{Problem}
\newtheorem{chal}{Challenge}

\begin{document}
\title{Problem Set 1}
\date{August 7, 2017}
\author{Intro to Real Analysis}

\maketitle


\begin{prob}
 Prove that $\sqrt{3}$ is irrational.  Does
  a similar argument work to show that $\sqrt{6}$ is irrational?
  Where does your proof break down for $\sqrt{9}$?
\end{prob}

\begin{prob}
 Show that there is no rational number $r$ satisfying $2^r = 3$.
\end{prob}

\begin{prob}
 Decide which of the following represent true statements
 about the nature of sets.  For any that are false, provide
 a specific counterexample.
 \begin{enumerate}[(a)]
  \item If $A_1 \supseteq A_2 \supseteq A_3 \supseteq A_4 \cdots$
  are all sets containing an infinite number of elements,
  then the intersection $\bigcap^{\infty}_{n=1}A_n$ is infinite as well.
  \item If $A_1 \supseteq A_2 \supseteq A_3 \supseteq A_4 \cdots$
  are all finite, nonempty sets of real numbers,
  then the intersection $\bigcap^{\infty}_{n=1}A_n$ is finite and nonempty.
  \item $A \cap (B \cup C) = (A \cap B) \cup C$.
  \item $A \cap (B \cap C) = (A \cap B) \cap C$.
  \item $A \cap (B \cup C) = (A \cap B) \cup (A \cap C)$.
 \end{enumerate}
\end{prob}

\begin{prob}
 Produce an infinite collection of sets $A_1, A_2, A_3, \ldots$
 with the property that every $A_i$ has an infinite number of
 elements, $A_i \cap A_j = \emptyset$ for all $i \neq j$,
 and $\bigcup^{\infty}_{i=1} A_i = \mathbb{N}$.
\end{prob}


\begin{prob}[De Morgan's Laws]
 Let $A$ and $B$ be subsets of $\R$.
 \begin{enumerate}[(a)]
  \item If $x \in (A \cap B)^c$, explain why
  $x \in A^c \cup B^c$.  This shows that 
  $(A \cap B)^c \subseteq A^c \cup B^c$.
  \item Prove the reverse inclusion
  $(A \cap B)^c \supseteq A^c \cup B^c$, 
  and conclude that $(A \cap B)^c = A^c \cup B^c$.
  \item Prove that $(A \cup B)^c = A^c \cap B^c$.
 \end{enumerate}
\end{prob}

\vspace{5mm}

{\bf The following problem is optional.  It
will not contribute to or detract from your grade, but you are encouraged
to attempt it.}

\vspace{5mm}

\begin{chal}
\begin{enumerate}[(a)]
 \item Show how induction can be used to conclude that
 \[
 (A_1 \cup A_2 \cup \cdots \cup A_n)^c = A^c_1 \cap A^c_2 \cdots \cap A^c_n
 \]
 for any finite $n \in \mathbb{N}$.
 \item It is tempting to appeal to induction to conclude
 \[
\left(\bigcup^{\infty}_{i=1} A_i\right)^c = \bigcap^{\infty}_{i=1} A^c_i, 
 \]
 but induction does not apply here.  In general,
 induction can only prove that a particular statement
 holds for every value of $n \in \mathbb{N}$,
 but this does not necessarily imply the infinite case.
 To illustrate this point,
 find a collection of sets $B_1, B_2, B_3, \ldots$
 where the statement $\bigcap^n_{i=1} B_i \neq \emptyset$ is true for
 every $n \in \mathbb{N}$, but the statement $\bigcap^{\infty}_{i=1} B_i \neq \emptyset$
 fails.
 \item Nevertheless, the infinite version of De Morgan's Law
 stated in $(b)$ is a valid statement.  Provide a proof that
 does not use induction.
\end{enumerate}
\end{chal}



\vspace{5mm}

*All questions taken from \emph{Understanding Analysis: 2nd Edition} by Stephen Abbott.


\end{document}