
\documentclass{amsart}
%\usepackage{cmbright}
\usepackage{amsmath, amsfonts, amssymb, amscd}
%\usepackage{tableau}
\usepackage{color}
\usepackage{xcolor}
\usepackage{graphicx}
\usepackage{array}
\usepackage{mathtools}
\usepackage{multirow}
\usepackage{framed}
\usepackage{tikz}
\usetikzlibrary{matrix,arrows}
\usepackage[square,sort,comma,numbers]{natbib}
\usepackage{enumerate}




%\oddsidemargin.5cm
%\evensidemargin.5cm
\addtolength{\oddsidemargin}{-.525in}
\addtolength{\evensidemargin}{-.525in}
\addtolength{\textwidth}{1in}

\addtolength{\topmargin}{-.87in}
\addtolength{\textheight}{1.5in}




\newcommand\coolunder[2]{\mathrlap{\smash{\underbrace{\phantom{%
    \begin{matrix} #2 \end{matrix}}}_{\mbox{$#1$}}}}#2}

%\input{tableau}


\newcommand{\+}[1]{\ensuremath{\mathbf{#1}}}
\newcommand{\vect}[1]{\boldsymbol{#1}} % Uncomment for BOLD vectors.
%\newcommand{\vect}[1]{\vec{#1}} % Uncomment for ARROW vectors.
\newcommand{\C}{{\mathbb C}}
\newcommand{\Z}{{\mathbb Z}}
\renewcommand{\P}{{\mathbb P}}
\newcommand{\OG}{\operatorname{OG}}
\newcommand{\OF}{\operatorname{OF}}
\newcommand{\bull}{{\scriptscriptstyle \bullet}}
\newcommand{\la}{\lambda}
\newcommand{\euler}[1]{\chi_{_{#1}}}
\newcommand{\cO}{{\mathcal O}}
\newcommand{\cG}{{\mathcal G}}
\newcommand{\cQ}{{\mathcal Q}}
\newcommand{\R}{{\mathbb R}}
\newcommand{\wt}{\widetilde}
\newcommand{\diag}{\operatorname{diag}}
\newcommand{\comp}{\operatorname{comp}}
\newcommand{\comment}[1]{}
\newcommand{\type}{\mathfrak{t}}
\newcommand{\op}{\text{op}}
\newcommand{\row}{{\bf r}}
\newcommand{\col}{{\bf c}}
\newcommand{\sym}{\mathfrak{S}}
\newcommand{\codim}{\text{codim}}
\DeclarePairedDelimiter{\ceil}{\lceil}{\rceil}
\DeclarePairedDelimiter{\floor}{\lfloor}{\rfloor}
\renewcommand{\emptyset}{\varnothing}

\newtheorem{thm}{Theorem}
\newtheorem{lemma}[thm]{Lemma}
\newtheorem{prop}[thm]{Proposition}
\newtheorem{cor}[thm]{Corollary}

\theoremstyle{definition}
\newtheorem{definition}[thm]{Definition}
\newtheorem{example}[thm]{Example}
\newtheorem{conj}[thm]{Conjecture}
\newtheorem{obs}[thm]{Observation}
\newtheorem{fact}[thm]{Fact}
\newtheorem{remark}[thm]{Remark}
\newtheorem{prob}{Problem}
\newtheorem{chal}{Challenge}

\begin{document}
\title{Sample Proof}
\date{September 08, 2017}
\author{Intro to Real Analysis}

\maketitle



\begin{prob}
Prove the following statement:
If $(x_n)$ is bounded and diverges, then there exist two subsequences of $(x_n)$ that converge 
  to different limits.
\end{prob}

\begin{proof}
 Since $(x_n)$ is bounded, there exists a number $M$ such that $x_n \in [-M,M]$ for all natural numbers $n$.
 Moreover, by the Bolzano-Weierstrass Theorem, there exists a convergent
 subsequence $(x_{n_k})$ converging to some real number $L$.
 Increasing $M$ if necessary, we can assume $L \in (-M,M)$.
 
 Since $(x_n)$ diverges, there exists a real number $\epsilon > 0$ such that
 the set $\{x_n: x_n \not\in (L-\epsilon, L+\epsilon)\}$ is infinite.
 Shrinking $\epsilon$ if necessary, we can assume $(L-\epsilon, L+\epsilon) \subset [-M,M]$.
 Then, either $[-M, L-\epsilon]$ or $[L+\epsilon, M]$ contains infinitely many points of $(x_n)$.
 
 Without loss of generality, we can assume $[L+\epsilon, M]$ contains infinitely many points of $(x_n)$.
 Consider the subsequence of $(x_n)$ consisting of all points lying in $[L+\epsilon, M]$.
 By Bolzano-Weierstrass, it has a convergent subsequence $(x_{m_k})$ converging to some
 limit $L' \in [L+\epsilon, M]$.  Clearly $L' > L$, so we have produced
 subsequences converging to different limits, as desired.
\end{proof}



\end{document}