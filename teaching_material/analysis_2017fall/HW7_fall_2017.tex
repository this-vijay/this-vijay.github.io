
\documentclass{amsart}
%\usepackage{cmbright}
\usepackage{amsmath, amsfonts, amssymb, amscd}
%\usepackage{tableau}
\usepackage{color}
\usepackage{xcolor}
\usepackage{graphicx}
\usepackage{array}
\usepackage{mathtools}
\usepackage{multirow}
\usepackage{framed}
\usepackage{tikz}
\usetikzlibrary{matrix,arrows}
\usepackage[square,sort,comma,numbers]{natbib}
\usepackage{enumerate}




%\oddsidemargin.5cm
%\evensidemargin.5cm
\addtolength{\oddsidemargin}{-.525in}
\addtolength{\evensidemargin}{-.525in}
\addtolength{\textwidth}{1in}

\addtolength{\topmargin}{-.87in}
\addtolength{\textheight}{1.5in}




\newcommand\coolunder[2]{\mathrlap{\smash{\underbrace{\phantom{%
    \begin{matrix} #2 \end{matrix}}}_{\mbox{$#1$}}}}#2}

%\input{tableau}


\newcommand{\+}[1]{\ensuremath{\mathbf{#1}}}
\newcommand{\vect}[1]{\boldsymbol{#1}} % Uncomment for BOLD vectors.
%\newcommand{\vect}[1]{\vec{#1}} % Uncomment for ARROW vectors.
\newcommand{\C}{{\mathbb C}}
\newcommand{\Z}{{\mathbb Z}}
\renewcommand{\P}{{\mathbb P}}
\newcommand{\OG}{\operatorname{OG}}
\newcommand{\OF}{\operatorname{OF}}
\newcommand{\bull}{{\scriptscriptstyle \bullet}}
\newcommand{\la}{\lambda}
\newcommand{\euler}[1]{\chi_{_{#1}}}
\newcommand{\cO}{{\mathcal O}}
\newcommand{\cG}{{\mathcal G}}
\newcommand{\cQ}{{\mathcal Q}}
\newcommand{\R}{{\mathbb R}}
\newcommand{\wt}{\widetilde}
\newcommand{\diag}{\operatorname{diag}}
\newcommand{\comp}{\operatorname{comp}}
\newcommand{\comment}[1]{}
\newcommand{\type}{\mathfrak{t}}
\newcommand{\op}{\text{op}}
\newcommand{\row}{{\bf r}}
\newcommand{\col}{{\bf c}}
\newcommand{\sym}{\mathfrak{S}}
\newcommand{\codim}{\text{codim}}
\DeclarePairedDelimiter{\ceil}{\lceil}{\rceil}
\DeclarePairedDelimiter{\floor}{\lfloor}{\rfloor}
\renewcommand{\emptyset}{\varnothing}

\newtheorem{thm}{Theorem}
\newtheorem{lemma}[thm]{Lemma}
\newtheorem{prop}[thm]{Proposition}
\newtheorem{cor}[thm]{Corollary}

\theoremstyle{definition}
\newtheorem{definition}[thm]{Definition}
\newtheorem{example}[thm]{Example}
\newtheorem{conj}[thm]{Conjecture}
\newtheorem{obs}[thm]{Observation}
\newtheorem{fact}[thm]{Fact}
\newtheorem{remark}[thm]{Remark}
\newtheorem{prob}{Problem}
\newtheorem{chal}{Challenge}

\begin{document}
\title{Problem Set 7}
\date{October 09, 2017}
\author{Intro to Real Analysis}

\maketitle


\begin{prob}
 Recall that Thomae's function $t(x)$ is defined by
 \[
 t(x) =
 \begin{cases}
  1 &\text{ if } x = 0 \\
  1/n &\text{ if } x = m/n \in \mathbb{Q} \setminus \{0\} \text{ is in lowest terms with } n>0 \\
  0 &\text{ if } x \not\in \mathbb{Q}.
 \end{cases}
 \]
\begin{enumerate}[(a)]
 \item Construct three different sequences $(x_n)$, $(y_n)$, and $(z_n)$,
 each of which converges to $1$ without using the number $1$ as a term in the sequence.
 \item Compute $\lim t(x_n), \lim t(y_n)$, and $\lim t(z_n)$.
 \item Make a conjecture for the value of $\lim_{x \to 1} t(x)$, and prove it
 using the $\epsilon-\delta$ definition of functional convergence.
\end{enumerate}
 \end{prob}

 
 \begin{prob}
  We write $\lim_{x \to c} f(x) = \infty$ if for all $M > 0$ we can find a $\delta > 0$
  such that whenever $0 < |x-c| <\delta$, it follows that $f(x) > M$.
  \begin{enumerate}[(a)]
   \item Show that $\lim_{x \to 0} {1}/{x^2} = \infty$.
   \item Construct a definition for the statement $\lim_{x \to \infty} f(x) = L$.
   Show that $\lim_{x \to \infty} 1/x = 0$.
  \end{enumerate}
 \end{prob}

\begin{prob}
Prove the Squeeze Theorem.  That is,
 let $f$, $g$, and $h$ satisfy $f(x) \leq g(x) \leq h(x)$ for all $x$ in some common domain $A$.
 If $\lim_{x \to c} f(x) = \lim_{x \to c} h(x) = L$ at some limit point $c$ of $A$, show that
 $\lim_{x \to c} g(x) = L$ as well.
\end{prob}

\begin{prob}
 Prove the Contraction Mapping Theorem.  That is,
 let $f$ be a function defined on all of $\R$, and assume there is a constant
 $c$ such that $0 < c < 1$ and
 \[|f(x) - f(y)| \leq c|x-y|\]
 for all $x,y \in \R$.
 \begin{enumerate}
  \item Show that $f$ is continuous on $\R$.
  \item Pick some point $y_1 \in \R$ and construct a sequence
  \[(y_1, f(y_1), f(f(y_1)), \ldots).\]
  Writing $y_{n+1} = f(y_n)$, show that the resulting sequence
  $(y_n)$ is a Cauchy sequence.  Hence we can let $y = \lim y_n$.
  \item Prove that $y$ is a fixed point of $f$ and that it is unique
  in this regard.
  \item Finally, prove that if $x$ is \emph{any} arbitrary point in $\R$,
  then the sequence $(x, f(x), f(f(x)), \ldots)$ converges to $y$ defined in (b).
 \end{enumerate}
\end{prob}




\vspace{5mm}

{\bf The following problem is optional.  It
will not contribute to or detract from your grade, but you are encouraged
to attempt it.}

\vspace{5mm}

\begin{chal}
Let $F$ and $U$ be closed and open sets in $\R$ respectively. 
Construct functions $f$ and $g$ from $\R$ to $\R$ whose
sets of discontinuities are precisely $F$ and $U$ respectively.
For hints, see problem $4.3.14$ in the textbook.
\end{chal}


\vspace{5mm}

*All questions taken from \emph{Understanding Analysis: 2nd Edition} by Stephen Abbott.


\end{document}