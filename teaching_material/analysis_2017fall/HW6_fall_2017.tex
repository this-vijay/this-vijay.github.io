
\documentclass{amsart}
%\usepackage{cmbright}
\usepackage{amsmath, amsfonts, amssymb, amscd}
%\usepackage{tableau}
\usepackage{color}
\usepackage{xcolor}
\usepackage{graphicx}
\usepackage{array}
\usepackage{mathtools}
\usepackage{multirow}
\usepackage{framed}
\usepackage{tikz}
\usetikzlibrary{matrix,arrows}
\usepackage[square,sort,comma,numbers]{natbib}
\usepackage{enumerate}




%\oddsidemargin.5cm
%\evensidemargin.5cm
\addtolength{\oddsidemargin}{-.525in}
\addtolength{\evensidemargin}{-.525in}
\addtolength{\textwidth}{1in}

\addtolength{\topmargin}{-.87in}
\addtolength{\textheight}{1.5in}




\newcommand\coolunder[2]{\mathrlap{\smash{\underbrace{\phantom{%
    \begin{matrix} #2 \end{matrix}}}_{\mbox{$#1$}}}}#2}

%\input{tableau}


\newcommand{\+}[1]{\ensuremath{\mathbf{#1}}}
\newcommand{\vect}[1]{\boldsymbol{#1}} % Uncomment for BOLD vectors.
%\newcommand{\vect}[1]{\vec{#1}} % Uncomment for ARROW vectors.
\newcommand{\C}{{\mathbb C}}
\newcommand{\Z}{{\mathbb Z}}
\renewcommand{\P}{{\mathbb P}}
\newcommand{\OG}{\operatorname{OG}}
\newcommand{\OF}{\operatorname{OF}}
\newcommand{\bull}{{\scriptscriptstyle \bullet}}
\newcommand{\la}{\lambda}
\newcommand{\euler}[1]{\chi_{_{#1}}}
\newcommand{\cO}{{\mathcal O}}
\newcommand{\cG}{{\mathcal G}}
\newcommand{\cQ}{{\mathcal Q}}
\newcommand{\R}{{\mathbb R}}
\newcommand{\wt}{\widetilde}
\newcommand{\diag}{\operatorname{diag}}
\newcommand{\comp}{\operatorname{comp}}
\newcommand{\comment}[1]{}
\newcommand{\type}{\mathfrak{t}}
\newcommand{\op}{\text{op}}
\newcommand{\row}{{\bf r}}
\newcommand{\col}{{\bf c}}
\newcommand{\sym}{\mathfrak{S}}
\newcommand{\codim}{\text{codim}}
\DeclarePairedDelimiter{\ceil}{\lceil}{\rceil}
\DeclarePairedDelimiter{\floor}{\lfloor}{\rfloor}
\renewcommand{\emptyset}{\varnothing}

\newtheorem{thm}{Theorem}
\newtheorem{lemma}[thm]{Lemma}
\newtheorem{prop}[thm]{Proposition}
\newtheorem{cor}[thm]{Corollary}

\theoremstyle{definition}
\newtheorem{definition}[thm]{Definition}
\newtheorem{example}[thm]{Example}
\newtheorem{conj}[thm]{Conjecture}
\newtheorem{obs}[thm]{Observation}
\newtheorem{fact}[thm]{Fact}
\newtheorem{remark}[thm]{Remark}
\newtheorem{prob}{Problem}
\newtheorem{chal}{Challenge}

\begin{document}
\title{Problem Set 6}
\date{September 11, 2017}
\author{Intro to Real Analysis}

\maketitle


\begin{prob}
 Given $A \subset \R$, let $L$ be the set
 of all limit points of $A$.
 \begin{enumerate}[(a)]
  \item Show that the set $L$ is closed.
  \item Argue that if $x$ is a limit point
  of $A \cup L$, then $x$ is a limit point of $A$.
 \end{enumerate}
\end{prob}

\begin{prob}
 Prove that the only sets that are both open and closed
 are $\R$ and the empty set $\emptyset$.
\end{prob}

\begin{prob}
 Let $C$ be the Cantor set.  We will prove that the set 
 $C+C = \{x+y: x,y \in C\}$
 is equal to the closed interval $[0,2]$.
 Clearly $C+C \subset [0,2]$, so we
 only need to prove the reverse inclusion.
 Thus, given $s \in [0,2]$, we must find two
 elements $x,y \in C$ satisfying $x+y = s$.
 \begin{enumerate}[(a)]
  \item Show that there exist $x_1,y_1 \in C_1$
  for which $x_1 + y_1 = s$.
  Show in general that, for an arbitrary $n \in \mathbb{N}$,
  we can always find $x_n,y_n \in C_n$
  for which $x_n+y_n=s$.
  \item Although $(x_n)$ and $(y_n)$ may not
  converge, show that they can still be used to
  produce the desired $x$ and $y$ in $C$
  satisfying $x+y = s$.
 \end{enumerate}
\end{prob}

\begin{prob}
 Let $K$ and $L$ be nonempty compact sets, and define
 the \emph{distance} between $K$ and $L$ to be
 \[
 d = \text{inf}\{|x-y|:x \in K \text{ and } y \in L\}.
 \]
 \begin{enumerate}[(a)]
  \item If $K$ and $L$ are disjoint, show $d>0$
  and that $d = |x_0-y_0|$ for some $x_0 \in K$
  and $y_0 \in L$.
  \item Show that it's possible to have $d=0$
  if we only assume that the disjoint sets
  $K$ and $L$ are closed.
 \end{enumerate}
\end{prob}

\begin{prob}
 Finish the proof of the Heine-Borel Theorem from class.
 In particular, assume a set $K$ is closed and bounded
 (and hence that every sequence in $K$ has a subsequence
 that converges to a limit that is also in $K$).
 Prove that any open cover for $K$ has a finite subcover.
 (Two possible approaches are outlined in problems 
 3.3.9 and 3.3.10 of Abbott's book.)
\end{prob}



\vspace{5mm}

{\bf The following problem is optional.  It
will not contribute to or detract from your grade, but you are encouraged
to attempt it.}

\vspace{5mm}

\begin{chal}
 Repeat the Cantor set construction we did in class, starting
 with the interval $[0,1]$.  This time however,
 remove the open middle fourth from each component.
 \begin{enumerate}[(a)]
  \item Is the resulting set compact? Perfect?
  \item Using the methods of Section 3.1 of Abbott's book,
  compute the length and dimension of this
  Cantor-like set.
 \end{enumerate}
\end{chal}


\vspace{5mm}

*All questions taken from \emph{Understanding Analysis: 2nd Edition} by Stephen Abbott.


\end{document}