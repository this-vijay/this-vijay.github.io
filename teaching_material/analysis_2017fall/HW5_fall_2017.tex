
\documentclass{amsart}
%\usepackage{cmbright}
\usepackage{amsmath, amsfonts, amssymb, amscd}
%\usepackage{tableau}
\usepackage{color}
\usepackage{xcolor}
\usepackage{graphicx}
\usepackage{array}
\usepackage{mathtools}
\usepackage{multirow}
\usepackage{framed}
\usepackage{tikz}
\usetikzlibrary{matrix,arrows}
\usepackage[square,sort,comma,numbers]{natbib}
\usepackage{enumerate}




%\oddsidemargin.5cm
%\evensidemargin.5cm
\addtolength{\oddsidemargin}{-.525in}
\addtolength{\evensidemargin}{-.525in}
\addtolength{\textwidth}{1in}

\addtolength{\topmargin}{-.87in}
\addtolength{\textheight}{1.5in}




\newcommand\coolunder[2]{\mathrlap{\smash{\underbrace{\phantom{%
    \begin{matrix} #2 \end{matrix}}}_{\mbox{$#1$}}}}#2}

%\input{tableau}


\newcommand{\+}[1]{\ensuremath{\mathbf{#1}}}
\newcommand{\vect}[1]{\boldsymbol{#1}} % Uncomment for BOLD vectors.
%\newcommand{\vect}[1]{\vec{#1}} % Uncomment for ARROW vectors.
\newcommand{\C}{{\mathbb C}}
\newcommand{\Z}{{\mathbb Z}}
\renewcommand{\P}{{\mathbb P}}
\newcommand{\OG}{\operatorname{OG}}
\newcommand{\OF}{\operatorname{OF}}
\newcommand{\bull}{{\scriptscriptstyle \bullet}}
\newcommand{\la}{\lambda}
\newcommand{\euler}[1]{\chi_{_{#1}}}
\newcommand{\cO}{{\mathcal O}}
\newcommand{\cG}{{\mathcal G}}
\newcommand{\cQ}{{\mathcal Q}}
\newcommand{\R}{{\mathbb R}}
\newcommand{\wt}{\widetilde}
\newcommand{\diag}{\operatorname{diag}}
\newcommand{\comp}{\operatorname{comp}}
\newcommand{\comment}[1]{}
\newcommand{\type}{\mathfrak{t}}
\newcommand{\op}{\text{op}}
\newcommand{\row}{{\bf r}}
\newcommand{\col}{{\bf c}}
\newcommand{\sym}{\mathfrak{S}}
\newcommand{\codim}{\text{codim}}
\DeclarePairedDelimiter{\ceil}{\lceil}{\rceil}
\DeclarePairedDelimiter{\floor}{\lfloor}{\rfloor}
\renewcommand{\emptyset}{\varnothing}

\newtheorem{thm}{Theorem}
\newtheorem{lemma}[thm]{Lemma}
\newtheorem{prop}[thm]{Proposition}
\newtheorem{cor}[thm]{Corollary}

\theoremstyle{definition}
\newtheorem{definition}[thm]{Definition}
\newtheorem{example}[thm]{Example}
\newtheorem{conj}[thm]{Conjecture}
\newtheorem{obs}[thm]{Observation}
\newtheorem{fact}[thm]{Fact}
\newtheorem{remark}[thm]{Remark}
\newtheorem{prob}{Problem}
\newtheorem{chal}{Challenge}

\begin{document}
\title{Problem Set 5}
\date{September 04, 2017}
\author{Intro to Real Analysis}

\maketitle


\begin{prob}
 Let $(a_n)$ and $(b_n)$ be Cauchy sequences.  Decide whether
 each of the following sequences is a Cauchy sequence, justifying
 each conclusion:
 \begin{enumerate}[(a)]
  \item $c_n = |a_n - b_n|$
  \item $c_n = (-1)^{n}a_n$
  \item $c_n = [[a_n]]$, where $[[x]]$
  refers to the greatest integer less than or equal to $x$.
 \end{enumerate}
\end{prob}

\begin{prob}
 In the previous problem set, you established
 the equivalence of the Axiom of Completeness
 and the Monotone Convergence Theorem (make sure
 you understand why).  You also showed
 that the Nested Interval Property is
 equivalent to these other two in the presence
 of the Archimedean Property.
 \begin{enumerate}[(a)]
  \item Assume the Bolzano-Weierstrass Theorem is true,
  and use it to prove the Monotone Convergence Theorem
  without making any appeal to the Archimedean Property.
  \item Use the Cauchy Criterion to prove the 
  Bolzano-Weierstrass Theorem, and find the point
  in the argument where the Archimedean Property is
  implicitly required.
  \item How do we know it is impossible to prove
  the Axiom of Completeness starting from the Archimedean Property?
 \end{enumerate}
\end{prob}

\begin{prob}
 Let $(a_n)$ be a sequence satisfying $(a_n) \to 0$,
 and also $a_k \geq a_{k+1}$ for all $k \in \mathbb{N}$.
 The Alternating Series Test (AST) states that
 $\sum^{\infty}_{n=1}(-1)^{n+1}a_n$ converges.
 Let $s_n = a_1 - a_2 + a_3 - \cdots \pm a_n$.
 \begin{enumerate}[(a)]
  \item Prove AST by showing that
  $(s_n)$ is a Cauchy sequence.
  \item Give another proof of AST using the Nested Interval Property.
  \item Supply yet another proof of AST using the Monotone
  Convergence Theorem, and by considering the subsequences
  $(s_{2n})$ and $(s_{2n+1})$.
 \end{enumerate}
\end{prob}

\begin{prob}
 Use the Cauchy Condensation Test to prove that the series $\sum^{\infty}_{n=1} \frac{1}{n^p}$
 converges if and only if $p >1$.
\end{prob}

\begin{prob}
 Given a series $\sum^{\infty}_{n=1}a_n$ with $a_n \neq 0$,
 the Ratio Test states that if $(a_n)$ satisfies
 \[
 \text{lim}\left|\frac{a_{n+1}}{a_n}\right| = r < 1,
 \]
 then the series converges absolutely.
 \begin{enumerate}[(a)]
  \item Let $r'$ satisfy $r < r' < 1$.  Explain
  why there exists an $N$ such that $n \geq N$
  implies $|a_{n+1}| \leq |a_n|r'$.
  \item Why does $a_N \sum(r')^n$ converge?
  \item Show that $\sum |a_n|$ converges,
  and conclude that $\sum a_n$ converges. 
 \end{enumerate}
\end{prob}

\vspace{10mm}


*All questions taken from \emph{Understanding Analysis: 2nd Edition} by Stephen Abbott.


\end{document}