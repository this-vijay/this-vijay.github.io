
\documentclass{amsart}
%\usepackage{cmbright}
\usepackage{amsmath, amsfonts, amssymb, amscd}
%\usepackage{tableau}
\usepackage{color}
\usepackage{xcolor}
\usepackage{graphicx}
\usepackage{array}
\usepackage{mathtools}
\usepackage{multirow}
\usepackage{framed}
\usepackage{tikz}
\usetikzlibrary{matrix,arrows}
\usepackage[square,sort,comma,numbers]{natbib}
\usepackage{enumerate}




%\oddsidemargin.5cm
%\evensidemargin.5cm
\addtolength{\oddsidemargin}{-.525in}
\addtolength{\evensidemargin}{-.525in}
\addtolength{\textwidth}{1in}

\addtolength{\topmargin}{-.87in}
\addtolength{\textheight}{1.5in}




\newcommand\coolunder[2]{\mathrlap{\smash{\underbrace{\phantom{%
    \begin{matrix} #2 \end{matrix}}}_{\mbox{$#1$}}}}#2}

%\input{tableau}


\newcommand{\+}[1]{\ensuremath{\mathbf{#1}}}
\newcommand{\vect}[1]{\boldsymbol{#1}} % Uncomment for BOLD vectors.
%\newcommand{\vect}[1]{\vec{#1}} % Uncomment for ARROW vectors.
\newcommand{\C}{{\mathbb C}}
\newcommand{\Z}{{\mathbb Z}}
\renewcommand{\P}{{\mathbb P}}
\newcommand{\OG}{\operatorname{OG}}
\newcommand{\OF}{\operatorname{OF}}
\newcommand{\bull}{{\scriptscriptstyle \bullet}}
\newcommand{\la}{\lambda}
\newcommand{\euler}[1]{\chi_{_{#1}}}
\newcommand{\cO}{{\mathcal O}}
\newcommand{\cG}{{\mathcal G}}
\newcommand{\cQ}{{\mathcal Q}}
\newcommand{\R}{{\mathbb R}}
\newcommand{\wt}{\widetilde}
\newcommand{\diag}{\operatorname{diag}}
\newcommand{\comp}{\operatorname{comp}}
\newcommand{\comment}[1]{}
\newcommand{\type}{\mathfrak{t}}
\newcommand{\op}{\text{op}}
\newcommand{\row}{{\bf r}}
\newcommand{\col}{{\bf c}}
\newcommand{\sym}{\mathfrak{S}}
\newcommand{\codim}{\text{codim}}
\DeclarePairedDelimiter{\ceil}{\lceil}{\rceil}
\DeclarePairedDelimiter{\floor}{\lfloor}{\rfloor}
\renewcommand{\emptyset}{\varnothing}

\newtheorem{thm}{Theorem}
\newtheorem{lemma}[thm]{Lemma}
\newtheorem{prop}[thm]{Proposition}
\newtheorem{cor}[thm]{Corollary}

\theoremstyle{definition}
\newtheorem{definition}[thm]{Definition}
\newtheorem{example}[thm]{Example}
\newtheorem{conj}[thm]{Conjecture}
\newtheorem{obs}[thm]{Observation}
\newtheorem{fact}[thm]{Fact}
\newtheorem{remark}[thm]{Remark}
\newtheorem{prob}{Problem}
\newtheorem{chal}{Challenge}

\begin{document}
\title{Problem Set 8}
\date{October 16, 2017}
\author{Intro to Real Analysis}

\maketitle

\begin{prob}
 Assume $f$ and $g$ are defined on all of $\R$
 and that $\lim_{x \to p} f(x) = q$
 and $\lim_{x \to q} g(x) = r$.
 \begin{enumerate}[(a)]
  \item Give an example to show that it may not be the case that
  \[\lim_{x \to p}g(f(x)) = r.\]
\item Show that the result in (a)
does follow if we assume $f$ and $g$
are continuous.
\item Does the result in (a)
hold if we only assume $f$ is continuous?
What if we only assume $g$ is continuous?
  \end{enumerate}
\end{prob}


\begin{prob}
Show whether or not the following functions are
\emph{uniformly} continuous on the interval $(0,1)$.
\begin{enumerate}[(a)]
 \item $f(x) = 1/x$.
 \item $g(x) = \sqrt{x^2 +1}$.
 \item $h(x) = x \sin(1/x)$.
\end{enumerate}
 \end{prob}
 

 \begin{prob}
  A function $f: A \to \R$ is called \emph{Lipschitz}
  if there exists a bound $M>0$ such that
  \[
  \left| \frac{f(x) - f(y)}{x-y} \right| \leq M
  \]
  for all $x \neq y \in A$.
  In other words, there is a uniform bound
  on the magnitude of the slopes of lines drawn
  through any two points on the graph of $f$.
  \begin{enumerate}[(a)]
   \item Show that if $f:A \to \R$
   is Lipschitz, then it is uniformly continuous on $A$.
   \item Does the converse hold? I.e. are uniformly continuous functions necessarily Lipschitz?
  \end{enumerate}
 \end{prob}

%\begin{prob}
% Let $g$ be continuous on an interval $A$ and let $F$ be the set of 
% points where $g$ fails to be one-to-one.  In other words,
% \[
% F = \{x \in A: f(x) = f(y) \text{ for some } y \neq x \text{ and } y \in A\}.
% \]
% Show that $F$ is either empty or uncountable.
%\end{prob}

 
\begin{prob}
Let $f$ be a continous one-to-one function
from an interval $A$ to $\R$.
\begin{enumerate}[(a)]
\item Show that $f$ is monotone.
\item Show that $f^{-1}$ is continuous.
\end{enumerate}
\end{prob}






\vspace{5mm}

{\bf The following problem is optional.  It
will not contribute to or detract from your grade, but you are encouraged
to attempt it.}

\vspace{5mm}

\begin{chal}
\begin{enumerate}[(a)]
\item Let $g$ be defined on all of $\R$.
Show that $g$ is continuous if and only if 
$g^{-1}(U)$ is open whenever $U \subset \R$ is open.
\item Let $f$ be a continuous function defined on all of $\R$.
Suppose $B$ is a set with property $\spadesuit \in \{\text{finite, compact, bounded, closed}\}$.
Does $g^{-1}(B)$ have property $\spadesuit$?
\end{enumerate}
\end{chal}


\vspace{5mm}

*All questions taken from \emph{Understanding Analysis: 2nd Edition} by Stephen Abbott.


\end{document}