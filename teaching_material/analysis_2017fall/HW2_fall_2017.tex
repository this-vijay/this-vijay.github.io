
\documentclass{amsart}
%\usepackage{cmbright}
\usepackage{amsmath, amsfonts, amssymb, amscd}
%\usepackage{tableau}
\usepackage{color}
\usepackage{xcolor}
\usepackage{graphicx}
\usepackage{array}
\usepackage{mathtools}
\usepackage{multirow}
\usepackage{framed}
\usepackage{tikz}
\usetikzlibrary{matrix,arrows}
\usepackage[square,sort,comma,numbers]{natbib}
\usepackage{enumerate}




%\oddsidemargin.5cm
%\evensidemargin.5cm
\addtolength{\oddsidemargin}{-.525in}
\addtolength{\evensidemargin}{-.525in}
\addtolength{\textwidth}{1in}

\addtolength{\topmargin}{-.87in}
\addtolength{\textheight}{1.5in}




\newcommand\coolunder[2]{\mathrlap{\smash{\underbrace{\phantom{%
    \begin{matrix} #2 \end{matrix}}}_{\mbox{$#1$}}}}#2}

%\input{tableau}


\newcommand{\+}[1]{\ensuremath{\mathbf{#1}}}
\newcommand{\vect}[1]{\boldsymbol{#1}} % Uncomment for BOLD vectors.
%\newcommand{\vect}[1]{\vec{#1}} % Uncomment for ARROW vectors.
\newcommand{\C}{{\mathbb C}}
\newcommand{\Z}{{\mathbb Z}}
\renewcommand{\P}{{\mathbb P}}
\newcommand{\OG}{\operatorname{OG}}
\newcommand{\OF}{\operatorname{OF}}
\newcommand{\bull}{{\scriptscriptstyle \bullet}}
\newcommand{\la}{\lambda}
\newcommand{\euler}[1]{\chi_{_{#1}}}
\newcommand{\cO}{{\mathcal O}}
\newcommand{\cG}{{\mathcal G}}
\newcommand{\cQ}{{\mathcal Q}}
\newcommand{\R}{{\mathbb R}}
\newcommand{\wt}{\widetilde}
\newcommand{\diag}{\operatorname{diag}}
\newcommand{\comp}{\operatorname{comp}}
\newcommand{\comment}[1]{}
\newcommand{\type}{\mathfrak{t}}
\newcommand{\op}{\text{op}}
\newcommand{\row}{{\bf r}}
\newcommand{\col}{{\bf c}}
\newcommand{\sym}{\mathfrak{S}}
\newcommand{\codim}{\text{codim}}
\DeclarePairedDelimiter{\ceil}{\lceil}{\rceil}
\DeclarePairedDelimiter{\floor}{\lfloor}{\rfloor}
\renewcommand{\emptyset}{\varnothing}

\newtheorem{thm}{Theorem}
\newtheorem{lemma}[thm]{Lemma}
\newtheorem{prop}[thm]{Proposition}
\newtheorem{cor}[thm]{Corollary}

\theoremstyle{definition}
\newtheorem{definition}[thm]{Definition}
\newtheorem{example}[thm]{Example}
\newtheorem{conj}[thm]{Conjecture}
\newtheorem{obs}[thm]{Observation}
\newtheorem{fact}[thm]{Fact}
\newtheorem{remark}[thm]{Remark}
\newtheorem{prob}{Problem}
\newtheorem{chal}{Challenge}

\begin{document}
\title{Problem Set 2}
\date{August 14, 2017}
\author{Intro to Real Analysis}

\maketitle


\begin{prob}
Let $A \subset \R$ be nonempty and bounded below.
Prove that $A$ has a greatest lower bound.
(Hint: Define $B = \{b \in \R: b \text{ is a lower
bound for } A\}$ and show that sup $B =$ inf $A$.)
\end{prob}

\begin{prob}
 Let $A_1, A_2, A_3, \ldots$ be a collection of 
 nonempty sets, each of which is bounded above.
 \begin{enumerate}[(a)]
  \item Find a formula for sup$(A_1 \cup A_2)$.
  Extend this to sup$(\bigcup^n_{k=1} A_k)$.
  \item Consider sup$(\bigcup^\infty_{k=1} A_k)$.
  Does the formula in (a) extend to the infinite case?
 \end{enumerate}
\end{prob}

\begin{prob}
 Compute (without proofs) the suprema and infima (if they exist)
 of the following sets:
 \begin{enumerate}[(a)]
  \item $\{\frac{m}{n}: m,n \in \mathbb{N} \text{ with } m < n\}$.
    \vspace{2mm}
  \item $\{\frac{(-1)^m}{n}: m,n \in \mathbb{N}\}$.
  \vspace{2mm}
  \item $\{\frac{n}{3n+1}: n \in \mathbb{N}\}$.
    \vspace{2mm}
  \item $\{\frac{m}{m+n}: m,n \in \mathbb{N}\}$.
 \end{enumerate}
\end{prob}

\begin{prob}
 \begin{enumerate}[(a)]
  \item If sup $A < $ sup $B$, show that there exists
  an element $b \in B$ that is an upper bound for $A$.
  \item Give an example to show that this is not the case
  if we only assume sup $A \leq$ sup $B$.
 \end{enumerate}
\end{prob}

\begin{prob}
 Let $A \subset \R$ be nonempty and bounded above, and
 let $s \in \R$ have the property that for all 
 $n \in \mathbb{N}, s + \frac{1}{n}$ is an upper bound
 for $A$ and $s - \frac{1}{n}$ is not
 an upper bound for $A$.  Show that $s = $ sup $A$.
\end{prob}

\begin{prob}
 Let $a < b$ be real numbers and consider the set 
 $T = \mathbb{Q} \cap [a,b]$.
 Show that sup $T = b$.
\end{prob}




\vspace{5mm}

{\bf The following problem is optional.  It
will not contribute to or detract from your grade, but you are encouraged
to attempt it.}

\vspace{5mm}

\begin{chal}
The \emph{Cut Property} of the real numbers is the following:
If $A$ and $B$ are nonempty, disjoint sets with
$A \cup B = \R$ and $a < b$ for all $a \in A$ and $b \in B$,
then there exists $c \in \R$ such that $x \leq c$ whenever
$x \in A$ and $x \geq c$ whenever $x \in B$.
\begin{enumerate}[(a)]
 \item Prove that the Cut Property is equivalent
 to the Axiom of Completeness (i.e. to the statement
 that any set bounded above has a least upper bound).
 \item Give an example showing that the Cut Property does
 not hold when $\R$ is replaced by $\mathbb{Q}$.
\end{enumerate}
\end{chal}



\vspace{5mm}

*All questions taken from \emph{Understanding Analysis: 2nd Edition} by Stephen Abbott.


\end{document}