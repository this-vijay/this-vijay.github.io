
\documentclass{amsart}
%\usepackage{cmbright}
\usepackage{amsmath, amsfonts, amssymb, amscd}
%\usepackage{tableau}
\usepackage{color}
\usepackage{xcolor}
\usepackage{graphicx}
\usepackage{array}
\usepackage{mathtools}
\usepackage{multirow}
\usepackage{framed}
\usepackage{tikz}
\usetikzlibrary{matrix,arrows}
\usepackage[square,sort,comma,numbers]{natbib}
\usepackage{enumerate}




%\oddsidemargin.5cm
%\evensidemargin.5cm
\addtolength{\oddsidemargin}{-.525in}
\addtolength{\evensidemargin}{-.525in}
\addtolength{\textwidth}{1in}

\addtolength{\topmargin}{-.87in}
\addtolength{\textheight}{1.5in}




\newcommand\coolunder[2]{\mathrlap{\smash{\underbrace{\phantom{%
    \begin{matrix} #2 \end{matrix}}}_{\mbox{$#1$}}}}#2}

%\input{tableau}


\newcommand{\+}[1]{\ensuremath{\mathbf{#1}}}
\newcommand{\vect}[1]{\boldsymbol{#1}} % Uncomment for BOLD vectors.
%\newcommand{\vect}[1]{\vec{#1}} % Uncomment for ARROW vectors.
\newcommand{\C}{{\mathbb C}}
\newcommand{\Z}{{\mathbb Z}}
\renewcommand{\P}{{\mathbb P}}
\newcommand{\OG}{\operatorname{OG}}
\newcommand{\OF}{\operatorname{OF}}
\newcommand{\bull}{{\scriptscriptstyle \bullet}}
\newcommand{\la}{\lambda}
\newcommand{\euler}[1]{\chi_{_{#1}}}
\newcommand{\cO}{{\mathcal O}}
\newcommand{\cG}{{\mathcal G}}
\newcommand{\cQ}{{\mathcal Q}}
\newcommand{\R}{{\mathbb R}}
\newcommand{\wt}{\widetilde}
\newcommand{\diag}{\operatorname{diag}}
\newcommand{\comp}{\operatorname{comp}}
\newcommand{\comment}[1]{}
\newcommand{\type}{\mathfrak{t}}
\newcommand{\op}{\text{op}}
\newcommand{\row}{{\bf r}}
\newcommand{\col}{{\bf c}}
\newcommand{\sym}{\mathfrak{S}}
\newcommand{\codim}{\text{codim}}
\DeclarePairedDelimiter{\ceil}{\lceil}{\rceil}
\DeclarePairedDelimiter{\floor}{\lfloor}{\rfloor}
\renewcommand{\emptyset}{\varnothing}

\newtheorem{thm}{Theorem}
\newtheorem{lemma}[thm]{Lemma}
\newtheorem{prop}[thm]{Proposition}
\newtheorem{cor}[thm]{Corollary}

\theoremstyle{definition}
\newtheorem{definition}[thm]{Definition}
\newtheorem{example}[thm]{Example}
\newtheorem{conj}[thm]{Conjecture}
\newtheorem{obs}[thm]{Observation}
\newtheorem{fact}[thm]{Fact}
\newtheorem{remark}[thm]{Remark}
\newtheorem{prob}{Problem}
\newtheorem{chal}{Challenge}

\begin{document}
\title{Schubert Polynomials Day 4}
\date{October 28, 2017}
\author{ATMW Schubert Varieties 2017}

\maketitle

\begin{prob}[braid relation of divided difference operators]
 Prove that $\partial_1 \partial{2} \partial_1 = \partial_{2} \partial{1} \partial_{2}$
 holds for an arbitrary monomial.  The twisted Leibniz rule ($\partial_i(PQ) = (\partial_i P)Q + s_i(P)\partial_i(Q)$) may be helpful.
\end{prob}


\begin{prob}
 Suppose $w$ and $v$ are permutations with the same length.
 Show that $\partial_v(S_w)$ is $1$ if $w=v$ and $0$ otherwise.
\end{prob}

\begin{prob}
 Show that the Schubert polynomials are linearly independent.  The previous exercise will
 be useful.
\end{prob}



\begin{prob}
 Let $w = 2431$.  Compute $S_w$ using divided difference operators.
\end{prob}

\begin{prob}
Compute $S_w$ by melting Schubert's sweater:
\begin{enumerate}[(a)]
 \item $w = 4132$ (this is easy, if you modify the example done on the board!)
 \item $w = s_3 = 1243$ (beware of non-reduced diagrams!)
 \item $1432$ (but only if you are enjoying this!)
 \item $2413$ (since Grassmannian permutations are good for health!)
\end{enumerate}
\end{prob}

\begin{prob}
 Prove Stanley's formula for $S_w$, which expresses
 it as the sum of monomials $x_{b_1}\ldots x_{b_l}$ over all words $b$ compatible with reduced words $a$ for $w$.
 In this proof, you can assume that Fomin's technique of melting Schubert's sweater is valid.
\end{prob}

\begin{prob}
Let $w$ be a $k$-Grassmannian permutation. Find a bijection between SSYT on $\lambda(w)$ in the
alphabet $\{1, \ldots, k\}$
on the one hand,
and melted sweaters associated to $w$ on the other hand.
\end{prob}


\end{document}