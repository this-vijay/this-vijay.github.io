
\documentclass{amsart}
%\usepackage{cmbright}
\usepackage{amsmath, amsfonts, amssymb, amscd}
%\usepackage{tableau}
\usepackage{color}
\usepackage{xcolor}
\usepackage{graphicx}
\usepackage{array}
\usepackage{mathtools}
\usepackage{multirow}
\usepackage{framed}
\usepackage{tikz}
\usetikzlibrary{matrix,arrows}
\usepackage[square,sort,comma,numbers]{natbib}
\usepackage{enumerate}




%\oddsidemargin.5cm
%\evensidemargin.5cm
\addtolength{\oddsidemargin}{-.525in}
\addtolength{\evensidemargin}{-.525in}
\addtolength{\textwidth}{1in}

\addtolength{\topmargin}{-.87in}
\addtolength{\textheight}{1.5in}




\newcommand\coolunder[2]{\mathrlap{\smash{\underbrace{\phantom{%
    \begin{matrix} #2 \end{matrix}}}_{\mbox{$#1$}}}}#2}

%\input{tableau}


\newcommand{\+}[1]{\ensuremath{\mathbf{#1}}}
\newcommand{\vect}[1]{\boldsymbol{#1}} % Uncomment for BOLD vectors.
%\newcommand{\vect}[1]{\vec{#1}} % Uncomment for ARROW vectors.
\newcommand{\C}{{\mathbb C}}
\newcommand{\Z}{{\mathbb Z}}
\renewcommand{\P}{{\mathbb P}}
\newcommand{\OG}{\operatorname{OG}}
\newcommand{\OF}{\operatorname{OF}}
\newcommand{\bull}{{\scriptscriptstyle \bullet}}
\newcommand{\la}{\lambda}
\newcommand{\euler}[1]{\chi_{_{#1}}}
\newcommand{\cO}{{\mathcal O}}
\newcommand{\cG}{{\mathcal G}}
\newcommand{\cQ}{{\mathcal Q}}
\newcommand{\R}{{\mathbb R}}
\newcommand{\wt}{\widetilde}
\newcommand{\diag}{\operatorname{diag}}
\newcommand{\comp}{\operatorname{comp}}
\newcommand{\comment}[1]{}
\newcommand{\type}{\mathfrak{t}}
\newcommand{\op}{\text{op}}
\newcommand{\row}{{\bf r}}
\newcommand{\col}{{\bf c}}
\newcommand{\sym}{\mathfrak{S}}
\newcommand{\codim}{\text{codim}}
\DeclarePairedDelimiter{\ceil}{\lceil}{\rceil}
\DeclarePairedDelimiter{\floor}{\lfloor}{\rfloor}
\renewcommand{\emptyset}{\varnothing}

\newtheorem{thm}{Theorem}
\newtheorem{lemma}[thm]{Lemma}
\newtheorem{prop}[thm]{Proposition}
\newtheorem{cor}[thm]{Corollary}

\theoremstyle{definition}
\newtheorem{definition}[thm]{Definition}
\newtheorem{example}[thm]{Example}
\newtheorem{conj}[thm]{Conjecture}
\newtheorem{obs}[thm]{Observation}
\newtheorem{fact}[thm]{Fact}
\newtheorem{remark}[thm]{Remark}
\newtheorem{prob}{Problem}
\newtheorem{chal}{Challenge}

\begin{document}
\title{Schubert Polynomials Day 1}
\date{October 23, 2017}
\author{ATMW Schubert Varieties 2017}

\maketitle

\begin{prob}
For each permutation $w \in \{ 4612375, 6174235, 6324571\} \subset S_7$,
\begin{enumerate}[(a)]
 \item Compute the inversion set $I(w)$, draw the Rothe diagram $D(w)$,
 write the Lehmer code $c(w)$, draw the shape $\lambda(w)$,
 and mark the values of the rank function on the essential set Ess$(w)$.
 \item Verify that the permutation $w$ can be reconstructed from its Lehmer code,
 as well as from the values of the rank function on the essential set.
 \item Give a reduced decomposition for $w$.
\end{enumerate}
\end{prob}


\begin{prob}
\begin{enumerate}[(a)]
\item Give an algorithm for producing a reduced decomposition for a permutation $w \in S_n$, and
show that your algorithm does indeed produce a decomposition whose length is
equal to $l(w)$ (i.e. the cardinality of the inversion set $I(w)$).
Does your algorithm agree with either of the 'magic tricks' described in lecture?
\item Show that any decomposition for $w$ must have length at least $l(w)$.
Hence the shortest possible decompositions for $w$ have
length exactly equal to $l(w)$.
\end{enumerate}
\end{prob}

\begin{prob}
 Suppose $v \leq w$ in $S_n$, and let $v = s_{j_1}\ldots s_{j_m}$ be a
 reduced decomposition.  Can we always
 find a reduced decomposition for $w = s_{i_1} \ldots s_{i_l}$
 such that $(j_1, \ldots, j_m)$ is a subsequence of $(i_1, \ldots, i_l)$?
\end{prob}

\begin{prob}
 We define the \emph{left} weak and strong Bruhat orders
 by using left multiplication to define our covering relations.
 In particular we say $v$ precedes $w$ in the left weak (respectively strong) Bruhat order 
 if $l(v)+1 = l(w)$ and $s_i v = w$ for some simple
 transposition $s_i$ (respectively $t_{ij}v = w$ for some transposition $t_{ij}$).
 \begin{enumerate}[(a)]
  \item Show (without using the subword property) that the left strong Bruhat order coincides with the right strong Bruhat order
  defined in lecture.
  \item Does the left weak Bruhat order coincide with the right weak Bruhat order?
 \end{enumerate}
\end{prob}

\begin{prob}[Deletion Property]
 Suppose $s_{i_1} \cdots s_{i_m}$ is a nonreduced decomposition of a permutation $w$.
 Show that there exist integers $p < q$ such that
 $w = s_{i_1} \cdots \hat{s_{i_p}} \cdots \hat{s_{i_q}} \cdots s_{i_l}$.
 Here you may want to use the exchange lemma from lecture.
\end{prob}



\end{document}