
\documentclass{amsart}
%\usepackage{cmbright}
\usepackage{amsmath, amsfonts, amssymb, amscd}
%\usepackage{tableau}
\usepackage{color}
\usepackage{xcolor}
\usepackage{graphicx}
\usepackage{array}
\usepackage{mathtools}
\usepackage{multirow}
\usepackage{framed}
\usepackage{tikz}
\usetikzlibrary{matrix,arrows}
\usepackage[square,sort,comma,numbers]{natbib}
\usepackage{enumerate}




%\oddsidemargin.5cm
%\evensidemargin.5cm
\addtolength{\oddsidemargin}{-.525in}
\addtolength{\evensidemargin}{-.525in}
\addtolength{\textwidth}{1in}

\addtolength{\topmargin}{-.87in}
\addtolength{\textheight}{1.5in}




\newcommand\coolunder[2]{\mathrlap{\smash{\underbrace{\phantom{%
    \begin{matrix} #2 \end{matrix}}}_{\mbox{$#1$}}}}#2}

%\input{tableau}


\newcommand{\+}[1]{\ensuremath{\mathbf{#1}}}
\newcommand{\vect}[1]{\boldsymbol{#1}} % Uncomment for BOLD vectors.
%\newcommand{\vect}[1]{\vec{#1}} % Uncomment for ARROW vectors.
\newcommand{\C}{{\mathbb C}}
\newcommand{\Z}{{\mathbb Z}}
\renewcommand{\P}{{\mathbb P}}
\newcommand{\OG}{\operatorname{OG}}
\newcommand{\OF}{\operatorname{OF}}
\newcommand{\bull}{{\scriptscriptstyle \bullet}}
\newcommand{\la}{\lambda}
\newcommand{\euler}[1]{\chi_{_{#1}}}
\newcommand{\cO}{{\mathcal O}}
\newcommand{\cG}{{\mathcal G}}
\newcommand{\cQ}{{\mathcal Q}}
\newcommand{\R}{{\mathbb R}}
\newcommand{\wt}{\widetilde}
\newcommand{\diag}{\operatorname{diag}}
\newcommand{\comp}{\operatorname{comp}}
\newcommand{\comment}[1]{}
\newcommand{\type}{\mathfrak{t}}
\newcommand{\op}{\text{op}}
\newcommand{\row}{{\bf r}}
\newcommand{\col}{{\bf c}}
\newcommand{\sym}{\mathfrak{S}}
\newcommand{\codim}{\text{codim}}
\DeclarePairedDelimiter{\ceil}{\lceil}{\rceil}
\DeclarePairedDelimiter{\floor}{\lfloor}{\rfloor}
\renewcommand{\emptyset}{\varnothing}

\newtheorem{thm}{Theorem}
\newtheorem{lemma}[thm]{Lemma}
\newtheorem{prop}[thm]{Proposition}
\newtheorem{cor}[thm]{Corollary}

\theoremstyle{definition}
\newtheorem{definition}[thm]{Definition}
\newtheorem{example}[thm]{Example}
\newtheorem{conj}[thm]{Conjecture}
\newtheorem{obs}[thm]{Observation}
\newtheorem{fact}[thm]{Fact}
\newtheorem{remark}[thm]{Remark}
\newtheorem{prob}{Problem}
\newtheorem{chal}{Challenge}

\begin{document}
\title{Schubert Polynomials Day 3}
\date{October 25, 2017}
\author{ATMW Schubert Varieties 2017}

\maketitle

\begin{prob}[Wiring Diagrams for Words]
\begin{enumerate}[(a)]
\item Show that a decomposition for a permutation $w$ is reduced if and only if no two wires cross twice in its wiring diagram.
\item Suppose no two wires cross twice in a wiring diagram for $w$.
Show that wires $i$ and $j$ in the wiring diagram cross if and only if $(i,j) \in I(w)$.
\item Can you use this result to give a quick proof of the Deletion Property? What about the Strong Exchange Property?
\end{enumerate}
\end{prob}


\begin{prob}[Wiring Diagrams and Connectedness of Reduced Word Graph]
Let $w = s_{i_1} \cdots s_{i_l}$ be reduced, and let $m = \min\{i_1,\ldots,i_l\}$.
Check what happens to the wiring diagram as you
follow the algorithm from class for pushing $s_m$ to the right as far as possible.
\end{prob}

\begin{prob}[Basic properties of divided difference operators]
 \begin{enumerate}[(a)]
  \item  Show that for any polynomial $P \in \mathbb{Z}[x_1,\ldots,x_n]$, we have that
 $(x_i - x_{i+1})$ divides $P - s_i(P)$.
Thus $\partial_i(P)$ is a well-defined polynomial for all $1 \leq i \leq n-1$.
\item Show that if $P$ is homogeneous of degree $d$, then $\partial_i(P)$ is 
either homegeneous of degree $d-1$, or zero.
\item Show that $\partial_i(PQ) = (\partial_iP)Q +(s_iP)(\partial_iQ)$ for polynomials $P$ and $Q$.
Thus $\partial_i$ behaves a bit like a derivative.
\item Show that if $\partial_i(P) = 0$, then $P$ is symmetric in $x_i$ and $x_{i+1}$.
 \end{enumerate}
\end{prob}



\begin{prob}[Playing with the operators]
 Let $P = x^2 y \in \Z[x,y,z]$.
 Find all polynomials we can generate using the operators $\partial_1$ and $\partial_2$.
 You can save some time by invoking the results of the next problem below!
\end{prob}

\begin{prob}[Relations satisfied by the operators]
Prove the following facts about the operators $\partial_i$ on $\mathbb{Z}[x_1,\ldots,x_n]$.
 \begin{enumerate}[(a)]
  \item $\partial_i^2 = 0$ ,
  \item $\partial_i \partial_j = \partial_j \partial_i$ if $|i-j| > 1$, and
  \item $\partial_i \partial_j  \partial_i = \partial_j \partial_i  \partial_j$ if $|i-j| =1$.
 \end{enumerate}
\end{prob}




\end{document}