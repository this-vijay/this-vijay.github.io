
\documentclass{amsart}
%\usepackage{cmbright}
\usepackage{amsmath, amsfonts, amssymb, amscd}
%\usepackage{tableau}
\usepackage{color}
\usepackage{xcolor}
\usepackage{graphicx}
\usepackage{array}
\usepackage{mathtools}
\usepackage{multirow}
\usepackage{framed}
\usepackage{tikz}
\usetikzlibrary{matrix,arrows}
\usepackage[square,sort,comma,numbers]{natbib}
\usepackage{enumerate}
\usepackage{comment}



%\oddsidemargin.5cm
%\evensidemargin.5cm
\addtolength{\oddsidemargin}{-.525in}
\addtolength{\evensidemargin}{-.525in}
\addtolength{\textwidth}{1in}

\addtolength{\topmargin}{-.87in}
\addtolength{\textheight}{1.5in}




\newcommand\coolunder[2]{\mathrlap{\smash{\underbrace{\phantom{%
    \begin{matrix} #2 \end{matrix}}}_{\mbox{$#1$}}}}#2}

%\input{tableau}


\newcommand{\+}[1]{\ensuremath{\mathbf{#1}}}
\newcommand{\vect}[1]{\boldsymbol{#1}} % Uncomment for BOLD vectors.
%\newcommand{\vect}[1]{\vec{#1}} % Uncomment for ARROW vectors.
\newcommand{\C}{{\mathbb C}}
\newcommand{\Z}{{\mathbb Z}}
\renewcommand{\P}{{\mathbb P}}
\newcommand{\OG}{\operatorname{OG}}
\newcommand{\OF}{\operatorname{OF}}
\newcommand{\bull}{{\scriptscriptstyle \bullet}}
\newcommand{\la}{\lambda}
\newcommand{\euler}[1]{\chi_{_{#1}}}
\newcommand{\cO}{{\mathcal O}}
\newcommand{\cG}{{\mathcal G}}
\newcommand{\cQ}{{\mathcal Q}}
\newcommand{\R}{{\mathbb R}}
\newcommand{\wt}{\widetilde}
\newcommand{\diag}{\operatorname{diag}}
\newcommand{\comp}{\operatorname{comp}}
\newcommand{\type}{\mathfrak{t}}
\newcommand{\op}{\text{op}}
\newcommand{\row}{{\bf r}}
\newcommand{\col}{{\bf c}}
\newcommand{\sym}{\mathfrak{S}}
\newcommand{\codim}{\text{codim}}
\DeclarePairedDelimiter{\ceil}{\lceil}{\rceil}
\DeclarePairedDelimiter{\floor}{\lfloor}{\rfloor}
\renewcommand{\emptyset}{\varnothing}

\newtheorem{thm}{Theorem}
\newtheorem{lemma}[thm]{Lemma}
\newtheorem{prop}[thm]{Proposition}
\newtheorem{cor}[thm]{Corollary}

\theoremstyle{definition}
\newtheorem{definition}[thm]{Definition}
\newtheorem{example}[thm]{Example}
\newtheorem{conj}[thm]{Conjecture}
\newtheorem{obs}[thm]{Observation}
\newtheorem{fact}[thm]{Fact}
\newtheorem{remark}[thm]{Remark}
\newtheorem{prob}{Problem}
\newtheorem{chal}{Challenge}

\begin{document}
\title{Midterm Exam}
\date{September 22, 2016}
\author{Introduction to Manifolds}

\maketitle


The exam consists of six questions  worth a total of $100$ points.
You may take up to three hours to complete the exam.

\vspace{3mm}

\begin{comment}
\begin{prob}[$15$ pts]
Let $I$ be the $3 \times 3$ identity matrix, and let $A = \begin{bmatrix}
3 & 2 & 6 \\ 2 & 2 & 5 \\ -2 & -1 & -4 
\end{bmatrix}$.
Let $\Gamma_I, \Gamma_A \subset \R^6$ denote the graphs of the corresponding linear transformations from
$\R^3$ to $\R^3$.
Do $\Gamma_I$ and $\Gamma_A$ have transverse intersection?
Justify your answer using the definition of transverse intersection: namely that
\[\Gamma_I \pitchfork \Gamma_A \Longleftrightarrow T_p\Gamma_I + T_p\Gamma_A = T_p\R^6 \text{ for all } p \in \Gamma_I \cap \Gamma_A.\]
\end{prob}

\vspace{3mm}
\end{comment}

\begin{comment}
\begin{prob}[$15$ pts]
Let $M$ be a \emph{compact} smooth manifold.  Show that there is no smooth submersion $F: M \to \R^k$ for any $k >0$.
\end{prob}

\vspace{3mm}
\end{comment}





%\begin{prob}[$10$ pts]
% Let and $f^1$ and $f^2$ be smooth functions from $\R^2 \to \R$, and suppose
% $Z(f^1)$ and $Z(f^2)$ intersect transversely at a point $p \in \R^2$.
% Prove that $F := (f^1,f^2): \R^2 \to \R^2$ gives local coordinates in some neighborhood
% of $p$.  (In other words, there exists some neighborhood of $p$ on which
% the restriction of $F$ is a local diffeomorphism.)
%\end{prob}

\begin{comment}
\begin{prob}[$15$ pts]
Let $X = y^2 \frac{\partial}{\partial x}$ and $Y = x^2 \frac{\partial}{\partial y}$
be vector fields on $\R^2$.
Compute the Lie bracket $[X,Y]$.  Namely, determine smooth functions $f(x,y)$ and $g(x,y)$ for which
 $[X,Y] = f(x,y) \frac{\partial}{\partial x} + g(x,y) \frac{\partial}{\partial y}$.
\end{prob}


\vspace{3mm}
\end{comment}



\begin{comment}
\begin{prob}[$15$ pts]
Determine whether each of the following statements is true or false.  If false, provide a counterexample.  If true, briefly justify your answer.
\begin{enumerate}[(a)]
 \item Let $F: N \to M$ be a one-to-one immersion, and let $n$ and $m$ denote the dimensions of $N$ and $M$ respectively.
 For any point $q \in F(N) \subset M$, there exists a coordinate neighborhood $(U, x^1, \ldots, x^m)$ about $q$ such that
 $F(N) \cap U$ is defined by the vanishing of $x^1, \ldots, x^{m-n}$.
 \item Any alternating bilinear map $\omega \in A_2(\R^3)$ is \emph{decomposable}, 
 in the sense that $\omega = \alpha_1 \wedge \alpha_2$ for some
 covectors $\alpha_1, \alpha_2 \in V^*$.
  \item Any smooth vector field on the torus vanishes on at least one point.
\end{enumerate}
\end{prob}
\vspace{3mm}
\end{comment}


\begin{prob}[$15$ pts]
Let $F: N \to M$ be a smooth map of smooth manifolds.
Show that if $F_{*,p}$ has rank $k$ for some $p \in N$,
then there exists an open neighborhood $U$ of $p$ such that 
the differential $F_{*,q}$ has rank at least $k$, for any $q \in U$.
\end{prob}


\vspace{3mm}

\begin{prob}[$15$ pts]
 Let $M$ be a smooth manifold.  Let $F: M \to M$ be  a smooth map, and suppose $c \in M$ is a  regular value of $F$.
 Show that the subspace topology on $F^{-1}(c)$ is in fact the discrete topology. 
\end{prob}


\vspace{3mm}

\begin{prob}[$15$ pts]
Let $G$ be a Lie group, and let $H$ be a subgroup of $G$.  Let $h_0$ be a point of $H$.
Suppose there exists an adapted chart $(U, \phi) = (U, x^1, \ldots, x^n)$ relative to $H$ about the point $h_0$.
In particular, suppose $H \cap U$ is defined by the vanishing of $x^1, \ldots, x^k$ for some
$1 \leq k \leq n$.  Prove that $H$ is a regular submanifold of $G$.
\end{prob}


\vspace{3mm}




\begin{prob}[$15$ pts]
 Let $M$ be a smooth manifold.  Prove that the projection map $\pi: TM \to M$ is a submersion.
\end{prob}

\vspace{3mm}

\begin{prob}[$20$ pts]
We identify the set of $n \times n$ matrices with the Euclidean space $\R^{n^2}$.
Let $A \subset \R^{n^2}$ denote the set of symmetric matrices and let $B \subset \R^{n^2}$ denote the set of skew-symmetric matrices.
 \begin{enumerate}[(a)]
  \item  Prove that $A$ and $B$ are both regular submanifolds of $\R^{n^2}$.
  \item  Prove that $A$ and $B$ intersect transversely, in the sense that for any point $p \in A \cap B$,
  we have $T_pA + T_pB = T_p (\R^{n^2})$.
 \end{enumerate}
\end{prob}

\vspace{3mm}

\begin{prob}[$20$ pts]
 Let $S^1 \subset \R^2$ denote the unit circle, and let  $M := S^1 \times S^1 \subset \R^4$
 denote the torus.
 \begin{enumerate}[(a)]
  \item Construct a $C^{\infty}$ atlas for $M$. 
  \item Prove that the tangent bundle $TM$ is trivial.
 \end{enumerate}

\end{prob}


\end{document}