
\documentclass{amsart}
%\usepackage{cmbright}
\usepackage{amsmath, amsfonts, amssymb, amscd}
%\usepackage{tableau}
\usepackage{color}
\usepackage{xcolor}
\usepackage{graphicx}
\usepackage{array}
\usepackage{mathtools}
\usepackage{multirow}
\usepackage{framed}
\usepackage{tikz}
\usetikzlibrary{matrix,arrows}
\usepackage[square,sort,comma,numbers]{natbib}
\usepackage{enumerate}




%\oddsidemargin.5cm
%\evensidemargin.5cm
\addtolength{\oddsidemargin}{-.525in}
\addtolength{\evensidemargin}{-.525in}
\addtolength{\textwidth}{1in}

\addtolength{\topmargin}{-.87in}
\addtolength{\textheight}{1.5in}




\newcommand\coolunder[2]{\mathrlap{\smash{\underbrace{\phantom{%
    \begin{matrix} #2 \end{matrix}}}_{\mbox{$#1$}}}}#2}

%\input{tableau}


\newcommand{\+}[1]{\ensuremath{\mathbf{#1}}}
\newcommand{\vect}[1]{\boldsymbol{#1}} % Uncomment for BOLD vectors.
%\newcommand{\vect}[1]{\vec{#1}} % Uncomment for ARROW vectors.
\newcommand{\C}{{\mathbb C}}
\newcommand{\Z}{{\mathbb Z}}
\newcommand{\QH}{{\mathbb H}}
\renewcommand{\P}{{\mathbb P}}
\newcommand{\OG}{\operatorname{OG}}
\newcommand{\OF}{\operatorname{OF}}
\newcommand{\bull}{{\scriptscriptstyle \bullet}}
\newcommand{\la}{\lambda}
\newcommand{\euler}[1]{\chi_{_{#1}}}
\newcommand{\cO}{{\mathcal O}}
\newcommand{\cG}{{\mathcal G}}
\newcommand{\cQ}{{\mathcal Q}}
\newcommand{\R}{{\mathbb R}}
\newcommand{\wt}{\widetilde}
\newcommand{\diag}{\operatorname{diag}}
\newcommand{\comp}{\operatorname{comp}}
\newcommand{\comment}[1]{}
\newcommand{\type}{\mathfrak{t}}
\newcommand{\op}{\text{op}}
\newcommand{\row}{{\bf r}}
\newcommand{\col}{{\bf c}}
\newcommand{\sym}{\mathfrak{S}}
\newcommand{\codim}{\text{codim}}
\DeclarePairedDelimiter{\ceil}{\lceil}{\rceil}
\DeclarePairedDelimiter{\floor}{\lfloor}{\rfloor}
\renewcommand{\emptyset}{\varnothing}

\newtheorem{thm}{Theorem}
\newtheorem{lemma}[thm]{Lemma}
\newtheorem{prop}[thm]{Proposition}
\newtheorem{cor}[thm]{Corollary}

\theoremstyle{definition}
\newtheorem{definition}[thm]{Definition}
\newtheorem{example}[thm]{Example}
\newtheorem{conj}[thm]{Conjecture}
\newtheorem{obs}[thm]{Observation}
\newtheorem{fact}[thm]{Fact}
\newtheorem{remark}[thm]{Remark}
\newtheorem{prob}{Problem}
\newtheorem{chal}{Challenge}

\begin{document}
\title{Problem Set 9}
\date{October 31, 2016}
\author{Introduction to Manifolds}

\maketitle

\begin{prob}
Consider the top form $\omega = dx^1 \wedge \ldots \wedge dx^n \in \Omega^n(\R^n)$
and the radial vector field
$X = \sum x^i \frac{\partial}{\partial x^i} \in \mathfrak{X}(\R^n)$.
\begin{enumerate}[(a)]
 \item Compute the contraction $\iota_X\omega$.
 \item Show that $\iota_X\omega$ restricts to a nowhere vanishing top form
 on any sphere centered at the origin.
\end{enumerate}
\end{prob}


\begin{prob}
Construct an oriented atlas on $S^1$.
\end{prob}

\begin{prob}
Let $M$ be a smooth manifold.  Prove that the tangent bundle
$TM$ is orientable.  (Hint: either exhibit an oriented atlas
as in Problem 21.9 of [Tu], or construct a nonwhere-vanishing
top form.)
\end{prob}

\begin{prob}
Orient the unit sphere $S^n$ in $R^{n+1}$ as the boundary
of the closed unit ball.  Let $U = \{x \in S^n: x^{n+1}>0\}$ 
be the upper hemisphere.  Note that $(U,\pi) := (U,x^1,\ldots,x^n)$
is a coordinate chart on $S^n$.
\begin{enumerate}[(a)]
 \item Show that the projection $\pi: U \to \R^n$ is
 orientation-preserving if and only if $n$ is even.
 \item Show that the antipodal map $a: S^n \to S^n$
 is orientation-preserving if and only if $n$ is odd.
 \item Show that $\R P^n$ is orientable if $n$ is odd.
\end{enumerate}
\end{prob}



\begin{prob}
Let $T^2 = \{(w,x,y,z): w^2 + x^2 = y^2 + z^2 =1\} \subset \R^4$ be the oriented torus with
orientation form $(-x dw + w dx) \wedge (-z dy + y dz)$.
Let $S^2 = \{(x,y,z): x^2 + y^2 + z^2 = 1\} \subset \R^3$
be the oriented sphere with orientation form
$x dy \wedge dz - y dx \wedge dz + z dx \wedge dy$.
Compute the integrals
\begin{enumerate}[(a)]
 \item  $\int_{T^2} wy dx \wedge dz$,
 \item  $\int_{S^2} x dy \wedge dz + y dz \wedge dx + z dx \wedge dy$, and 
 \item  $\int_{S^2} xz dy \wedge dz + yz dz \wedge dx + x^2 dx \wedge dy$.
\end{enumerate}
\end{prob}



\end{document}