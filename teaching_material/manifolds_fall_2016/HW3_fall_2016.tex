
\documentclass{amsart}
%\usepackage{cmbright}
\usepackage{amsmath, amsfonts, amssymb, amscd}
%\usepackage{tableau}
\usepackage{color}
\usepackage{xcolor}
\usepackage{graphicx}
\usepackage{array}
\usepackage{mathtools}
\usepackage{multirow}
\usepackage{framed}
\usepackage{tikz}
\usetikzlibrary{matrix,arrows}
\usepackage[square,sort,comma,numbers]{natbib}
\usepackage{enumerate}




%\oddsidemargin.5cm
%\evensidemargin.5cm
\addtolength{\oddsidemargin}{-.525in}
\addtolength{\evensidemargin}{-.525in}
\addtolength{\textwidth}{1in}

\addtolength{\topmargin}{-.87in}
\addtolength{\textheight}{1.5in}




\newcommand\coolunder[2]{\mathrlap{\smash{\underbrace{\phantom{%
    \begin{matrix} #2 \end{matrix}}}_{\mbox{$#1$}}}}#2}

%\input{tableau}


\newcommand{\+}[1]{\ensuremath{\mathbf{#1}}}
\newcommand{\vect}[1]{\boldsymbol{#1}} % Uncomment for BOLD vectors.
%\newcommand{\vect}[1]{\vec{#1}} % Uncomment for ARROW vectors.
\newcommand{\C}{{\mathbb C}}
\newcommand{\Z}{{\mathbb Z}}
\renewcommand{\P}{{\mathbb P}}
\newcommand{\OG}{\operatorname{OG}}
\newcommand{\OF}{\operatorname{OF}}
\newcommand{\bull}{{\scriptscriptstyle \bullet}}
\newcommand{\la}{\lambda}
\newcommand{\euler}[1]{\chi_{_{#1}}}
\newcommand{\cO}{{\mathcal O}}
\newcommand{\cG}{{\mathcal G}}
\newcommand{\cQ}{{\mathcal Q}}
\newcommand{\R}{{\mathbb R}}
\newcommand{\wt}{\widetilde}
\newcommand{\diag}{\operatorname{diag}}
\newcommand{\comp}{\operatorname{comp}}
\newcommand{\comment}[1]{}
\newcommand{\type}{\mathfrak{t}}
\newcommand{\op}{\text{op}}
\newcommand{\row}{{\bf r}}
\newcommand{\col}{{\bf c}}
\newcommand{\sym}{\mathfrak{S}}
\newcommand{\codim}{\text{codim}}
\DeclarePairedDelimiter{\ceil}{\lceil}{\rceil}
\DeclarePairedDelimiter{\floor}{\lfloor}{\rfloor}
\renewcommand{\emptyset}{\varnothing}

\newtheorem{thm}{Theorem}
\newtheorem{lemma}[thm]{Lemma}
\newtheorem{prop}[thm]{Proposition}
\newtheorem{cor}[thm]{Corollary}

\theoremstyle{definition}
\newtheorem{definition}[thm]{Definition}
\newtheorem{example}[thm]{Example}
\newtheorem{conj}[thm]{Conjecture}
\newtheorem{obs}[thm]{Observation}
\newtheorem{fact}[thm]{Fact}
\newtheorem{remark}[thm]{Remark}
\newtheorem{prob}{Problem}
\newtheorem{chal}{Challenge}

\begin{document}
\title{Problem Set 3}
\date{August 19, 2016}
\author{Introduction to Manifolds}

\maketitle



\begin{prob}

 \begin{enumerate}[(a)]
  \item Show that the `line with two origins' is locally Euclidean and second
  countable, but not Hausdorff.  (This pathological space
  is defined by quotienting 
  $\{(x,y) \in \R^2: y = \pm 1\}$ by the equivalence relation
  $(x,-1) \sim (x, +1)$ for all $x \neq 0$.)
  \item Show that the disjoint union of uncountably many copies
  of $\R$ is locally Euclidean and Hausdorff, but not 
  second countable.
 \end{enumerate}
\end{prob}



\begin{prob}
Let $S^n$ denote the unit sphere in $\R^{n+1}$,
with north pole $N := (0,\ldots,0,1)$ and south pole
$S := (0,\ldots,-1)$.
Let $p_N: S^n \setminus \{N\} \to \R^n$ be the \emph{stereographic projection}
defined by
\[
(x^1,\ldots,x^{n+1}) \mapsto \frac{(x^1,\ldots,x^n)}{1-x^{n+1}}.
\]
Let $p_S: S^n \setminus \{S\} \to \R^n$ be defined by
$p_S(\+x) = p_N(-\+x)$.
\begin{enumerate}[(a)]
\item Show that $p_N$ is bijective, with inverse
\[
 p_N^{-1}(y_1,\ldots,y_n) = \frac{2y^1,\ldots,2y^n,|y|^2-1}{|y|^2+1}.
\]
\item Compute the transition map $p_S \circ p_N^{-1}$.
Verify that the atlas 
\[\{(S^n \setminus \{N\},p_N),(S^n \setminus \{S\},p_S)\}\]
defines a differentiable structure on $S^n$.
\end{enumerate}
\end{prob}


\begin{prob}
 Let $\R \P^n$ denote the \emph{set} of 
 lines through the origin in $\R^{n+1}$.
 Let $[x^1:\ldots:x^{n+1}]$ denote the
 equivalence class of vectors in $\R^{n+1}$ related by
 scalar multiplication
 (hence $[x^1:\ldots:x^{n+1}] = [\lambda x^1: 
 \ldots: \lambda x^{n+1}]$ for
 any $\lambda \in \R^*$).
 Let $U_i := \{[x^1:\ldots:x^{n+1}] : x^i \neq 0\} \subset
 \R\P^n$,
 and let $\phi_i : U_i \to \R^{n}$ be defined by
 $[x^1:\ldots:x^{n+1}] \mapsto (\frac{x^1}{x^i},\ldots,
 \widehat{\frac{x^i}{x^i}},\ldots, \frac{x^{n+1}}{x^i})$,
  for $i = 1,\ldots,n+1$.  
 Finally, define a collection of subsets of $\R \P^n$,
  \[\mathcal{B} := \{\phi_i^{-1}(V): V \subset \R^n 
  \text{ is open },  1 \leq i \leq n+1\}.\]
  \begin{enumerate}[(a)]
   \item Show that $\mathcal{B}$ is a basis for a topology on 
   $\R \P^n$.
   \item Show that $\R \P^n$ is second countable and Hausdorff
   under this topology.
   \item Show that the set of charts $\{(U_i,\phi_i)\}$ forms a 
   $C^\infty$-atlas on $\R \P^n$.  Hence $\R \P^n$ is a smooth manifold.
  \end{enumerate}

\end{prob}




\begin{prob}
Consider the following $C^\infty$-atlases on $\R$,
each consisting of a single chart:
\[
\mathfrak{A} := \{(\R, \phi: x \mapsto x)\}, \text{ and }
\mathfrak{B} := \{(\R, \psi: x \mapsto x^{1/3})\}.
\]
\begin{enumerate}[(a)]
 \item Show that the resulting differentiable structures on $\R$
 are distinct (i.e. that the induced maximal atlases are
 distinct).
 \item Show that there is a diffeomorphism between the smooth
 manifolds $(\R, \mathfrak{A})$ and $(\R, \mathfrak{B})$.
\end{enumerate}
\end{prob}






\end{document}