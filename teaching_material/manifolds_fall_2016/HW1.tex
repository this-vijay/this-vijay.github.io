
\documentclass{amsart}
%\usepackage{cmbright}
\usepackage{amsmath, amsfonts, amssymb, amscd}
%\usepackage{tableau}
\usepackage{color}
\usepackage{xcolor}
\usepackage{graphicx}
\usepackage{array}
\usepackage{mathtools}
\usepackage{multirow}
\usepackage{framed}
\usepackage{tikz}
\usetikzlibrary{matrix,arrows}
\usepackage[square,sort,comma,numbers]{natbib}
\usepackage{enumerate}




%\oddsidemargin.5cm
%\evensidemargin.5cm
\addtolength{\oddsidemargin}{-.525in}
\addtolength{\evensidemargin}{-.525in}
\addtolength{\textwidth}{1in}

\addtolength{\topmargin}{-.87in}
\addtolength{\textheight}{1.5in}




\newcommand\coolunder[2]{\mathrlap{\smash{\underbrace{\phantom{%
    \begin{matrix} #2 \end{matrix}}}_{\mbox{$#1$}}}}#2}

%\input{tableau}


\newcommand{\+}[1]{\ensuremath{\mathbf{#1}}}
\newcommand{\vect}[1]{\boldsymbol{#1}} % Uncomment for BOLD vectors.
%\newcommand{\vect}[1]{\vec{#1}} % Uncomment for ARROW vectors.
\newcommand{\C}{{\mathbb C}}
\newcommand{\Z}{{\mathbb Z}}
\renewcommand{\P}{{\mathbb P}}
\newcommand{\OG}{\operatorname{OG}}
\newcommand{\OF}{\operatorname{OF}}
\newcommand{\bull}{{\scriptscriptstyle \bullet}}
\newcommand{\la}{\lambda}
\newcommand{\euler}[1]{\chi_{_{#1}}}
\newcommand{\cO}{{\mathcal O}}
\newcommand{\cG}{{\mathcal G}}
\newcommand{\cQ}{{\mathcal Q}}
\newcommand{\R}{{\mathbb R}}
\newcommand{\wt}{\widetilde}
\newcommand{\diag}{\operatorname{diag}}
\newcommand{\comp}{\operatorname{comp}}
\newcommand{\comment}[1]{}
\newcommand{\type}{\mathfrak{t}}
\newcommand{\op}{\text{op}}
\newcommand{\row}{{\bf r}}
\newcommand{\col}{{\bf c}}
\newcommand{\sym}{\mathfrak{S}}
\newcommand{\codim}{\text{codim}}
\DeclarePairedDelimiter{\ceil}{\lceil}{\rceil}
\DeclarePairedDelimiter{\floor}{\lfloor}{\rfloor}
\renewcommand{\emptyset}{\varnothing}

\newtheorem{thm}{Theorem}
\newtheorem{lemma}[thm]{Lemma}
\newtheorem{prop}[thm]{Proposition}
\newtheorem{cor}[thm]{Corollary}

\theoremstyle{definition}
\newtheorem{definition}[thm]{Definition}
\newtheorem{example}[thm]{Example}
\newtheorem{conj}[thm]{Conjecture}
\newtheorem{obs}[thm]{Observation}
\newtheorem{fact}[thm]{Fact}
\newtheorem{remark}[thm]{Remark}
\newtheorem{prob}{Problem}


\begin{document}
\title{Problem Set 1}
\date{August 01, 2016}
\author{Introduction to Manifolds}

\maketitle


\begin{framed}
\begin{thm}[Taylor's Theorem with Remainder]\label{T:complicated_taylor}
Let $U \subset \R^n$ be an open subset that is star-shaped with respect to a point $p = (p^1,\ldots,p^n) \in U$.
Suppose $f:U \to \R$ is a $C^\infty$ function on $U$.  Let $k$ be any positive integer.  We then have
\begin{align*}
f(x) = f(p) + \sum^n_{i=1} (x^i - p^i) \frac{\partial f}{\partial x^i} (p) + \ldots
+ \frac{1}{k!} \sum_{i_1,\ldots,i_k} (x^{i_1} - p^{i_1})\cdots(x^{i_k} - p^{i_k}) 
\frac{\partial^k f}{\partial x^{i_1} \cdots \partial x^{i_k}}(p) \\
+ {\frac{1}{k!}}\sum_{i_1,\ldots,i_{k+1}}(x^{i_1} - p^{i_1})\cdots(x^{i_{k+1}} - p^{i_{k+1}})\int^1_0 (1-t)^k 
\frac{\partial^{k+1} f}{\partial x^{i_1} \cdots \partial x^{i_{k+1}}}(p + t(x-p)) dt.
\end{align*}
\end{thm}
\end{framed}




\begin{prob}
Prove Theorem \ref{T:complicated_taylor} for $k = 2$.  It may help to define the
 path $\gamma(t) := p + t(x-p)$ with $0 \leq t \leq 1$, and to consider
 $\frac{d}{dt} f(\gamma(t))$, as in the proof of the other formulation of Taylor's theorem
 in the textbook.
\end{prob} 


\begin{prob}
 Prove that the following function is $C^\infty$ at $0$ but not real-analytic at $0$:
 \begin{equation*}
  f(x) =
  \begin{cases}
   e^{-1/x} &\text{ if } x > 0, \text{ and} \\
   0 &\text{otherwise}.
  \end{cases}
 \end{equation*}
\end{prob}

\begin{prob}
 Prove that the set of all point derivations of $C^{\infty}_p(\R^n)$ forms a vector space.
\end{prob}

\begin{prob}
Let $A$ be an algebra over a field $K$.  If $D_1$ and $D_2$ are derivations
of $A$, show that $D_1 \circ D_2$ is not necessarily a derivation of $A$, but that
$D_1 \circ D_2 - D_2 \circ D_1$ is always a derivation of $A$.
\end{prob}






\end{document}