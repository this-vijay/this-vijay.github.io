
\documentclass{amsart}
%\usepackage{cmbright}
\usepackage{amsmath, amsfonts, amssymb, amscd}
%\usepackage{tableau}
\usepackage{color}
\usepackage{xcolor}
\usepackage{graphicx}
\usepackage{array}
\usepackage{mathtools}
\usepackage{multirow}
\usepackage{framed}
\usepackage{tikz}
\usetikzlibrary{matrix,arrows}
\usepackage[square,sort,comma,numbers]{natbib}
\usepackage{enumerate}




%\oddsidemargin.5cm
%\evensidemargin.5cm
\addtolength{\oddsidemargin}{-.525in}
\addtolength{\evensidemargin}{-.525in}
\addtolength{\textwidth}{1in}

\addtolength{\topmargin}{-.87in}
\addtolength{\textheight}{1.5in}




\newcommand\coolunder[2]{\mathrlap{\smash{\underbrace{\phantom{%
    \begin{matrix} #2 \end{matrix}}}_{\mbox{$#1$}}}}#2}

%\input{tableau}


\newcommand{\+}[1]{\ensuremath{\mathbf{#1}}}
\newcommand{\vect}[1]{\boldsymbol{#1}} % Uncomment for BOLD vectors.
%\newcommand{\vect}[1]{\vec{#1}} % Uncomment for ARROW vectors.
\newcommand{\C}{{\mathbb C}}
\newcommand{\Z}{{\mathbb Z}}
\newcommand{\QH}{{\mathbb H}}
\renewcommand{\P}{{\mathbb P}}
\newcommand{\OG}{\operatorname{OG}}
\newcommand{\OF}{\operatorname{OF}}
\newcommand{\bull}{{\scriptscriptstyle \bullet}}
\newcommand{\la}{\lambda}
\newcommand{\euler}[1]{\chi_{_{#1}}}
\newcommand{\cO}{{\mathcal O}}
\newcommand{\cG}{{\mathcal G}}
\newcommand{\cQ}{{\mathcal Q}}
\newcommand{\R}{{\mathbb R}}
\newcommand{\wt}{\widetilde}
\newcommand{\diag}{\operatorname{diag}}
\newcommand{\comp}{\operatorname{comp}}
\newcommand{\comment}[1]{}
\newcommand{\type}{\mathfrak{t}}
\newcommand{\op}{\text{op}}
\newcommand{\row}{{\bf r}}
\newcommand{\col}{{\bf c}}
\newcommand{\sym}{\mathfrak{S}}
\newcommand{\codim}{\text{codim}}
\DeclarePairedDelimiter{\ceil}{\lceil}{\rceil}
\DeclarePairedDelimiter{\floor}{\lfloor}{\rfloor}
\renewcommand{\emptyset}{\varnothing}

\newtheorem{thm}{Theorem}
\newtheorem{lemma}[thm]{Lemma}
\newtheorem{prop}[thm]{Proposition}
\newtheorem{cor}[thm]{Corollary}

\theoremstyle{definition}
\newtheorem{definition}[thm]{Definition}
\newtheorem{example}[thm]{Example}
\newtheorem{conj}[thm]{Conjecture}
\newtheorem{obs}[thm]{Observation}
\newtheorem{fact}[thm]{Fact}
\newtheorem{remark}[thm]{Remark}
\newtheorem{prob}{Problem}
\newtheorem{chal}{Challenge}

\begin{document}
\title{Problem Set 7}
\date{October 7, 2016}
\author{Introduction to Manifolds}

\maketitle

\begin{prob}
 Prove that an open subgroup $H$ of a connected Lie group $G$ is equal to $G$.
\end{prob}

\begin{prob}
 Show that the differential of the determinant map at $A \in GL(n,\R)$ is given by
 \[
 \text{det}_{*,A}(AX) = \text{det}(A)\text{trace}(X)
 \]
 for $X \in \R^{n \times n}$
\end{prob}



\begin{prob}
Prove that 
\begin{enumerate}[(a)]
 \item the real orthogonal group $O(n,\R)$ is compact, but
 \item the complex orthogonal group $O(n,\C)$ is not compact.
\end{enumerate}
\end{prob}


\begin{prob}
Show that the special unitary group $SU(2) \subset GL(2,\C)$
is diffeomorphic to the three-sphere.
(There are several approaches to this problem.  For example, one approach is
outlined in problem $15.13$ of [Tu].  We will discuss at least two other approaches in class later.)
 \end{prob}

 \begin{prob}
Let $A \in \mathfrak{gl}(n,\R)$ and let $\tilde{A}$ be the
left-invariant vector field on $GL(n,\R)$ generated by $A$.
Show that $c(t) := e^{tA}$ is the integral curve of $\tilde{A}$
starting at the identity matrix $I$.
Find the integral curve of $\tilde{A}$ starting at $g \in GL(n,\R)$.
 \end{prob}

 
 \vspace{5mm}

{\bf The following problem is optional.  It
will not contribute to or detract from your grade.}
 
\vspace{5mm}

\begin{chal}
Let $G$ be a Lie group.  Construct a subgroup $H \subset G$ that is an immersed submanifold of $G$ but \emph{not} 
a Lie subgroup.
\end{chal}


\end{document}