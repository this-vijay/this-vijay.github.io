
\documentclass{amsart}
%\usepackage{cmbright}
\usepackage{amsmath, amsfonts, amssymb, amscd}
%\usepackage{tableau}
\usepackage{color}
\usepackage{xcolor}
\usepackage{graphicx}
\usepackage{array}
\usepackage{mathtools}
\usepackage{multirow}
\usepackage{framed}
\usepackage{tikz}
\usetikzlibrary{matrix,arrows}
\usepackage[square,sort,comma,numbers]{natbib}
\usepackage{enumerate}




%\oddsidemargin.5cm
%\evensidemargin.5cm
\addtolength{\oddsidemargin}{-.525in}
\addtolength{\evensidemargin}{-.525in}
\addtolength{\textwidth}{1in}

\addtolength{\topmargin}{-.87in}
\addtolength{\textheight}{1.5in}




\newcommand\coolunder[2]{\mathrlap{\smash{\underbrace{\phantom{%
    \begin{matrix} #2 \end{matrix}}}_{\mbox{$#1$}}}}#2}

%\input{tableau}


\newcommand{\+}[1]{\ensuremath{\mathbf{#1}}}
\newcommand{\vect}[1]{\boldsymbol{#1}} % Uncomment for BOLD vectors.
%\newcommand{\vect}[1]{\vec{#1}} % Uncomment for ARROW vectors.
\newcommand{\C}{{\mathbb C}}
\newcommand{\Z}{{\mathbb Z}}
\newcommand{\QH}{{\mathbb H}}
\renewcommand{\P}{{\mathbb P}}
\newcommand{\OG}{\operatorname{OG}}
\newcommand{\OF}{\operatorname{OF}}
\newcommand{\bull}{{\scriptscriptstyle \bullet}}
\newcommand{\la}{\lambda}
\newcommand{\euler}[1]{\chi_{_{#1}}}
\newcommand{\cO}{{\mathcal O}}
\newcommand{\cG}{{\mathcal G}}
\newcommand{\cQ}{{\mathcal Q}}
\newcommand{\R}{{\mathbb R}}
\newcommand{\wt}{\widetilde}
\newcommand{\diag}{\operatorname{diag}}
\newcommand{\comp}{\operatorname{comp}}
\newcommand{\comment}[1]{}
\newcommand{\type}{\mathfrak{t}}
\newcommand{\op}{\text{op}}
\newcommand{\row}{{\bf r}}
\newcommand{\col}{{\bf c}}
\newcommand{\sym}{\mathfrak{S}}
\newcommand{\codim}{\text{codim}}
\DeclarePairedDelimiter{\ceil}{\lceil}{\rceil}
\DeclarePairedDelimiter{\floor}{\lfloor}{\rfloor}
\renewcommand{\emptyset}{\varnothing}

\newtheorem{thm}{Theorem}
\newtheorem{lemma}[thm]{Lemma}
\newtheorem{prop}[thm]{Proposition}
\newtheorem{cor}[thm]{Corollary}

\theoremstyle{definition}
\newtheorem{definition}[thm]{Definition}
\newtheorem{example}[thm]{Example}
\newtheorem{conj}[thm]{Conjecture}
\newtheorem{obs}[thm]{Observation}
\newtheorem{fact}[thm]{Fact}
\newtheorem{remark}[thm]{Remark}
\newtheorem{prob}{Problem}
\newtheorem{chal}{Challenge}

\begin{document}
\title{Problem Set 8}
\date{October 20, 2016}
\author{Introduction to Manifolds}

\maketitle

\begin{prob}[Problem (7-9) of Lee]
 Let $G$ be a Lie group, and let $G_0$ denote the
 connected component of the identity.
\begin{enumerate}[(a)]
 \item Show that $G_0$ is an embedded Lie subgroup of $G$,
 and that each connected component of $G$ is diffeomorphic
 to $G_0$.
 \item If $H$ is any connected open subgroup of $G$, show that
 $H = G_0$. 
\end{enumerate}
 \end{prob}

\begin{prob}[Problem (7-13) of Lee]
 Prove that $SO(3,\R)$ is Lie isomorphic to $SU(2,\C)/\{\pm I\}$
 and diffeomorphic to $\R P^3$ (you can use the strategy outlined in Lee).
\end{prob}



\begin{prob}
Let $G$ be a Lie group of dimension $n$ with Lie algebra $\mathfrak{g}$.
For each $g \in G$, let $c_g := l_g \circ r_{g^{-1}}:G \to G$
be the corresponding conjugation map.  Note that
the differential at the identity $c_{g*}: \mathfrak{g} \to \mathfrak{g}$
is a linear isomorphism, and hence $c_{g*} \in GL(\mathfrak{g})$.
Show that $Ad: G \to GL(\mathfrak{g})$ defined by
$Ad(g) = c_{g*}$ is a homomorphism of Lie groups; i.e. that
it is an abstract group homomorphism \emph{and} a smooth map between
manifolds.
\end{prob}

\begin{prob}
\begin{enumerate}[(a)]
 \item Let $S$ be a regular submanifold of codimension $k$ in a smooth manifold $M$ of dimension $n$.
Let $(U,\phi) = (U, f^1, \ldots, f^n)$ be local coordinates about a point $p \in S$
such that $S \cap U$ is defined by the vanishing of $f^1, \ldots, f^k$.
Show that $T_pS = \bigcap^k_{i = 1} \text{ker} (df^i)$.
\item Note that a vector field $X$ on $M$ restricts to a vector field on a submanifold $S \subset M$ if $X_p \in T_pS$
for any $p \in S$.  Construct a vector field on $\R^{2n}$ that restricts to a nowhere-vanishing vector
field on the unit sphere $S^{2n-1} \subset \R^{2n}$ (you can use part (a) to show that it does indeed
restrict to $S^{2n-1}$).
\end{enumerate}
\end{prob}




\begin{prob}
Using the fact that the pullback of a smooth $k$-form is a
smooth $k$-form, prove that if $\pi: \tilde{M}\to M$
is a surjective submersion, then the pullback map
$\pi^*: \Omega^*(M) \to \Omega^*(\tilde{M})$ is
an injective algebra homomorphism.
\end{prob}




\end{document}