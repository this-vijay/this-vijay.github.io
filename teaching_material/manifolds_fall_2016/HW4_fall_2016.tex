
\documentclass{amsart}
%\usepackage{cmbright}
\usepackage{amsmath, amsfonts, amssymb, amscd}
%\usepackage{tableau}
\usepackage{color}
\usepackage{xcolor}
\usepackage{graphicx}
\usepackage{array}
\usepackage{mathtools}
\usepackage{multirow}
\usepackage{framed}
\usepackage{tikz}
\usetikzlibrary{matrix,arrows}
\usepackage[square,sort,comma,numbers]{natbib}
\usepackage{enumerate}




%\oddsidemargin.5cm
%\evensidemargin.5cm
\addtolength{\oddsidemargin}{-.525in}
\addtolength{\evensidemargin}{-.525in}
\addtolength{\textwidth}{1in}

\addtolength{\topmargin}{-.87in}
\addtolength{\textheight}{1.5in}




\newcommand\coolunder[2]{\mathrlap{\smash{\underbrace{\phantom{%
    \begin{matrix} #2 \end{matrix}}}_{\mbox{$#1$}}}}#2}

%\input{tableau}


\newcommand{\+}[1]{\ensuremath{\mathbf{#1}}}
\newcommand{\vect}[1]{\boldsymbol{#1}} % Uncomment for BOLD vectors.
%\newcommand{\vect}[1]{\vec{#1}} % Uncomment for ARROW vectors.
\newcommand{\C}{{\mathbb C}}
\newcommand{\Z}{{\mathbb Z}}
\renewcommand{\P}{{\mathbb P}}
\newcommand{\OG}{\operatorname{OG}}
\newcommand{\OF}{\operatorname{OF}}
\newcommand{\bull}{{\scriptscriptstyle \bullet}}
\newcommand{\la}{\lambda}
\newcommand{\euler}[1]{\chi_{_{#1}}}
\newcommand{\cO}{{\mathcal O}}
\newcommand{\cG}{{\mathcal G}}
\newcommand{\cQ}{{\mathcal Q}}
\newcommand{\R}{{\mathbb R}}
\newcommand{\wt}{\widetilde}
\newcommand{\diag}{\operatorname{diag}}
\newcommand{\comp}{\operatorname{comp}}
\newcommand{\comment}[1]{}
\newcommand{\type}{\mathfrak{t}}
\newcommand{\op}{\text{op}}
\newcommand{\row}{{\bf r}}
\newcommand{\col}{{\bf c}}
\newcommand{\sym}{\mathfrak{S}}
\newcommand{\codim}{\text{codim}}
\DeclarePairedDelimiter{\ceil}{\lceil}{\rceil}
\DeclarePairedDelimiter{\floor}{\lfloor}{\rfloor}
\renewcommand{\emptyset}{\varnothing}

\newtheorem{thm}{Theorem}
\newtheorem{lemma}[thm]{Lemma}
\newtheorem{prop}[thm]{Proposition}
\newtheorem{cor}[thm]{Corollary}

\theoremstyle{definition}
\newtheorem{definition}[thm]{Definition}
\newtheorem{example}[thm]{Example}
\newtheorem{conj}[thm]{Conjecture}
\newtheorem{obs}[thm]{Observation}
\newtheorem{fact}[thm]{Fact}
\newtheorem{remark}[thm]{Remark}
\newtheorem{prob}{Problem}
\newtheorem{chal}{Challenge}

\begin{document}
\title{Problem Set 4}
\date{August 28, 2016}
\author{Introduction to Manifolds}

\maketitle



%\begin{prob}
% Let $M \subset \R^n$ be a smooth manifold of dimension $m$,
% and let $q$ be a point in $M$.
% For $i \in \{1,2\}$, suppose $U_i \subset \R^m$ is an open neighborhood of the origin and $P_i: U_i \to \R^n$ is a smooth parametrization of $M$ on some neighborhood of $q = P_i(0)$.
% Recall that the Jacobian $J(P_i)|_0$ is defined to be
% the matrix $[\frac{\partial P^j_i}{\partial r^k}(0)] \in M_{n \times m}$.
% Prove {\bf directly}\footnote{Feel free
% to use standard results from linear algebra and multivariable calculus, like the chain rule.  But do not use more general results concerning
% the differential of a smooth map between abstract manifolds.} that as linear transformations,
% the matrices $J(P_1)|_0$ and $J(P_2)|_0$ have the same image in $\R^n$.  
%\end{prob}

\begin{prob}
 Let $L: \R^n \to \R^m$ be a linear map.  For any $p \in \R^n$ there is a canonical identification $T_p(\R^n) \to \R^n$
 given by 
 \[
  \sum {a^i} \frac{\partial}{\partial x^i}\bigg|_p \mapsto \langle a^1, \ldots a^n \rangle.
 \]
 Show that the differential $L_{*,p}: T_p(\R^n) \to T_{L(p)}(\R^m)$ is exactly the map $L$
 under the above identification.
\end{prob}


\begin{prob}
Given smooth manifolds $M$ and $N$, let $\pi_1: M \times N \to M$
and $\pi_2: M \times N \to N$ be the two projections.
Prove that for any $(p,q) \in M \times N$,
\[
\pi_{1*} \oplus \pi_{2*}: T_{(p,q)}(M \times N) \to T_pM \oplus T_qN
\]
is an isomorphism.
\end{prob}


\begin{prob}
Let $G$ be a Lie group with multiplication map $\mu: G \times G \to G$, inverse map $\iota: G \to G$, and identity $e$.
\begin{enumerate}[(a)]
 \item Show that $\mu_{*,(e,e)}(X,Y) = X + Y$
 for any $(X,Y) \in T_eG \oplus T_eG \cong T_{(e,e)}(G \times G)$.
 \item Show that $\iota_{*,e}(X) = -X$ for
 any $X \in T_eG$.
\end{enumerate}
\end{prob}





\begin{prob}
Let $M$ be a smooth manifold.
A function $f: M \to \R$ is said to be a \emph{local maximum}
if there is a neighborhood $U$ of $p$ such that
$f(p) \geq f(q)$ for all $q \in U$.
\begin{enumerate}[(a)]
 \item Let $I$ be an open interval in $\R$.
 Prove that if a differentiable function $f:I \to \R$ has a local maximum at $p \in I$,
 then $f'(p) = 0$.
 \item Using (a), prove that a local maximum of a smooth function $f:M \to \R$ is a
 critical point of $f$.
\end{enumerate}
\end{prob}

\begin{prob}
 Let $f: \R^2 \to \R$ be a smooth function.  Prove that its graph $\Gamma(f) := \{(x,y,f(x,y)) \in \R^3 \}$
is a regular submanifold of $\R^3$.
\end{prob}






\end{document}