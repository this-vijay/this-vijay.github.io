
\documentclass{amsart}
%\usepackage{cmbright}
\usepackage{amsmath, amsfonts, amssymb, amscd}
%\usepackage{tableau}
\usepackage{color}
\usepackage{xcolor}
\usepackage{graphicx}
\usepackage{array}
\usepackage{mathtools}
\usepackage{multirow}
\usepackage{framed}
\usepackage{tikz}
\usetikzlibrary{matrix,arrows}
\usepackage[square,sort,comma,numbers]{natbib}
\usepackage{enumerate}




%\oddsidemargin.5cm
%\evensidemargin.5cm
\addtolength{\oddsidemargin}{-.525in}
\addtolength{\evensidemargin}{-.525in}
\addtolength{\textwidth}{1in}

\addtolength{\topmargin}{-.87in}
\addtolength{\textheight}{1.5in}




\newcommand\coolunder[2]{\mathrlap{\smash{\underbrace{\phantom{%
    \begin{matrix} #2 \end{matrix}}}_{\mbox{$#1$}}}}#2}

%\input{tableau}


\newcommand{\+}[1]{\ensuremath{\mathbf{#1}}}
\newcommand{\vect}[1]{\boldsymbol{#1}} % Uncomment for BOLD vectors.
%\newcommand{\vect}[1]{\vec{#1}} % Uncomment for ARROW vectors.
\newcommand{\C}{{\mathbb C}}
\newcommand{\Z}{{\mathbb Z}}
\renewcommand{\P}{{\mathbb P}}
\newcommand{\OG}{\operatorname{OG}}
\newcommand{\OF}{\operatorname{OF}}
\newcommand{\bull}{{\scriptscriptstyle \bullet}}
\newcommand{\la}{\lambda}
\newcommand{\euler}[1]{\chi_{_{#1}}}
\newcommand{\cO}{{\mathcal O}}
\newcommand{\cG}{{\mathcal G}}
\newcommand{\cQ}{{\mathcal Q}}
\newcommand{\R}{{\mathbb R}}
\newcommand{\wt}{\widetilde}
\newcommand{\diag}{\operatorname{diag}}
\newcommand{\comp}{\operatorname{comp}}
\newcommand{\comment}[1]{}
\newcommand{\type}{\mathfrak{t}}
\newcommand{\op}{\text{op}}
\newcommand{\row}{{\bf r}}
\newcommand{\col}{{\bf c}}
\newcommand{\sym}{\mathfrak{S}}
\newcommand{\codim}{\text{codim}}
\DeclarePairedDelimiter{\ceil}{\lceil}{\rceil}
\DeclarePairedDelimiter{\floor}{\lfloor}{\rfloor}
\renewcommand{\emptyset}{\varnothing}

\newtheorem{thm}{Theorem}
\newtheorem{lemma}[thm]{Lemma}
\newtheorem{prop}[thm]{Proposition}
\newtheorem{cor}[thm]{Corollary}

\theoremstyle{definition}
\newtheorem{definition}[thm]{Definition}
\newtheorem{example}[thm]{Example}
\newtheorem{conj}[thm]{Conjecture}
\newtheorem{obs}[thm]{Observation}
\newtheorem{fact}[thm]{Fact}
\newtheorem{remark}[thm]{Remark}
\newtheorem{prob}{Problem}
\newtheorem{chal}{Challenge}

\begin{document}
\title{Final Exam}
\date{November 18, 2016}
\author{Introduction to Manifolds}

\maketitle


The exam consists of five questions  worth a total of $100$ points.
You may take up to three hours to complete the exam.


\begin{prob}[$20$ pts]
 Let $x,y,z$ be the standard coordinates on $\R^3$.  A plane in $\R^3$
 is vertical if it is defined by 
 $ax + by = 0$ for some $(a,b) \neq (0,0)$.
 Prove that restricted to a vertical plane,
 $dx \wedge dy = 0$.
\end{prob}


\begin{prob}[$20$ pts]
Determine whether each of the following statements is true or false.  If false, provide a counterexample.  If true, briefly justify your answer.
\begin{enumerate}[(a)]
 \item Let $F: N \to M$ be a one-to-one immersion, and let $n$ and $m$ denote the dimensions of $N$ and $M$ respectively.
 For any point $q \in F(N) \subset M$, there exists a coordinate neighborhood $(U, x^1, \ldots, x^m)$ about $q$ such that
 $F(N) \cap U$ is defined by the vanishing of $x^1, \ldots, x^{m-n}$.
 \item Any alternating bilinear map $\omega \in A_2(\R^3)$ is \emph{decomposable}, 
 in the sense that $\omega = \alpha_1 \wedge \alpha_2$ for some
 covectors $\alpha_1, \alpha_2 \in V^*$.
\end{enumerate}
\end{prob}





\begin{prob}[$20$ pts]
Let $M$ be a smooth manifold.  Recall
that the Lie bracket of two smooth vector fields is
again a smooth vector field.  Is the Lie bracket $C^{\infty}(M)$-linear in both variables?  Justify your answer.
\end{prob}





\begin{prob}[$20$ pts]
Let $M$ be a smooth manifold.  Let $f: M \to \R$ be a smooth function, and
suppose $c$ is a regular value of $f$.
Prove that $f^{-1}([c,\infty))$ is a smooth manifold with boundary. 
\end{prob}


\begin{prob}[$20$ pts]
 Suppose $p$ is a point in a compact oriented surface $M$ without boundary, and
 that $i: C \to M \setminus \{p\}$ is the inclusion of a small circle around the puncture.
 Prove that the restriction map $i^* : H^1(M \setminus \{p\}) \to H^1(C)$ is the zero map. (Hint: You may need to use Stokes' Theorem).
\end{prob}


%\begin{prob} [$20$ pts]
% Suppose $M$ is a smooth manifold with finite dimensional de Rham cohomology vector spaces $H^k(M)$.
% We define the Euler characteristic of $M$ to be 
% \[
% \chi(M) = \sum^{n}_{k=0} (-1)^k \text{dim}H^k(M).
%\]
%If $\{U,V\}$ is an open cover of $M$ such that $U$, $V$, and $U \cap V$
%all have finite dimensional cohomology, prove that
%\[
%\chi(M) - \chi(U) - \chi(V) + \chi(U \cap V) = 0.
%\]
%\end{prob}





\end{document}